%================================================================================
%  TEMPLATE DE RAPPORT DE STAGE – ENIB + DEAL
%================================================================================

\documentclass[12pt,a4paper]{report}

% ------------------------------------------
% PACKAGES
% ------------------------------------------
\usepackage[utf8]{inputenc}
\usepackage[T1]{fontenc}
\usepackage[main=french,provide=*]{babel}
\usepackage{microtype}

\usepackage[none]{hyphenat}
\hyphenation{imports}


\usepackage{graphicx}
\usepackage{lmodern}
\usepackage{geometry}

\geometry{
  left=2cm,
  right=2cm,
  top=2cm,
  bottom=3.5cm
}

\usepackage[raggedright]{titlesec}
\usepackage{setspace}
\usepackage{fancyhdr}
\usepackage{color}
\usepackage{tocloft}

\renewcommand{\cftchapfont}{\raggedright\bfseries\color{ENIBblue}}
\renewcommand{\cftsecfont}{\raggedright}
\renewcommand{\cftsubsecfont}{\raggedright}


% Sommaire compact
\setlength{\cftbeforechapskip}{0.3em}
\setlength{\cftbeforesecskip}{0.1em}
\setlength{\cftbeforesubsecskip}{0pt}


\usepackage{float}

% --- Glossaire ---
\usepackage[acronym]{glossaries}
\makeglossaries
\setacronymstyle{short-long}
% --- Glossaire des sigles ---
\newacronym{deal}{DEAL}{Direction de l’Environnement, de l’Aménagement et du Logement}
\newacronym{seb}{SEB}{Service Eau et Biodiversité}
\newacronym{ubio}{UBIO}{Unité Biodiversité}
\newacronym{enib}{ENIB}{École Nationale d’Ingénieurs de Brest}

\newacronym{crs}{CRS}{Communication Réseau Système}
\newacronym{cai}{CAI}{Conception d'Applications Interactives}

\newacronym{sinp}{SINP}{Système d’Information de l’iNventaire du Patrimoine naturel}
\newacronym{ofb}{OFB}{Office Français de la Biodiversité}
\newacronym{seor}{SEOR}{Société d’Études Ornithologiques de La Réunion}
\newacronym{ifremer}{Ifremer}{Institut Français de Recherche pour l’Exploitation de la Mer}

\newacronym{quadrige}{Quadrige}{Base nationale Ifremer pour le suivi du milieu marin et littoral}
\newacronym{geonature}{Géonature}{Application libre de gestion des données naturalistes}

\newacronym{eee}{EEE}{Espèces Exotiques Envahissantes}
\newacronym{erc}{ERC}{Éviter – Réduire – Compenser}
\newacronym{rse}{RSE}{Responsabilité Sociétale des Entreprises}




% --- Empêche printglossary de créer un saut de page + un titre automatique ---
\renewcommand*{\glossarysection}[2][]{%
    % Ne rien faire : pas de saut de page, pas de section automatique
}

% ------------------------------------------------------------------
% Style personnalisé du glossaire : ENIBStyle
% ------------------------------------------------------------------
\newglossarystyle{enibstyle}{
  \setglossarystyle{long}

  % Désactive toute numérotation de pages en fin d'entrée
  \renewcommand*{\glspostdescription}{}

  % Style : acronyme en bleu ENIB
  \renewcommand*{\glsnamefont}[1]{\textbf{\color{ENIBblue}##1}}

  % Structure du tableau (2 colonnes)
  \renewenvironment{theglossary}{
    \begin{longtable}{@{}p{0.25\textwidth}p{0.72\textwidth}@{}}
  }{
    \end{longtable}
  }

  % Chaque entrée du glossaire (sans lignes)
  \renewcommand{\glossentry}[2]{%
    \glsentryitem{##1}%
    \glstarget{##1}{\glossentryname{##1}} &
    \glossentrydesc{##1}%
    \tabularnewline
  }
}



\setstretch{1.55}

\hyphenpenalty=500
\exhyphenpenalty=50


% Empêcher l'interligne d'affecter les titres
\usepackage{etoolbox}
\makeatletter
\pretocmd{\chapter}{\singlespacing}{}{}
\pretocmd{\section}{\singlespacing}{}{}
\pretocmd{\subsection}{\singlespacing}{}{}
\makeatother



\setlength{\parskip}{0.6em}
\setlength{\parindent}{0pt}



% Correction du warning fancyhdr
\setlength{\headsep}{22pt}  % valeur par défaut : 25pt
\setlength{\headheight}{42pt}
\setlength{\footskip}{34pt}
\addtolength{\topmargin}{-2pt}

% ------------------------------------------
% COULEURS
% ------------------------------------------
\definecolor{ENIBblue}{RGB}{5,55,105}



% ------------------------------------------
% STYLE TITRES (harmonisé)
% ------------------------------------------

% ===== CHAPITRES NUMÉROTÉS =====
\titleformat{\chapter}
  {\Huge\bfseries\color{ENIBblue}}
  {\thechapter.\ }{0.75em}{}
\titlespacing*{\chapter}{0pt}{1em}{1.2em}

% Amélioration : éviter page blanche inutile avant le premier chapitre
\preto\chapter{%
  \ifnum\value{page}>1\clearpage\fi
}



% ===== CHAPITRES NON NUMÉROTÉS =====
\newcommand{\BlueChapter}[1]{%
  \clearpage
  \phantomsection
  \vspace*{0.6cm} % <-- ESPACE ENTRE L’EN-TÊTE ET LE TITRE
  {\Huge\bfseries\color{ENIBblue}\raggedright #1}\par  
  \addcontentsline{toc}{chapter}{#1}%
  \vspace{1.5em}
}


% ===== SECTIONS =====
\titleformat{\section}
  {\large\bfseries\color{black}}
  {\thesection.\ }{0.75em}{}
\titlespacing*{\section}{0pt}{1.5em}{1.2em}



% ===== SOUS-SECTIONS =====
\titleformat{\subsection}
  {\normalsize\bfseries\color{black}}
  {\thesubsection.\ }{0.75em}{}
\titlespacing*{\subsection}{0pt}{1em}{0.8em}

\setlength{\textfloatsep}{2em}
\setlength{\floatsep}{1.6em}
\setlength{\intextsep}{1.6em}

% ===== CHAPITRE ANNEXES (version compacte) =====
\newcommand{\BlueChapterAnnexes}[1]{%
  \clearpage
  \phantomsection
  {\Huge\bfseries\color{ENIBblue}\raggedright #1}\par
  \addcontentsline{toc}{chapter}{#1}%
  \vspace{0.8em} % <-- volontairement réduit
}

\usepackage{tikz}
\usetikzlibrary{shapes.geometric, arrows.meta, positioning}

    
\usepackage[colorlinks=true,
            linkcolor=black,
            urlcolor=ENIBblue,
            citecolor=ENIBblue]{hyperref}

\usepackage[normalem]{ulem} % pour \uline sans casser \emph

\usepackage{needspace}
\pretocmd{\section}{\needspace{5\baselineskip}}{}{}
\pretocmd{\subsection}{\needspace{4\baselineskip}}{}{}


\let\oldhref\href
\renewcommand{\href}[2]{\oldhref{#1}{\uline{#2}}}




% ------------------------------------------
% TABLE DES MATIÈRES — BLEU
% ------------------------------------------
\renewcommand{\cfttoctitlefont}{\Huge\bfseries\color{ENIBblue}}
\renewcommand{\cftloftitlefont}{\Huge\bfseries\color{ENIBblue}}

\renewcommand{\cftchapfont}{\bfseries\color{ENIBblue}}
\renewcommand{\cftchappagefont}{\bfseries\color{ENIBblue}}



% Style du titre de la table des matières
\renewcommand{\contentsname}{\color{ENIBblue}\LARGE\bfseries Table des matières}

\setlength{\cftbeforetoctitleskip}{0.5em}
\setlength{\cftaftertoctitleskip}{1em}
% Style du titre de la liste des figures
\renewcommand{\listfigurename}{\color{ENIBblue}\Huge\bfseries Liste des figures}

% Style français des annexes
\addto\captionsfrench{%
  \renewcommand{\appendixname}{Annexe}
}

% ------------------------------------------
% INFORMATIONS DU RAPPORT
% ------------------------------------------
\newcommand{\StudentName}{Basile André}
\newcommand{\StudentEmail}{b1andre@enib.fr}
\newcommand{\StageShortTitle}{Développement Géonature}
\newcommand{\StageTitle}{Développement d'un module externe de l'application Géonature pour importer les données de Quadrige}
\newcommand{\DatesStage}{Septembre – Décembre 2025}
\newcommand{\Entreprise}{DEAL Réunion}
\newcommand{\TuteurEntr}{Rémi Bouilly}
\newcommand{\TuteurAcad}{Jean-François Favennec}
\newcommand{\SignatureTuteurEntr}{images/autres/remi_bouilly.png}

% ------------------------------------------
% PAGE DE GARDE (COMPLÈTE)
% ------------------------------------------
\newcommand{\PageDeGarde}{
\begin{titlepage}
\thispagestyle{empty}

\noindent
\begin{minipage}[t][0.48\textheight][t]{\textwidth}
    \includegraphics[width=\textwidth, height=0.48\textheight]{images/logos/Corail-ile-de-la-reunion.jpg}

    \vspace*{-0.48\textheight}
    \begin{minipage}[t]{\textwidth}
        \vspace{0.6cm}

        \begin{minipage}{0.5\textwidth}
            \hspace{0.2cm}
            \includegraphics[height=1.7cm]{images/logos/logo_DEAL.png}
        \end{minipage}
        \begin{minipage}{0.48\textwidth}
            \raggedleft
            {\Large\color{white}\DatesStage\hspace{0.2cm}}
        \end{minipage}

        \vspace{1.6cm}
        \begin{minipage}{0.6\textwidth}
            \raggedright
            \hspace{0.2cm}{\Large\bfseries\color{white}\Entreprise}\\
            \hspace{0.2cm}{\Large\bfseries\color{white}ENIB}
        \end{minipage}

        \vspace{1.8cm}
        \begin{center}
            {\fontsize{28}{32}\selectfont\bfseries\color{white}\StageShortTitle}
        \end{center}

    \end{minipage} 
\end{minipage}   

\vspace{1.8cm}
\begin{minipage}{0.6\textwidth}
    \textbf{Stagiaire :} \StudentName\\[0.25cm]

    \textbf{Tuteur entreprise :} \TuteurEntr\\[0.25cm]

    \textbf{Tuteur académique :} \TuteurAcad\\[-0.1cm]

    \hspace{1.8cm}%
    \includegraphics[height=3cm]{\SignatureTuteurEntr}
\end{minipage}

\vfill

\begin{center}
    \begin{minipage}{0.25\textwidth}
        \centering
        \includegraphics[height=1.1cm]{images/logos/Enib_inp_2025.png}
    \end{minipage}
    \begin{minipage}{0.45\textwidth}
        \centering
        \rule{\textwidth}{0.4pt}
    \end{minipage}
    \begin{minipage}{0.25\textwidth}
        \raggedleft {\small \StudentEmail}
    \end{minipage}

    \vspace{0.25cm}
    {\Large\bfseries \StudentName}
\end{center}

\end{titlepage}
}


% ------------------------------------------
% EN-TÊTE + PIED DE PAGE
% ------------------------------------------
\renewcommand{\headrulewidth}{0.8pt}
\renewcommand{\headrule}{\color{ENIBblue}\rule{\headwidth}{0.8pt}}

\fancypagestyle{standard}{
    \fancyhf{}

    \fancyhead[L]{\includegraphics[height=0.9cm]{images/logos/Enib_inp_2025.png}}
    \fancyhead[R]{%
  \raisebox{-0.2\height}{%
    \includegraphics[height=1.6cm]{images/logos/logo_DEAL.png}%
  }%
}


    \fancyfoot[C]{\color{ENIBblue}\rule{\textwidth}{0.4pt}}
    \fancyfoot[L]{\small\color{ENIBblue} \StudentEmail}
    \fancyfoot[C]{\small\color{ENIBblue} Stage Assistant Ingénieur\\[-0.2em]\textbf{\StageShortTitle}}
    \fancyfoot[R]{\small\color{ENIBblue} \thepage}
}

\makeatletter
\let\ps@plain\ps@standard
\makeatother

% ===== ENVIRONNEMENT TEXTE LIMINAIRE (remerciements, résumés) =====
\newenvironment{frontmattertext}{%
  \setstretch{1.3}%
  \setlength{\parskip}{0.4em}%
}{}


%===============================================================================
% DOCUMENT
%===============================================================================

% Styles et mise en page

\begin{document}

\PageDeGarde

\pagestyle{standard}
\setcounter{page}{1}

% début du contenu 
% ------------------------------------------

\BlueChapter{Remerciements}
\begin{frontmattertext}
MRC AU CHAT ET A ZIZOU
DEDICACE A PERSONNE FALLAIT ETRE LA
\end{frontmattertext}

\BlueChapter{Résumé (FR)}
\begin{frontmattertext}
\noindent
Au cours de ma formation d’ingénieur généraliste à l’\gls{enib}, j’ai eu l’opportunité d’explorer
des domaines variés, dont l’informatique, le traitement des données et la communication réseau.
Ces enseignements, combinés au contexte environnemental actuel, m’ont donné l’envie d’explorer
comment les outils numériques peuvent répondre à des enjeux concrets, en particulier ceux liés
à l’environnement et au milieu marin, un domaine qui m’inspire depuis longtemps.
C’est cette motivation qui m’a conduit à choisir la \gls{deal},
où le développement d’outils
informatiques joue un rôle clé dans la valorisation et l’accessibilité des données,
offrant une occasion idéale d’allier mes compétences techniques à un sujet qui a du
sens.

\par\medskip

J’ai effectué mon stage au sein de la \gls{deal} Réunion, plus précisément
au \gls{seb}, à l’\gls{ubio}.
L’île de La Réunion présente des enjeux importants de suivi et de préservation de la biodiversité, en raison de la singularité de ses écosystèmes et des pressions 
qu’ils subissent (notamment les \gls{eee} et le développement du territoire). Ce contexte renforce la nécessité d’un suivi rigoureux des données naturalistes. 
\medskip

La \gls{deal} utilise \gls{geonature} pour centraliser et gérer ces données.
Le sujet principal de mon stage consistait à développer un module externe permettant d’y importer automatiquement les données marines issues de \gls{quadrige}. 
Ces données, essentielles pour le suivi environnemental littoral, 
représentent un enjeu majeur pour les missions de la DEAL.

\medskip

N'ayant pas suivi le module \gls{cai}, ce stage m’a permis d’acquérir des compétences complémentaires en développement
logiciel et applicatif, à travers la conception d’un module d’import, la manipulation de données et l’interfaçage de
systèmes hétérogènes, afin d’obtenir un outil simple d’utilisation, robuste et maintenable.

\medskip

Il m’a également offert une immersion dans le fonctionnement d’une administration publique de taille moyenne,
où l’alternance entre travail autonome et collaboration avec les agents en charge des données naturalistes et de
l’administration de \gls{geonature} s’est révélée particulièrement enrichissante. Cette expérience en environnement
professionnel réel a renforcé ma compréhension des enjeux liés à la gestion des données environnementales et a affiné
ma capacité à questionner la pertinence, l’utilité et le sens des projets auxquels je contribue, afin de m’assurer qu’ils
soient en accord avec mes valeurs et porteurs d’un impact positif.
\end{frontmattertext}

\BlueChapter{Abstract (EN)}
\begin{frontmattertext}
\noindent
As part of my general engineering studies at \gls{enib}, I had the opportunity to explore
a wide range of fields, including computer science, 
 
data processing and network communications. These courses, 
combined with the current environmental context, strengthened my interest in understanding 
how digital tools can address real-world challenges — particularly those related to the environment 
and the marine domain, which has long inspired me. This motivation led me to choose \gls{deal}, where 
the development of digital tools plays a key role in enhancing the value and accessibility of 
environmental data, offering an ideal opportunity to align my technical skills with a topic that 
holds meaning for me.

\par\medskip

I completed my internship at \gls{deal} Réunion, more specifically within the \gls{seb} and the \gls{ubio}. 
Réunion Island faces major biodiversity challenges due to its high proportion of endemic 
species and the presence of particularly sensitive natural habitats. These ecosystems are under strong pressure, 
notably from \gls{eee} and ongoing land-use development, which reinforces the need for rigorous naturalist data.
\gls{deal} uses \gls{geonature} to centralise and manage these datasets.
The main objective of my internship was to develop an external module for this application, enabling the automated 
import of marine and coastal environmental data produced by \gls{ifremer}. These data are essential for biodiversity
monitoring and are accessed through \gls{quadrige},Ifremer’s dedicated information system. 

\par\medskip

Since I had not taken the \gls{cai} module, this internship allowed me to acquire complementary 
skills in software and application development through the design of an import module, data 
manipulation and the interfacing of heterogeneous systems, with the aim of producing a simple, robust 
and maintainable tool.

\par\medskip

It also offered me insight into the functioning of a medium-sized public administration, where the alternation 
between autonomous work and collaboration with staff in charge of naturalist data and the administration of \gls{geonature} 
proved particularly enriching. This experience in a real professional environment strengthened my understanding of the
challenges associated with environmental data management and refined my ability to reflect on the relevance, usefulness 
and purpose of the projects I contribute to, ensuring that they align with my values and have a positive impact.
\end{frontmattertext}

% ------------------------------------------
% TABLE DES MATIÈRES 
% ------------------------------------------
\clearpage
\phantomsection

\begingroup
  \small
  \setstretch{1.0}        % interligne normal
  \setlength{\parskip}{0pt} % pas d’espace entre entrées
  \tableofcontents
\endgroup

\clearpage


% GLOSSAIRE
\BlueChapter{Glossaire}

\vspace{1.2em}

\begingroup
\setlength{\LTpre}{0pt}
\setlength{\LTpost}{0pt}
\printglossary[type=\acronymtype, style=enibstyle, title={}]
\endgroup

% ------------------------------------------
% CHAPITRES NUMÉROTÉS
% ------------------------------------------

% ------------------------------------------
% PRESENTATION DE L'ENTREPRISE

\chapter{Présentation de l'entreprise}
\section{DEAL Réunion}
La Direction de l’Environnement, de l’Aménagement et du Logement (\gls{deal}) est le service déconcentré de l’État. 
Elle met en œuvre, à l’échelle régionale, les politiques publiques relevant du Ministère de la Transition 
Écologique et de la Cohésion des Territoires, ainsi que du Ministère de la Transition Énergétique. La Réunion 
est un territoire insulaire soumis à de fortes pressions environnementales et à des enjeux d’aménagement complexes. 
Dans ce contexte, la \gls{deal} occupe une place centrale. Elle se situe au croisement des questions d’environnement, 
de biodiversité, d’eau, d’urbanisme et de développement territorial.
\par\medskip

La \gls{deal} Réunion assure l’application des réglementations environnementales. Elle instruit les projets 
d’aménagement et gère les risques naturels. Elle suit la ressource en eau et met en œuvre les politiques de 
protection des milieux naturels. Elle travaille en étroite collaboration avec les collectivités et plusieurs 
établissements publics, tels que l’\gls{ofb}, le \gls{pnrun} ou l’\gls{ifremer}. Elle coopère 
aussi avec de nombreuses associations locales, comme la \gls{seor} ou la \gls{srepen}, qui contribuent activement 
au suivi et à la préservation de la biodiversité insulaire.
\par\medskip

Au sein de cette structure, le Service Eau et Biodiversité (\gls{seb}) porte les missions liées à la préservation 
des milieux aquatiques et terrestres. Il développe la connaissance des espèces, veille à leur protection et régule 
les activités susceptibles d’impacter la biodiversité. Le \gls{seb} se trouve ainsi au cœur des enjeux écologiques 
de l’île.
\par\medskip

Mon stage s’est déroulé au sein de l’Unité Biodiversité (\gls{ubio}). Cette unité est responsable du suivi 
des espèces et des habitats naturels. Elle gère et valorise les données naturalistes et instruit les dossiers
réglementaires liés à la biodiversité. Elle anime également le Système d’Information sur la Nature et les 
Paysages (\gls{sinp}) régional et assure la gestion de la plateforme \glsfirst{borbonica}. L’unité intervient 
aussi sur des thématiques transversales telles que les espèces exotiques envahissantes (\gls{eee}), la séquence 
\gls{erc} ou la diffusion des connaissances naturalistes. L’organisation interne de l’\gls{ubio} est présentée 
en annexe (Fig.~\ref{fig:organigramme-ubio}).
\par\medskip

L’organigramme interne montre une équipe composée de profils scientifiques, techniques et administratifs. 
Ces personnels travaillent de manière complémentaire pour répondre aux enjeux liés à la biodiversité du 
territoire. Mon stage s’inscrit dans cette dynamique, au sein du pôle dédié aux données naturalistes. Il 
contribue à la structuration et à la modernisation des outils numériques utilisés par la \gls{deal}.
\par\medskip

Cette présentation de la \gls{deal} et de son organisation permet de situer le contexte global de mon stage. 
Le chapitre suivant propose un \textbf{diagnostic \gls{rse}} de la structure. Il vise à évaluer ses pratiques 
au regard des enjeux sociaux, environnementaux et organisationnels.
\par\medskip
\section{Diagnostic RSE}
En raison des responsabilités qui lui sont confiées, la \gls{deal} Réunion doit intégrer des considérations sociales, 
environnementales et organisationnelles dans l’ensemble de ses pratiques. Le diagnostic qui suit s’appuie sur la norme 
ISO~26000 et sur les principes de la \gls{rne}, afin d’évaluer la manière dont 
la structure prend en compte ces enjeux.
\subsection{Risques et impacts des activités}
\noindent
\textbf{Impacts environnementaux.}
La \gls{deal} n’engendre pas d’impacts industriels directs. Cependant, ses activités reposent largement sur 
l’utilisation d’outils numériques intensifs : traitement de données naturalistes, alimentation des plateformes 
telles que le \gls{sinp}, \gls{geonature} ou \gls{borbonica}, ainsi que l’import automatisé de données issues de 
l’API \gls{quadrige}, permettant l’accès aux données marines produites par l'\gls{ifremer}. 
Ces pratiques impliquent une consommation énergétique, l’usage d’équipements informatiques et des besoins 
croissants en stockage et en traitement.  
\par\medskip

\textbf{Impacts sociétaux.}
Les décisions publiques s’appuient fortement sur la qualité des données produites et centralisées par la \gls{deal}. Une information environnementale fiable est essentielle pour les collectivités, les bureaux d’études, les associations ou les services de l’État.  
En facilitant l’intégration des données marines issues de \gls{quadrige}, mon travail renforce la transparence, l’égalité d’accès à la connaissance et la capacité des acteurs à prendre des décisions éclairées.
\par\medskip

\textbf{Impacts sociaux.}
Les équipes de la \gls{deal} évoluent dans un environnement de travail pluridisciplinaire mobilisant expertise scientifique, compétences réglementaires et gestion de données.  
Mon immersion au sein de l’\gls{ubio} m’a permis de contribuer à l’amélioration de certains processus internes, notamment par la simplification du flux de données marines. Cette contribution technique a eu pour effet indirect de diminuer la charge de travail liée aux imports manuels, renforçant ainsi l’efficacité opérationnelle du pôle.

\subsection*{Enquête RSE : actions mises en place}

La \gls{deal} engage plusieurs actions structurantes en cohérence avec la norme ISO~26000.

\par\medskip
\textbf{Environnement.}
L’administration encourage la réduction des déplacements, le recours aux outils numériques et une modernisation progressive des systèmes d’information. Mon module s’inscrit dans cette démarche de sobriété numérique en réduisant les traitements redondants et en automatisant les échanges de données.

\par\medskip
\textbf{Social et gouvernance.}
Les conditions de travail, la qualité du dialogue entre services et l’accompagnement des stagiaires constituent des axes importants. Les échanges réguliers au sein du pôle biodiversité ont permis d’adapter l’outil développé aux besoins réels, témoignant d’un fonctionnement concerté et d’une volonté d’amélioration continue.

\par\medskip
\textbf{Sociétal.}
La \gls{deal} contribue directement à la diffusion de données environnementales essentielles au suivi scientifique et aux politiques publiques. Les collaborations avec l'\gls{ofb}, l'\gls{ifremer}, le Parc national ou encore les associations naturalistes renforcent l’ancrage territorial de son action.  
Le module développé participe à cette dynamique en améliorant l’accessibilité et la qualité des données marines, dont dépend une partie de la stratégie environnementale régionale.

\subsection{Stratégie globale en matière de RSE}
\noindent
Bien qu’elle agisse dans un cadre réglementaire strict, la \gls{deal} cherche à renforcer la cohérence et la qualité 
de ses pratiques internes. Elle se trouve dans une dynamique de \textit{pré-conformité active}, allant au-delà des obligations 
minimales en matière de gestion des données, de modernisation numérique et de diffusion de l’information environnementale.

Mon travail s’inscrit dans cette stratégie : l’automatisation de l’import des données de \gls{quadrige} constitue une étape 
vers un système d’information plus robuste, plus efficace et plus aligné avec les principes de responsabilité numérique.

\subsection{Conclusion personnelle}
\noindent
Ce diagnostic montre que la \gls{deal} Réunion intègre progressivement les 
enjeux de la RSE et de la RNE au cœur de ses pratiques, malgré les contraintes 
propres à une administration publique.  
Mon stage m’a permis d’apporter une contribution concrète à cette dynamique, en 
améliorant la gestion des données marines et en facilitant le travail quotidien des agents. 
Cette expérience a renforcé ma conviction que le numérique, lorsqu’il est pensé de manière responsable, 
peut devenir un véritable levier au service de la transition écologique et de l’action publique.



% ------------------------------------------
% Organisation du stage
%-------------------------------------------
\chapter{Organisation du stage}
\section{Contexte du stage}
\noindent
Le \gls{sinp} est un dispositif collectif de mise en partage des données d’observations d’espèces sauvages sur l’île. 
Il a été mis en service à La Réunion en 2018, au travers de la plateforme Borbonica gérée conjointement par 
\gls{pnrun} et la \gls{deal}. 



Borbonica s’appuie sur un portail web, qui permet d’accéder aux différentes interfaces utilisateurs du SINP :
\begin{itemize}
    
    \item Borbonica obs : consultation cartographique des données du SINP. Il s’agit d’une 
    interface web basée sur le plugin Lizmap et reposant sur une base de données PostgreSQL. Cette plateforme offre un accès aux données 
    selon différents niveaux et profils (grand public, experts, validateurs, etc.) ; 
    
    \item Borbonica atlas : consultation des fiches espèces disponibles dans le SINP, sous forme de synthèse des principales informations 
    (observations, documents, photos, etc.). Il s’agit d’un outil issu de la solution open source GéoNature ;
    
    \item Borbonica stats : module statistique qui permet de consulter des tableaux de bord chiffrés sur le contenu de Borbonica 
    (données disponibles, principaux usages, etc.). Ce module s’appuie sur un projet Lizmap, exploitant des données de la base PostgreSQL.

\end{itemize}

Afin de pérenniser le SINP 974, un projet a été lancé en 2023, avec pour principaux objectifs la 
modernisation et la simplification du fonctionnement du système. Plusieurs nouvelles fonctionnalités 
vont être ajoutées, et certaines briques logicielles seront remplacés par de nouvelles solutions plus 
modernes et plus simples à utiliser et à maintenir. C’est le cas notamment de la solution Géonature qui 
est en cours de déploiement pour remplacer le périmètre de Borbonica obs.

\section{Objectifs du stage}
\noindent

À la suite de la modernisation engagée pour le SINP-974 et du déploiement progressif de GeoNature au sein de la \gls{deal}, 
le stage avait pour finalité d’apporter un appui opérationnel au renforcement de la chaîne d’acquisition et de gestion des données naturalistes. 
L’objectif principal était de faciliter l’intégration de données externes dans GeoNature et, plus largement, d’enrichir les informations 
accessibles via la plateforme régionale Borbonica.

Dans ce cadre, plusieurs objectifs spécifiques ont été définis :

\begin{itemize}

    \item développer un module externe GeoNature interfacé avec le système d’information Quadrige de l’\gls{ifremer}, 
        afin d’automatiser la récupération, le filtrage et la préparation des données issues du milieu marin ;

    \item poser les fondations techniques nécessaires à une future intégration des données produites par l’application 
        \gls{plantnet}, en réalisant un client Python et un premier outil de structuration des résultats ;

    \item documenter les procédures d’appel aux API et proposer une organisation des développements garantissant 
    leur maintenabilité et leur évolution dans le temps.

\end{itemize}

En complément du module Quadrige, une seconde orientation du stage concernait la préparation d’un futur connecteur GeoNature dédié aux données issues de \gls{plantnet}. 
Pl@ntNet produit chaque jour un volume important d’observations naturalistes géolocalisées, particulièrement pertinent pour le suivi des \gls{eee} végétales et des dynamiques de végétation.
Afin d’explorer la faisabilité de cette intégration, un travail préliminaire a consisté à :
\begin{itemize}
    \item étudier la documentation de l’API Pl@ntNet v3,
    \item concevoir un client Python robuste, capable d’interroger l’API selon plusieurs critères (taxon, polygone GeoJSON, plage temporelle),
    \item structurer les données extraites au format \textit{Darwin Core} afin d’assurer une compatibilité immédiate avec GeoNature et le SINP-974.
\end{itemize}
Ce travail prépare le développement d’un parser complet destiné au module \texttt{api2gn} de GeoNature.

Ces objectifs s’inscrivent pleinement dans la stratégie portée conjointement par la \gls{deal} Réunion et 
le \gls{pnrun}, visant à fiabiliser et à harmoniser les données du SINP-974 tout en renforçant la 
représentation de thématiques encore peu renseignées, comme le milieu marin. Ils ont guidé l’ensemble des actions 
menées durant le stage et constituent le cadre des travaux détaillés dans les sections suivantes.

\section{Méthodologie de travail}
\noindent

La méthodologie adoptée au cours du stage s’est construite progressivement, en fonction des contraintes techniques 
rencontrées et de l’avancement du projet. Le développement a d’abord été mené en local, faute d’accès immédiat aux 
serveurs du SINP-974. Cette situation a nécessité la mise en place d’un environnement de travail stable, capable de 
reproduire autant que possible les conditions de production. La séparation stricte entre le backend Python/Flask et 
le frontend Angular, conforme à l’architecture modulaire de GeoNature, a facilité cette organisation et permis de 
développer chaque composante de manière indépendante puis intégrée.

L’usage systématique de Git a constitué un élément central de la méthodologie. Il a permis de versionner les 
évolutions, de tester les fonctionnalités étape par étape, puis de transférer les développements vers les serveurs 
de la DEAL dès que l’accès a été ouvert. Ce mode de travail, fondé sur des allers-retours réguliers entre 
l’environnement local et l’environnement distant, a offert une solution efficace pour contourner les limitations 
imposées par la plateforme d’accès à distance utilisée au sein de l’administration.

Parallèlement, une attention particulière a été portée à la configuration de l’environnement de développement. 
Les contraintes liées au proxy ministériel et aux versions spécifiques des outils techniques (Python, Angular, Node)
ont rendu nécessaire la mise en place d’un mécanisme permettant d’adapter automatiquement la configuration réseau 
selon le contexte de connexion. L’utilisation de pyenv a également permis de stabiliser la version de Python utilisée, 
élément indispensable pour assurer la compatibilité avec GeoNature.

Enfin, l’ensemble du travail s’est appuyé sur une démarche itérative, combinant phases de développement, tests exploratoires 
et documentation progressive. Les appels à l’API Quadrige ont d’abord été éprouvés par des tests unitaires avec un client Python local,
avant d’être intégrés au module. De même, les premières interfaces Angular ont fait l’objet de tests fonctionnels 
en local afin de vérifier la cohérence et la fluidité de l’ergonomie.

Cette méthodologie, mêlant adaptation, rigueur technique et cycles courts d’expérimentation, a permis d’assurer la robustesse 
du développement malgré un environnement parfois instable et d’anticiper au mieux les étapes d’intégration dans l’infrastructure du SINP-974.

% ------------------------------------------
% Introduction au fonctionnement de Géonature
% ------------------------------------------
\chapter{Introduction à GeoNature}
\section{Principe général de GeoNature}

GeoNature est une application web dédiée à la gestion, la centralisation et la valorisation des
données naturalistes. Développé initialement en 2010 puis entièrement refondu en 2017, le
projet est aujourd’hui maintenu par le Parc national des Écrins. L’application repose sur une
architecture moderne combinant un backend Python/Flask et un frontend Angular, ce qui lui
permet d’assurer à la fois la saisie, la consultation, la validation et la restitution des données.

\begin{figure}[h!]
    \centering
    \includegraphics[width=0.85\textwidth]{images/back_front_geonature.png}
    \caption{Architecture générale du backend et du frontend de GeoNature}
    \label{fig:backfront}
\end{figure}

Le fonctionnement de GeoNature est modulaire : un noyau applicatif fournit les briques 
communes (API, référentiels, schéma de synthèse), et différents modules viennent étendre les 
fonctionnalités selon les besoins (Occtax, Occhab, Validation, Import, Export, etc.).  
La documentation officielle détaille l’ensemble de cette architecture
\href{https://docs.geonature.fr/}{(documentation officielle)}
ainsi que les sources du projet, disponibles sur 
\href{https://github.com/PnX-SI/GeoNature?tab=readme-ov-file}{GitHub}.


\section{Architecture technique}

GeoNature combine une partie serveur (backend) et une partie cliente (frontend) qui dialoguent
via une API REST.

Le backend est développé en Python à l’aide du framework Flask. Il assure l’ensemble des
traitements métiers, l’accès aux données et la gestion des opérations spatiales via PostgreSQL
et PostGIS. La gestion des utilisateurs et de leurs permissions repose sur UsersHub, qui fournit
l’authentification et le contrôle des accès. L’API exposée par le backend constitue le point
central de communication avec l’interface web.

Le frontend, développé en Angular, constitue l’interface visible par l’utilisateur. Il interroge
exclusivement l’API du backend pour afficher les formulaires, les cartes, les graphiques et les
données attributaires. Des bibliothèques comme Leaflet ou Bootstrap renforcent les
fonctionnalités cartographiques et l’ergonomie générale.  
Cette séparation nette entre backend et frontend garantit la stabilité et la modularité du
système, tout en facilitant les évolutions futures.


\section{Organisation interne et modules}

L’application est structurée autour de modules fonctionnels s’appuyant tous sur les mêmes
référentiels : taxonomie (TaxHub), nomenclatures, utilisateurs (UsersHub) et schéma de
synthèse. Les modules principaux — Occtax pour les observations, Occhab pour les habitats,
ou encore la Validation — s’intègrent directement au cœur applicatif.

Chaque module possède son propre schéma de base de données, son API et ses composants
Angular. Cette organisation modulaire permet de faire évoluer GeoNature, d'ajouter de
nouveaux protocoles d’acquisition ou de développer des extensions externes.  
La documentation décrit précisément les bonnes pratiques et l’architecture à respecter pour
développer un \href{https://docs.geonature.fr/development.html\#developper-un-module-externe}{module GeoNature externe}.


% ------------------------------------------
% Developpement du module d'import Quadrige
% ------------------------------------------
\chapter{Developpement des modules d'\allowbreak imports de données}
\section{Contexte du projet}

Le \gls{sinp}, déployé à La Réunion à travers la plateforme Borbonica, constitue le dispositif régional de référence pour la centralisation, 
la gestion et la diffusion des données naturalistes. Depuis sa mise en service, il a permis de structurer et de valoriser un volume important 
d’observations, issues de sources multiples et couvrant un large éventail de taxons.

La dynamique de peuplement de la base reflète toutefois les modalités historiques et opérationnelles de production des données 
naturalistes sur le territoire. Les observations actuellement intégrées concernent majoritairement les milieux terrestres et 
dulçaquicoles, qui bénéficient de réseaux d’acteurs structurés, de protocoles d’acquisition largement diffusés et d’outils de 
saisie directement compatibles avec le SINP.

À l’inverse, les données issues du milieu marin demeurent plus faiblement représentées. Les indicateurs de contenu disponibles 
en 2024 montrent que les observations relatives aux taxons marins ne constituent qu’une part marginale des données référencées. 
Cette situation ne traduit pas une absence de connaissances ou de suivis en milieu marin, mais résulte principalement de la 
complexité des chaînes d’acquisition, de structuration et de diffusion propres aux données marines, historiquement gérées au sein 
de systèmes d’information spécialisés.

Dans ce contexte, l’enjeu principal n’était pas de refondre les outils existants, mais de renforcer progressivement la représentation 
des données marines au sein du \gls{sinp}, en s’appuyant sur des sources de données déjà structurées, pérennes et reconnues 
institutionnellement. Parmi celles-ci, le système d’information \gls{quadrige}, maintenu par l’Ifremer, occupe une place centrale. 
\gls{quadrige} regroupe un volume important de données environnementales et biologiques collectées en milieu marin, issues de programmes 
de surveillance, d’observation et de recherche conduits sur le long terme.

L’ouverture récente d’une \gls{api} \gls{graphql} par l’Ifremer constitue à ce titre une évolution majeure. Elle permet une interrogation 
directe, fine et structurée des données de \gls{quadrige}, en ciblant précisément les programmes, les zones géographiques, les périodes 
temporelles et les types d’observations d’intérêt. Le fonctionnement de l’API Quadrige et les modalités d’extraction des données sont décrits 
dans la documentation officielle mise à disposition par 
l’Ifremer \href{https://quadrige-core.ifremer.fr/api/extraction/doc?doc=standard&lang=fr&name=result&type=standard}{Documentation officielle de l’API Quadrige}.


Cette évolution ouvre la voie à une exploitation plus systématique de ces données 
par des acteurs institutionnels tels que la DEAL, et à leur mobilisation progressive dans le cadre du \gls{sinp}.

C’est dans ce cadre que s’inscrit le projet développé durant le stage. L’objectif n’était pas de réaliser une intégration 
directe et entièrement automatisée des données \gls{quadrige} dans la base GeoNature, mais de concevoir un module externe permettant 
d’explorer, d’extraire et de préparer ces données de manière structurée, traçable et reproductible. Le module développé interroge 
l’\gls{api} \gls{quadrige} afin d’identifier les programmes pertinents pour le territoire réunionnais, puis permet d’en extraire les observations 
associées selon des critères définis par l’utilisateur.
Les résultats de ces extractions ne prennent pas la forme de fichiers immédiatement importables dans GeoNature. 
Pour chaque programme sélectionné, le module génère une archive compressée contenant plusieurs éléments complémentaires : 
un fichier \gls{csv} brut correspondant aux données extraites, un fichier \gls{json} décrivant précisément les filtres appliqués lors 
de l’appel à l’\gls{api}, ainsi qu’un fichier README documentant le déroulement de l’export et les éventuelles anomalies rencontrées. 
Le module conserve également un accès aux trois derniers exports réalisés afin d’en faciliter la consultation et la réutilisation.

En complément de ces archives, le module expose également des liens directs vers les fichiers CSV générés lors de la phase d’extraction 
des programmes. Deux niveaux de fichiers sont distingués :
– un CSV brut issu directement de la réponse de l’\gls{api} \gls{quadrige}, contenant l’ensemble des instances des programmes dont au moins une 
occurrence est localisée sur le territoire ciblé ;
– un CSV filtré géographiquement, obtenu par un traitement a posteriori à l’aide de la bibliothèque pandas, permettant d’exclure 
les instances dont le code de monitoring location ne correspond pas au territoire d’intérêt (par exemple, codes ne commençant pas 
par le préfixe « 126- » pour La Réunion).

Ce filtrage complémentaire est nécessaire car, lors de l’extraction des programmes, la requête adressée à l’\gls{api} \gls{quadrige} sélectionne 
l’ensemble des programmes possédant au moins une instance sur le territoire demandé, mais retourne également les autres instances associées 
à ces programmes, y compris celles situées hors du périmètre géographique ciblé.

Cette approche intermédiaire répond à plusieurs objectifs. Elle garantit d’une part la traçabilité complète des extractions réalisées 
depuis \gls{quadrige}, en conservant une description explicite des paramètres et traitements appliqués. Elle offre d’autre part une souplesse 
d’usage, en laissant aux administrateurs et gestionnaires de données la possibilité de contrôler, d’analyser et, le cas échéant, d’adapter 
les fichiers produits avant toute intégration dans GeoNature ou Borbonica. Le module s’inscrit ainsi comme une brique préparatoire, destinée 
à sécuriser et à faciliter l’intégration future des données marines dans le \gls{sinp}.

Ce développement s’inscrit pleinement dans la dynamique d’amélioration continue portée par la DEAL Réunion et ses partenaires. 
Il vise à renforcer progressivement la place des données issues du milieu marin au sein du système régional, tout en respectant 
les contraintes techniques, méthodologiques et organisationnelles propres aux outils existants.



\section{Périmètre fonctionnel}

Le module d’import devait couvrir l’ensemble de la chaîne d’acquisition : de la découverte des
programmes \gls{quadrige} jusqu’à la production d’un fichier structuré pour GeoNature.  
La première étape consistait à interroger l’\gls{api} en mode authentifié afin d’obtenir la liste des
programmes disponibles pour un utilisateur donné. Un filtrage automatique sur un périmètre
géographique — principalement La Réunion, mais extensible à d’autres territoires comme les Îles
Éparses — permettait d’isoler les programmes pertinents. Une interface dédiée intégrée à
GeoNature offrait ensuite la possibilité de rechercher des programmes, d’affiner l’affichage par
mots-clés et de sélectionner ceux à importer.

Une fois les programmes choisis, l’utilisateur pouvait définir les filtres à appliquer aux données
(la période d'intérêt, les champs souhaités, ou encore la reprise des stations déjà extraites).  
Le module interrogeait alors l'api
pour récupérer les observations correspondantes 
Seuls les champs utiles au modèle \gls{sinp} étaient extraits : identifiants des programmes et stations,
localisation géographique, taxon observé, date, ainsi que les métadonnées essentielles (auteur,
organisme, méthode d’acquisition, etc.). Une transformation était appliquée pour obtenir une
structure compatible avec les mécanismes d’import de GeoNature.

Enfin, le module produisait un fichier CSV intermédiaire, destiné à être importé via
l’infrastructure existante de GeoNature. Chaque opération d’import était consignée dans un
historique affiché dans un second onglet, permettant de suivre les actions réalisées, leur date,
leur statut et les éventuelles erreurs rencontrées. L’ensemble du module était réservé aux
administrateurs, conformément aux pratiques habituelles de contrôle des imports dans GeoNature.

\section{Conception technique du module d’import \gls{quadrige}}

Le module \gls{quadrige} développé durant le stage a été conçu comme un module externe à GeoNature, 
respectant l’architecture recommandée par le projet tout en prenant en compte les contraintes 
spécifiques de l’\gls{api} \gls{graphql} mise à disposition par l’Ifremer. L’objectif principal était de 
proposer une chaîne d’extraction robuste, traçable et exploitable, sans perturber le cœur 
applicatif de GeoNature ni les processus existants de gestion des données naturalistes.

\subsection{Architecture générale et découplage frontend/backend}

Le module repose sur une architecture client--serveur classique, intégrée à GeoNature via un 
\textit{blueprint} Flask côté backend et un module Angular dédié côté frontend.  
Le backend est responsable de l’ensemble des interactions avec l’\gls{api} \gls{quadrige}, incluant 
l’authentification, la construction des requêtes \gls{graphql}, le suivi des extractions, la gestion 
des fichiers produits et l’exposition de routes REST sécurisées.

Le frontend se limite volontairement au pilotage des opérations : sélection des paramètres, 
lancement des extractions et visualisation des résultats. Il n’accède jamais directement à 
l’\gls{api} \gls{quadrige} ni aux paramètres sensibles, ce qui permet de renforcer la sécurité globale du 
dispositif.

Ce découplage permet :
\begin{itemize}
  \item de centraliser les accès à l’\gls{api} \gls{quadrige} et les jetons d’authentification ;
  \item de limiter la surface d’exposition des données sensibles ;
  \item de faciliter la maintenance et les évolutions indépendantes du backend et du frontend.
\end{itemize}

\subsection{Gestion de la configuration et des paramètres sensibles}

Le module s’appuie sur un fichier de configuration dédié, chargé côté backend via le mécanisme 
standard de GeoNature (\texttt{geonature/config}). Ce fichier centralise l’ensemble des paramètres 
techniques nécessaires au fonctionnement du module, notamment :
\begin{itemize}
  \item l’URL de l’\gls{api} \gls{graphql} \gls{quadrige} ;
  \item le jeton d’authentification requis pour les appels à l’\gls{api} ;
  \item des paramètres métiers optionnels, tels que les localisations suggérées ou les champs 
  extractibles.
\end{itemize}

Une route backend permet d’exposer cette configuration au frontend de manière contrôlée, afin 
de rendre l’interface dynamique et adaptable. Certaines données métiers restent actuellement 
définies côté frontend, mais l’architecture mise en place permettrait de les centraliser 
entièrement côté backend dans une version ultérieure, sans remise en cause du fonctionnement 
global.

\subsection{Extraction des programmes \gls{quadrige}}

La première étape du processus consiste à identifier les programmes \gls{quadrige} pertinents pour un 
territoire donné. Cette extraction repose sur une requête \gls{graphql} spécifique fournie par l’\gls{api} 
\gls{quadrige}, filtrée à partir d’un préfixe de \textit{monitoring location} correspondant au périmètre 
géographique ciblé.

L’\gls{api} \gls{quadrige} retournant l’ensemble des instances associées à un programme dès lors qu’au moins 
une station correspond au critère de recherche, un traitement complémentaire est appliqué côté 
backend. Les fichiers CSV bruts générés par l’\gls{api} sont filtrés a posteriori à l’aide de la 
bibliothèque \texttt{pandas}, afin de ne conserver que les programmes effectivement localisés sur 
le territoire demandé.

Cette étape permet de produire :
\begin{itemize}
  \item un CSV brut, conservé à des fins de traçabilité et de vérification ;
  \item un CSV filtré, utilisé pour l’affichage et la sélection des programmes dans le frontend.
\end{itemize}

Les métadonnées associées à chaque extraction (localisation, horodatage, filtre utilisé) sont 
sauvegardées afin de permettre la reprise des extractions et la consultation des résultats 
antérieurs.

\subsection{Extraction des données et gestion asynchrone des traitements}

L’extraction des données repose sur un mécanisme asynchrone propre à l’\gls{api} \gls{quadrige}. Pour chaque 
programme sélectionné, une requête \gls{graphql} est soumise afin de lancer un job d’extraction côté 
serveur Ifremer. Le module implémente un mécanisme de \textit{polling} permettant de suivre 
l’évolution de l’état de chaque extraction (PENDING, RUNNING, SUCCESS, WARNING, ERROR).

Les extractions sont gérées de manière indépendante, ce qui permet :
\begin{itemize}
  \item de paralléliser les traitements sur plusieurs programmes ;
  \item de gérer finement les échecs ou avertissements retournés par l’\gls{api} ;
  \item d’éviter un blocage global en cas d’échec partiel.
\end{itemize}

À l’issue des traitements, les fichiers ZIP générés par l’\gls{api} \gls{quadrige} sont téléchargés et stockés 
temporairement sur le serveur GeoNature. Chaque fichier est renommé selon une convention 
explicite intégrant le programme, la localisation et l’horodatage, garantissant ainsi sa 
traçabilité.

\subsection{Gestion des états, des erreurs et de la traçabilité}

Un soin particulier a été apporté à la gestion des états et des erreurs tout au long du processus 
d’extraction. Les situations suivantes sont explicitement prises en compte :
\begin{itemize}
  \item erreurs de communication avec l’\gls{api} \gls{graphql} ;
  \item délais excessifs lors des extractions longues ;
  \item incohérences ou réponses partielles retournées par l’\gls{api} ;
  \item extractions aboutissant à des avertissements ou à l’absence de données.
\end{itemize}

Chaque extraction génère un retour structuré indiquant son statut, les avertissements éventuels 
et les liens vers les fichiers produits. Ces informations sont transmises au frontend afin de 
permettre à l’utilisateur d’identifier rapidement les extractions exploitables et celles 
nécessitant une vérification.

\subsection{Interface utilisateur et pilotage des extractions}

Afin de rendre le fonctionnement du module plus concret, les figures suivantes présentent 
les principales interfaces utilisateur du module \gls{quadrige}, depuis l’exploration des programmes 
jusqu’à l’accès aux résultats d’extraction.


Le frontend Angular du module a été conçu pour accompagner l’utilisateur à travers les 
différentes étapes du processus : définition des filtres, extraction des programmes, sélection 
des données et lancement des extractions. Des contrôles de cohérence sont intégrés afin de 
prévenir les erreurs de paramétrage, notamment sur les champs obligatoires ou les périodes 
temporelles.

Compte tenu du volume potentiellement élevé de programmes retournés par l’extraction 
— pouvant dépasser plusieurs dizaines de résultats selon le périmètre géographique — 
l’interface intègre des mécanismes de recherche par mots-clés afin de faciliter leur exploration 
et leur sélection. Il est ainsi possible de filtrer dynamiquement la liste des programmes extraits 
en saisissant un terme dans la barre de recherche, permettant de cibler rapidement un programme 
par son nom ou ses métadonnées associées.



\begin{figure}[H]
    \centering
    \includegraphics[width=0.9\textwidth]{images/quadrige/extracted_programs.png}
    \caption{Liste des programmes \gls{quadrige} extraits pour un périmètre géographique donné, tri par mots clés avec la barre de recherche}
    \label{fig:ui-program-list}
\end{figure}



Cette logique de recherche est également appliquée lors des phases de paramétrage. 
Dans les interfaces de filtrage, l’utilisateur peut rechercher les champs disponibles à l’extraction 
en saisissant directement du texte dans le champ dédié, ce qui facilite la sélection lorsque la 
liste des champs est étendue. De la même manière, les localisations suggérées lors de la définition 
des filtres peuvent être recherchées par saisie textuelle, améliorant l’ergonomie et limitant les 
erreurs de sélection.



\begin{figure}[H]
    \centering
    \includegraphics[width=0.85\textwidth]{images/quadrige/program_filter.png}
    \caption{Interface de définition des filtres d’extraction des programmes \gls{quadrige}}
    \label{fig:ui-program-filter}
\end{figure}







Les résultats des extractions sont présentés sous forme de liens téléchargeables, accompagnés 
d’indicateurs visuels signalant les éventuels avertissements. Cette approche permet à 
l’administrateur de conserver une maîtrise complète sur les données produites avant toute 
intégration dans GeoNature ou dans des plateformes partenaires telles que Borbonica.


\begin{figure}[H]
    \centering
    \includegraphics[width=0.9\textwidth]{images/quadrige/extracted_data.png}
    \caption{Résultats des extractions \gls{quadrige} et accès aux fichiers générés}
    \label{fig:ui-extraction-results}
\end{figure}




Une interface spécifique permet également de définir les filtres d’extraction des données 
associées aux programmes sélectionnés (périodes temporelles, localisations, champs à extraire). 
À des fins de lisibilité, cette interface est présentée en annexe (voir Figure~\ref{fig:data-filter}).




\subsection{Synthèse}

L’architecture retenue pour le module \gls{quadrige} répond aux contraintes spécifiques de l’\gls{api} 
Ifremer tout en respectant les principes de modularité, de sécurité et de traçabilité propres à 
GeoNature. Elle constitue une base technique solide pour des évolutions futures, notamment vers 
une automatisation partielle des imports, tout en conservant un contrôle humain sur les données 
produites.

\section{Contraintes, dépendances et livrables du module \gls{quadrige}}

La réussite du projet dépendait principalement de l’accessibilité de l’\gls{api} \gls{quadrige}, de la
stabilité de ses services, et de la disponibilité d’une documentation actualisée. L’intégration dans
GeoNature nécessitait également de respecter la structure modulaire du noyau applicatif et les
contraintes du modèle \gls{sinp}.

Les livrables attendus comprenaient le code source du module, un fichier de configuration, une
documentation à destination des administrateurs (installation, configuration, maintenance) ainsi
qu’un guide d’utilisation orienté métier.

Plusieurs évolutions ont été envisagées : automatisation des imports périodiques, gestion des
imports incrémentaux, intégration à la gestion des droits de GeoNature et, à plus long terme,
publication du module dans le catalogue officiel des extensions GeoNature.



\section{Extension du module \gls{api2gn} pour l’intégration des données Pl@ntNet}

\subsection{Rôle du module \gls{api2gn} dans l’écosystème GeoNature}

Le module \texttt{\gls{api2gn}} constitue une brique transversale au sein de l’écosystème GeoNature. 
Il se distingue des modules d’import classiques, généralement fondés sur des fichiers intermédiaires produits manuellement, 
en permettant l’intégration directe et automatisée de données issues de sources externes.


Ces sources sont exposées via des services web, tels que des \gls{apirest}, des flux JSON structurés ou des services WFS. 
Le module a été conçu pour répondre à des besoins d’automatisation et de récurrence, tout en respectant les contraintes 
du modèle de données du \gls{sinp} et de la table \textit{Synthèse} de GeoNature.

Le code source du module api2GN est disponible publiquement sur GitHub \href{https://github.com/PnX-SI/api2GN}{Dépôt officiel api2GN}.

Son fonctionnement repose sur une séparation claire entre deux niveaux. 
Le premier correspond au cœur du moteur d’import, chargé de l’orchestration des appels aux \gls{api}, 
de la gestion des erreurs, de l’historisation des opérations et de l’insertion des données en base. 
Le second niveau est constitué de parseurs spécialisés, responsables de la transformation des données sources 
vers le modèle attendu par GeoNature.

Chaque parseur implémente ainsi une logique métier propre à une source donnée, tout en s’appuyant sur un socle commun. 
Ce socle prend en charge la gestion des géométries, la résolution des nomenclatures, la validation des mappings, 
l’historisation des imports et l’intégration aux mécanismes de sécurité de GeoNature.

Dans ce contexte, le module \texttt{\gls{api2gn}} apparaît particulièrement adapté à l’intégration de données issues 
de plateformes participatives ou ouvertes, telles que Pl@ntNet. 
Ces plateformes se caractérisent par des volumes importants, une fréquence de mise à jour élevée 
et des modalités d’accès très différentes de celles des bases institutionnelles comme \gls{quadrige}.

\subsection{Objectifs et positionnement du parser Pl@ntNet}

Le fonctionnement de l’API Pl@ntNet, utilisée pour l’extraction des observations, est documenté par la plateforme elle-même\footnote{\href{https://my-api.plantnet.org/}{Documentation de l’API Pl@ntNet}}.


Le développement du parser Pl@ntNet s’inscrit dans une logique complémentaire à celle du module \gls{quadrige}. 
Alors que l’import \gls{quadrige} vise des données institutionnelles, produites dans un cadre scientifique structuré 
et contrôlé, les données issues de Pl@ntNet relèvent d’un contexte participatif.

Ce type de données se caractérise par une production massive et continue d’observations, 
une grande hétérogénéité des contributeurs et une variabilité importante de la qualité des identifications. 
La fiabilité des données repose en grande partie sur des mécanismes automatisés de reconnaissance et de validation.

Dans ce contexte, l’objectif du parser Pl@ntNet n’était pas de proposer une intégration exhaustive 
de l’ensemble des observations disponibles. 
L’enjeu consistait plutôt à mettre en place une chaîne d’import automatisée mais maîtrisée, 
capable de fonctionner régulièrement, tout en conservant un niveau élevé d’exigence sur la qualité des données intégrées.

Le parser a ainsi été conçu comme un outil configurable et sélectif. 
Il permet d’appliquer des filtres spatiaux, temporels et taxonomiques, 
et impose un contrôle strict de la compatibilité des données avec le référentiel \gls{taxref}.

Cette approche illustre une stratégie d’intégration différente de celle retenue pour les données \gls{quadrige}. 
Elle repose sur une automatisation plus poussée, compensée par des mécanismes renforcés de validation et de traçabilité.

\subsection{Architecture technique et principes de conception}

Sur le plan technique, le parser Pl@ntNet repose sur l’architecture standard des parseurs du module \texttt{\gls{api2gn}}. 
Il hérite directement de la classe \texttt{JSONParser}, elle-même dérivée de la classe générique \texttt{Parser}. 
Cette hiérarchie permet de mutualiser l’ensemble des fonctionnalités communes aux sources de données exposées sous forme JSON.

Le parser conserve toutefois la responsabilité de la logique métier spécifique à Pl@ntNet. 
Le déroulement général d’une exécution suit une séquence clairement définie. 
Après l’initialisation du parser et le chargement de la configuration, 
les requêtes vers l’\gls{api} Pl@ntNet sont construites. 
Les observations sont ensuite récupérées de manière paginée, puis transformées et normalisées. 
Une phase de résolution taxonomique est alors appliquée, avant l’insertion des observations valides 
dans la table \textit{Synthèse} de GeoNature.

Un principe fondamental a guidé la conception du parser. 
Aucune logique métier n’est codée en dur dans le code Python. 
Les paramètres d’appel à l’\gls{api}, les filtres appliqués, le mapping des champs 
et les règles de validation sont intégralement définis dans la configuration.

Ce choix favorise la maintenabilité du module et facilite son adaptation à d’autres contextes territoriaux, 
sans remise en cause de l’architecture existante.

\subsection{Configuration dynamique et pilotage par fichier TOML}

L’un des apports majeurs du travail réalisé concerne la gestion de la configuration du parser Pl@ntNet. 
Celui-ci s’appuie sur un fichier \texttt{TOML}, chargé via le mécanisme standard de configuration de GeoNature.

Ce fichier permet de définir de manière centralisée les paramètres d’accès à l’\gls{api} Pl@ntNet, 
les filtres taxonomiques éventuels, les bornes temporelles d’extraction et l’emprise géographique ciblée, 
définie sous la forme d’un polygone GeoJSON. 
Le mapping entre les champs fournis par Pl@ntNet et ceux de la table \textit{Synthèse} 
est également entièrement décrit dans ce fichier.



Le parser implémente un mécanisme de configuration par défaut.

En l’absence de fichier \texttt{api2gn\_config.toml}, ou si celui-ci est incomplet, 
un jeu de valeurs par défaut intégrées au code est utilisé.
Ce comportement permet de garantir le fonctionnement du module en environnement de test 
et de limiter les risques de blocage en production, tout en signalant explicitement 
les incohérences via les logs.

Une route API dédiée permet de consulter la configuration effective chargée par GeoNature.
Cette route applique une validation métier non bloquante, signalant les incohérences potentielles (dates inversées, géométrie invalide, mapping incorrect) sans empêcher l’exécution du parser.
Ce choix vise à favoriser la transparence et l’autonomie des administrateurs, tout en conservant un haut niveau de robustesse.

Le parser Pl@ntNet met également en œuvre un mécanisme d’auto-création des métadonnées nécessaires à l’intégration dans GeoNature.
Lors de la première exécution, le parser vérifie l’existence de la source, du cadre d’acquisition et du jeu de données associés à Pl@ntNet.
En l’absence de ces éléments, ils sont automatiquement créés en base, dans le respect du modèle GeoNature.
Ce mécanisme permet de rendre le parser autonome et de limiter les opérations manuelles d’initialisation lors du déploiement.

La structuration des données repose sur le standard Darwin Core, tel que défini par le \href{https://www.gbif.org/fr/darwin-core}{GBIF}.


Cette stratégie renforce la robustesse du module, notamment en phase de test ou de développement, 
tout en incitant à une configuration explicite en environnement de production. 
Elle ouvre également la voie à une future interface d’administration graphique dédiée 
au pilotage de ces paramètres.



\subsection{Résolution taxonomique et contrôle de la qualité des données}

La résolution taxonomique constitue l’enjeu central de l’intégration des données Pl@ntNet dans GeoNature. 
Afin de garantir la cohérence avec le référentiel national TAXREF et les exigences du \gls{sinp}, 
le parser met en œuvre une chaîne de validation stricte.

Pour chaque observation, le nom scientifique fourni par l’\gls{api} Pl@ntNet est d’abord normalisé. 
Les mentions infra-spécifiques et les annotations non standard sont supprimées afin d’améliorer 
les chances de correspondance avec le référentiel.

Le parser tente ensuite de résoudre le \texttt{cd\_nom} en interrogeant en priorité le référentiel TAXREF local. 
En cas d’échec, le service \gls{taxrefld} du \gls{mnhn} est sollicité. 
Le \texttt{cd\_nom} obtenu est enfin vérifié dans la base locale de GeoNature.

Afin d’optimiser les performances lors des imports volumineux, un cache mémoire des résolutions taxonomiques est utilisé.
Ce cache permet d’éviter des requêtes répétées vers la base TAXREF locale ou les services \gls{taxrefld} pour un même nom scientifique, 
réduisant significativement le temps de traitement.


Par défaut, le parser fonctionne en mode strict. 
Toute observation pour laquelle aucun \texttt{cd\_nom} valide ne peut être résolu est rejetée. 
Ces rejets sont systématiquement journalisés, ce qui permet d’identifier les taxons problématiques 
et d’évaluer la qualité globale des données importées.

\subsection{Traçabilité, sécurité et automatisation des imports}

Le parser Pl@ntNet s’intègre pleinement aux mécanismes de traçabilité fournis par le module \texttt{\gls{api2gn}}. 
Chaque exécution est historisée dans une table dédiée. 
Cette historisation permet de conserver la date du dernier import, 
le nombre d’observations intégrées et le volume total de données importées.

Un mode \textit{dry-run} est systématiquement disponible. 
Il permet de simuler un import sans insertion en base. 
Ce mode constitue un outil essentiel pour tester une nouvelle configuration, 
évaluer l’impact d’un changement de périmètre ou analyser la qualité des données 
avant une intégration effective.

Le module intègre un mécanisme d’automatisation basé sur Celery.
Les parseurs peuvent être configurés avec une fréquence d’exécution stockée en base.
Un processus périodique vérifie les dates de dernier import et déclenche automatiquement les imports lorsque la 
fréquence définie est atteinte. Il devient ainsi possible d’envisager des imports périodiques automatisés, 
tout en conservant un contrôle fin sur leur déclenchement et leur suivi.

\subsection{Apports du travail réalisé}

Le développement du parser Pl@ntNet a permis de démontrer la capacité de GeoNature 
à intégrer des données participatives de manière automatisée, 
tout en respectant les exigences du \gls{sinp} en matière de qualité, de traçabilité 
et de cohérence taxonomique.

Au-delà du cas spécifique de Pl@ntNet, ce travail met en évidence la pertinence 
de l’architecture \texttt{\gls{api2gn}} pour l’intégration de sources de données hétérogènes. 
Il souligne également l’intérêt d’une configuration externalisée 
pour favoriser la réutilisation du module et son adaptation à différents contextes.

Le parser Pl@ntNet constitue ainsi un prototype fonctionnel et robuste. 
Il ouvre la voie à l’intégration d’autres sources de données ouvertes 
et contribue à l’enrichissement progressif de la base GeoNature, 
notamment pour les taxons terrestres peu couverts par les réseaux institutionnels traditionnels.

Le code source du parser Pl@ntNet et des modules développés durant le stage est versionné et documenté, 
afin d’en faciliter la reprise et l’évolution, les références étant fournies en Webographie.



% ------------------------------------------
% Travaux réalisés et compétences acquises
% ------------------------------------------
\chapter{Travaux réalisés et compétences \allowbreak acquises}

\section{Synthèse des travaux réalisés}

Les travaux réalisés au cours du stage se sont articulés autour de deux axes
principaux.

Le premier axe concerne le développement d’un module externe pour GeoNature,
dédié à l’exploration et à l’extraction de données marines issues du système
d’information \gls{quadrige}.

Le second axe porte sur un travail complémentaire autour de l’intégration
de données participatives produites par l’application \gls{plantnet}.
L’objectif était de poser les bases techniques d’un futur parseur,
adapté à une intégration automatisée dans GeoNature.

L’ensemble des développements a été mené dans une logique d’intégration
au système d’information existant.
Les choix effectués prennent en compte les contraintes techniques,
organisationnelles et métier propres au \gls{sinp}.

\subsection{Vue d’ensemble des développements}

Les développements réalisés comprennent plusieurs volets complémentaires.

Ils incluent la conception d’un module GeoNature externe,
interfacé avec l’\gls{api} de \gls{quadrige},
permettant d’explorer les programmes disponibles
et d’en extraire les données associées.

Ils comprennent également la mise en place d’une interface frontend
intégrée à GeoNature.
Cette interface permet de rechercher, filtrer et sélectionner
des programmes et des données à extraire.

Les extractions produisent des fichiers intermédiaires documentés,
destinés à une intégration différée dans GeoNature.
Cette approche permet de conserver un contrôle sur les données produites
avant toute insertion en base.

Enfin, une documentation technique et utilisateur a été rédigée
tout au long du projet.
Elle décrit les principes de fonctionnement du module,
les procédures d’installation,
ainsi que les modalités d’utilisation et de maintenance.

L’ensemble de ces travaux a été conçu pour produire un outil fonctionnel,
maintenable et réutilisable.
Le module est destiné à être repris ou étendu par les équipes de la
\gls{deal} ou par d’autres structures partenaires.

\subsection{Travaux principaux : module \gls{quadrige}}

Les travaux principaux du stage ont porté sur le développement
du module \gls{quadrige} pour GeoNature.

Ce module permet d’interroger l’\gls{api} GraphQL mise à disposition par
l’Ifremer afin d’identifier les programmes pertinents
pour un périmètre géographique donné.
Il offre ensuite la possibilité d’extraire les données associées
selon des critères définis par l’utilisateur.

Le module repose sur une architecture découplée,
séparant clairement le backend et le frontend.
Le backend est responsable des échanges avec l’\gls{api},
de l’authentification,
du suivi des extractions
et de la gestion des fichiers produits.
Le frontend se limite au pilotage des opérations
et à la visualisation des résultats.

Les extractions réalisées produisent des fichiers intermédiaires,
notamment des fichiers \gls{csv} et des archives compressées.
Ces fichiers sont accompagnés de métadonnées décrivant précisément
les paramètres d’extraction utilisés.
Cette organisation garantit la traçabilité des opérations
et facilite l’analyse des résultats.

Le module a été conçu comme une brique préparatoire.
Il ne réalise pas directement l’import des données en base GeoNature,
mais vise à sécuriser les étapes amont
et à faciliter une intégration future
dans le respect des règles du \gls{sinp}.

\subsection{Travaux complémentaires : \gls{plantnet}}

En parallèle du développement du module \gls{quadrige},
un travail complémentaire a été mené autour de l’intégration
des données issues de \gls{plantnet}.

L’objectif de ce travail était d’évaluer la faisabilité technique
d’une intégration automatisée de données participatives
dans GeoNature.
Il s’agissait également de poser les bases d’un futur parseur
utilisable par le module \gls{api2gn}.

Un client Python générique a été développé
pour interroger l’\gls{api} \gls{plantnet},
récupérer les observations
et enregistrer les réponses brutes au format \gls{json}.

Ces données ont ensuite été transformées
en fichiers \gls{csv} conformes au standard Darwin Core.
Une attention particulière a été portée à la normalisation des champs,
à la gestion des dates,
aux coordonnées géographiques
et à la génération d’identifiants stables.

Enfin, un pipeline automatisé, intégrant des mécanismes de validation et de contrôle
reposant sur une configuration externalisée.
Ce pipeline constitue aujourd’hui la base technique
du parseur \gls{plantnet} intégré au module \gls{api2gn}.

Bien que complémentaire au module \gls{quadrige},
ce travail s’inscrit dans la même logique globale.
Il vise à démontrer la capacité de GeoNature
à intégrer des sources de données hétérogènes,
tout en respectant les exigences de qualité,
de traçabilité et de cohérence du \gls{sinp}.


\section{Positionnement et choix d’ingénierie}
Le travail réalisé au cours de ce stage ne se limite pas
au développement ponctuel d’un script
ou à l’utilisation isolée d’une \gls{api} externe.

Le module \gls{quadrige} développé s’inscrit pleinement
dans l’écosystème GeoNature.
Il respecte l’architecture logicielle existante,
les conventions de développement
et les mécanismes de sécurité du système.

L’objectif n’était pas uniquement de rendre possible
l’interrogation de l’\gls{api} \gls{quadrige}.
Il s’agissait de concevoir une chaîne d’acquisition
structurée, reproductible et traçable.

Cette chaîne permet d’explorer les données disponibles,
de les extraire selon des critères maîtrisés,
puis de les préparer pour une intégration ultérieure.
Elle laisse volontairement une place
au contrôle humain et à l’analyse métier.

Le module a été conçu comme une extension à part entière
du système d’information.
Il respecte la séparation des responsabilités
entre le frontend et le backend,
ainsi que les principes de configuration centralisée
propres à GeoNature.

Une attention particulière a été portée
à la maintenabilité du code.
Les choix d’architecture,
la structuration des modules
et la documentation associée
ont été pensés pour faciliter
la reprise du projet par d’autres développeurs.

Le travail mené relève ainsi
d’une démarche d’ingénierie logicielle.
Il vise à produire un outil durable,
adaptable à d’autres territoires
et cohérent avec les pratiques
de gestion des données naturalistes
portées par la \gls{deal} et ses partenaires.


\section{Arbitrages techniques}
Plusieurs arbitrages techniques ont été réalisés
tout au long du stage.
Ces choix ont été guidés par les contraintes métier,
les usages existants
et les exigences de fiabilité
associées aux données environnementales.

Un premier choix structurant a consisté
à adopter une approche intermédiaire.
Le processus repose sur une succession d’étapes distinctes :
extraction des données,
contrôle,
puis intégration différée.

Une automatisation complète et immédiate
de l’import en base GeoNature
n’a pas été retenue.
Ce choix permet de conserver
un contrôle humain sur les données extraites,
notamment dans un contexte
où certaines informations
peuvent être soumises
à des règles de diffusion spécifiques.

Dans cette logique,
l’intégration directe des données \gls{quadrige}
dans la base GeoNature
n’a volontairement pas été implémentée.
Le module produit des fichiers intermédiaires documentés,
afin de sécuriser les étapes amont
et de faciliter l’analyse des résultats.

Concernant le filtrage géographique,
un traitement a posteriori
à l’aide de la bibliothèque \textit{pandas}
a été privilégié.
Ce choix s’explique par le fonctionnement même
de l’\gls{api} \gls{quadrige},
qui raisonne à l’échelle des programmes
et non des stations individuelles.

Enfin, la configuration des modules \gls{quadrige}
et \gls{plantnet}
a été entièrement externalisée
dans des fichiers dédiés.
Ce choix permet d’adapter les paramètres d’extraction
sans modifier le code,
de limiter les risques d’erreur
et de faciliter le déploiement
sur d’autres environnements.

Ces arbitrages traduisent une volonté claire
de privilégier la robustesse,
la traçabilité
et la réutilisabilité,
plutôt qu’une automatisation maximale
au détriment du contrôle des données.


\section{Difficultés rencontrées et solutions apportées}
Tout au long du stage, ces arbitrages ont été confrontés aux contraintes techniques concrètes du projet.
Son déroulement a ainsi été marqué par plusieurs difficultés ayant influencé son organisation.


L’absence d’une installation locale stable de GeoNature
a constitué l’une des principales difficultés.
Les incompatibilités entre certaines versions logicielles,
les contraintes liées à Docker
et les erreurs du serveur applicatif
ont rendu nécessaire
la mise en place d’un environnement de travail spécifique.

Cette situation a été renforcée
par les contraintes réseau de la \gls{deal}.
La présence d’un proxy strict
a perturbé l’utilisation
de plusieurs outils de développement.
La mise en place progressive
d’un mécanisme de gestion dynamique du proxy
a permis de stabiliser l’environnement.

La migration des serveurs du \gls{sinp}
vers un nouvel hébergeur
a également limité l’accès
à un environnement de test réaliste.
Pendant plusieurs semaines,
le développement a dû être poursuivi
sans possibilité de déploiement direct.

Une fois l’accès ouvert
via une machine virtuelle bastion,
les premières manipulations
ont été freinées par les restrictions
liées à l’accès distant.
L’utilisation de Git
comme canal principal de transfert
s’est révélée déterminante
pour poursuivre les tests
et ajuster le module.


Une difficulté est apparue en fin de stage, lors du déploiement du module Quadrige sur le serveur de la DEAL.

Le volume important de programmes Quadrige sur le périmètre de La Réunion était déjà connu en amont. Les tests réalisés 
en local permettaient d’extraire plusieurs dizaines de programmes sans difficulté particulière, y compris lors de traitements 
longs. Le module avait donc été conçu dès le départ pour gérer un nombre élevé de programmes, supérieur à 80.

Les problèmes sont apparus lors des premiers déploiements en environnement réel. Le serveur GeoNature, basé sur une distribution 
Debian, utilise une chaîne applicative reposant sur Apache et Gunicorn. Dans cette configuration, des timeouts bloquants sont 
appliqués aux requêtes HTTP dépassant une durée d’environ 30 secondes.

Lors du lancement des extractions Quadrige en conditions réelles, certaines requêtes longues dépassaient ce seuil. Cela entraînait 
l’interruption complète du traitement, même lorsque seules quelques extractions restaient en cours. Ces échecs ne reflétaient pas un 
problème fonctionnel de l’API Quadrige, mais une incompatibilité entre la durée des traitements et les contraintes du serveur applicatif.

Pour répondre à cette contrainte, j’ai revu la logique d’orchestration des appels à l’API Quadrige. Les extractions sont désormais 
lancées en batch, en soumettant l’ensemble des programmes sélectionnés lors d’une première phase rapide. Un mécanisme de polling 
global permet ensuite de suivre l’état de chaque extraction de manière non bloquante, à l’aide de requêtes courtes compatibles avec 
les timeouts imposés par Apache et Gunicorn.

Cette approche permet de découpler le lancement des extractions de leur suivi, en évitant une logique strictement séquentielle. 
Les appels lourds à l’API Quadrige sont remplacés par une succession de requêtes courtes de suivi d’état, compatibles avec les 
contraintes de timeout imposées par Apache et Gunicorn.

Bien que le traitement reste piloté par une requête unique côté GeoNature, cette organisation réduit fortement les risques de blocage 
liés aux traitements longs. Elle permet également de maîtriser le temps global d’exécution et d’éviter qu’un programme lent n’entraîne 
l’échec de l’ensemble de l’extraction

Cette solution ne repose pas sur une exécution asynchrone complète côté serveur GeoNature, mais sur une orchestration optimisée des 
appels à l’API Quadrige. Elle permet néanmoins de contourner efficacement les limitations liées aux timeouts applicatifs en production.

Enfin, la récupération des résultats et le téléchargement des fichiers sont réalisés programme par programme. Les erreurs sont gérées 
individuellement, ce qui permet de conserver les extractions valides même en cas d’échec partiel.

Cette évolution a permis de rendre le module compatible avec les contraintes de l’environnement de production. Elle a supprimé les 
timeouts rencontrés lors des premiers déploiements et renforcé la robustesse du module face aux montées en charge, tout en conservant 
une traçabilité fine des opérations réalisées.



Le travail sur les données \gls{plantnet}
a également soulevé des difficultés spécifiques.
L’\gls{api} renvoie des structures \gls{json}
très hétérogènes,
variables selon les taxons
et les métadonnées disponibles.
Cela a nécessité la conception
d’un parseur robuste et flexible.

Par ailleurs, le standard Darwin Core
impose un modèle strict,
qui ne correspond pas directement
aux champs fournis par \gls{plantnet}.
Un travail de mapping configurable
a été nécessaire
pour garantir la compatibilité
avec GeoNature et Borbonica.

Enfin, l’intégration du parseur
au module \gls{api2gn}
a mis en évidence certaines limites
dans les migrations SQL existantes.
Des ajustements ont été réalisés
afin de permettre une installation correcte du module.

Ces difficultés ont nécessité
une forte capacité d’adaptation.
Elles ont conduit
à renforcer la documentation,
à multiplier les tests exploratoires
et à clarifier progressivement
les environnements de travail.

Elles ont également contribué
à structurer une démarche rigoureuse,
essentielle dans un contexte
d’ingénierie logicielle appliquée
aux données environnementales.


\section{Compétences acquises et prise de recul}
Ce stage m’a permis d’acquérir et de consolider
un ensemble de compétences techniques,
méthodologiques
et transversales,
directement mobilisables
dans un contexte professionnel d’ingénierie.

Sur le plan technique,
j’ai développé une compréhension approfondie
de l’architecture de GeoNature.
J’ai travaillé à la fois sur le backend
(Python, Flask, SQLAlchemy)
et sur le frontend
(Angular, TypeScript),
en respectant une séparation stricte
des responsabilités.

J’ai acquis une expérience concrète
dans l’interfaçage de systèmes hétérogènes,
à travers l’utilisation d’API REST et GraphQL.
Ce travail m’a permis de mieux comprendre
les enjeux liés à l’authentification,
à la gestion des erreurs
et à la robustesse des échanges réseau.

La manipulation de données volumineuses
et hétérogènes
m’a conduit à mettre en œuvre
des traitements de normalisation,
de filtrage
et de validation.
J’ai notamment travaillé
sur des formats standards
tels que CSV, JSON
et Darwin Core,
en tenant compte
des contraintes imposées
par les référentiels nationaux.

Ce stage m’a également permis
de renforcer mes compétences
en ingénierie logicielle.
La structuration du code,
la gestion de la configuration,
l’écriture de tests
et la documentation
ont été des éléments centraux du travail.
J’ai appris à concevoir un outil
pensé pour être maintenu,
repris
et adapté dans le temps.

Sur le plan méthodologique,
j’ai travaillé en grande autonomie,
tout en échangeant régulièrement
avec mon tuteur et les agents concernés.
J’ai appris à organiser mon travail
dans un contexte de projet réel,
soumis à des contraintes de délais,
d’accès aux environnements
et d’incertitudes techniques.

Le stage m’a également permis
de mieux appréhender
le fonctionnement d’une administration publique.
J’ai découvert un cadre de travail
où les outils numériques
s’inscrivent dans des processus longs,
structurés
et fortement contraints.
Cette expérience m’a aidé
à adapter mes choix techniques
aux réalités organisationnelles
et aux usages métiers.

Au-delà des compétences techniques,
ce stage m’a conduit
à prendre du recul
sur la place du numérique
dans les politiques publiques environnementales.
Les outils développés
ne constituent pas une finalité en soi.
Ils participent à la structuration,
à la fiabilité
et à la diffusion de données
qui alimentent des décisions
ayant un impact direct
sur les territoires et les écosystèmes.

Travailler sur des données naturalistes
implique une responsabilité particulière.
Les choix techniques effectués
influencent la qualité des analyses produites,
la visibilité de certaines thématiques
et la capacité des acteurs
à agir de manière éclairée.

Ce stage m’a ainsi permis
de mieux distinguer
le fait de produire du code
et celui de concevoir un outil réellement utile.
Un outil utile est compréhensible,
maintenable,
adapté à son contexte d’usage
et aligné avec les enjeux
qu’il vise à servir.

Cette expérience correspond pleinement
aux objectifs de la formation
d’ingénieur généraliste dispensée à l’ENIB.
Elle a renforcé ma capacité
à articuler compétences techniques,
compréhension des enjeux sociétaux
et sens donné aux projets réalisés.


Les compétences acquises et les outils développés au cours de ce stage
ouvrent naturellement la voie à plusieurs perspectives d’évolution,
présentées dans le chapitre suivant.


% ------------------------------------------  

% ------------------------------------------
% PERSPECTIVES ET CONCLUSION
% ------------------------------------------
\chapter{Perspectives}
Les travaux réalisés au cours de ce stage ont permis de poser des bases techniques solides pour 
l’intégration de données externes au sein de \gls{geonature}, en particulier pour les données 
issues du milieu marin via \gls{quadrige}, ainsi que pour les données participatives produites 
par \gls{plantnet}.  
Plusieurs perspectives d’évolution ont été identifiées, tant sur le plan fonctionnel que 
technique. Elles visent à renforcer l’ergonomie des outils développés, à améliorer leur 
intégration dans l’écosystème GeoNature et à faciliter leur appropriation par les administrateurs 
et gestionnaires de données.

\section{Perspectives d’évolution du module d’import Quadrige}
Le module d’import {\gls{quadrige}} développé durant le stage constitue une première brique 
fonctionnelle permettant d’explorer, d’extraire et de préparer des données marines issues du 
système d’information \gls{quadrige}. Plusieurs axes d’amélioration peuvent être envisagés afin d’en 
accroître la maturité et de faciliter son intégration opérationnelle à long terme.

\subsection{Amélioration de l’ergonomie et de l’intégration visuelle}
    
Une première évolution concernerait l’intégration visuelle du module au sein de l’interface 
GeoNature. L’ajout d’un pictogramme dédié dans la barre de navigation ou dans la liste des 
modules permettrait d’identifier plus clairement le module \gls{quadrige} parmi les autres extensions 
disponibles.

Ce pictogramme contribuerait à améliorer l’expérience utilisateur en rendant le module plus 
visible et plus intuitif pour les administrateurs amenés à l’utiliser ponctuellement. Il 
faciliterait également son appropriation par de nouveaux utilisateurs, notamment dans un 
contexte de reprise du projet ou de déploiement sur d’autres territoires.

\subsection{Ajout d’une étape de mise en forme et d’intégration des données}

À l’heure actuelle, le module produit des archives de données (fichiers CSV bruts, CSV filtrés, 
fichiers JSON de paramètres) à l’issue de la phase d’extraction depuis l’API \gls{quadrige}. Ces 
archives constituent une étape intermédiaire volontaire, permettant un contrôle manuel des 
résultats avant toute intégration dans GeoNature.

Une évolution naturelle du module consisterait à ajouter une étape supplémentaire, positionnée 
après la génération des fichiers ZIP, dédiée à la mise en forme finale et à l’intégration des 
données dans la base GeoNature. Cette étape pourrait inclure la transformation des données vers 
le schéma de synthèse, ainsi que leur préparation pour un import direct dans la base de données.

Cette évolution permettrait de proposer une chaîne complète allant de l’extraction à 
l’intégration, tout en conservant la possibilité de valider manuellement les données en amont. 
Elle resterait compatible avec les exigences de traçabilité et de contrôle associées au 
\gls{sinp}.

\subsection{Clarification et extension de la gestion de la configuration}

Le module repose déjà sur un fichier de configuration centralisé, chargé côté backend dans 
l’environnement GeoNature. Ce fichier permet notamment de définir les paramètres techniques 
essentiels tels que l’URL de l’API \gls{quadrige} et le jeton d’authentification, qui sont ensuite mis à 
disposition du backend.

Le backend expose également le contenu de ce fichier de configuration via une route dédiée, 
permettant théoriquement au frontend d’accéder dynamiquement à ces informations. Toutefois, 
par manque de temps, certaines données métier — telles que les localisations suggérées lors du 
filtrage des programmes ou la liste des champs disponibles pour l’extraction — restent définies 
directement dans le frontend.

Une évolution souhaitable consisterait à étendre le rôle du fichier de configuration afin d’y 
inclure ces éléments métier, puis à adapter le frontend pour les charger dynamiquement depuis 
le backend. Cette approche permettrait de réduire les valeurs codées en dur dans l’interface, 
d’améliorer la cohérence entre backend et frontend et de faciliter l’adaptation du module à 
d’autres territoires ou contextes d’usage.

\subsection{Vers une automatisation partielle et maîtrisée des imports}

À plus long terme, le module pourrait évoluer vers une automatisation partielle ou contrôlée des 
imports, par exemple via des imports périodiques planifiés, des imports incrémentaux basés sur 
les dates de mise à jour des données, ou des déclenchements manuels avec validation préalable.

Toutefois, certaines spécificités des données \gls{quadrige} imposent de conserver un contrôle humain 
sur les résultats des extractions. Certaines informations peuvent être anonymisées (par exemple 
les données relatives aux observateurs) ou soumises à des moratoires de diffusion. Il est donc 
actuellement nécessaire de vérifier les résultats des extractions avant tout traitement visant à 
leur intégration dans GeoNature.

Ces contraintes devront être pleinement prises en compte dans toute évolution vers une 
automatisation plus poussée, afin de rester cohérent avec les exigences du \gls{sinp} et les 
pratiques de contrôle des données naturalistes.


\subsection{Renforcement de la stratégie de tests et de validation}



Les modules externes GeoNature sont conçus pour s’inscrire dans une logique de maintenance à 
long terme. À ce titre, la documentation officielle recommande la mise en place de tests 
automatisés, notamment des tests unitaires et fonctionnels côté backend (via PyTest) ainsi que 
des tests de bout en bout (\textit{end-to-end}) côté frontend (via Cypress).

Dans le cadre de ce stage, des tests ont été développés et validés sur la version locale du 
module Quadrige, afin de vérifier le bon fonctionnement des principaux composants : appels à 
l’API Quadrige, gestion des erreurs, génération des fichiers d’export et cohérence des routes 
exposées par le backend. De la même manière, des scénarios de tests frontend ont été amorcés 
pour valider les parcours utilisateurs essentiels.

Toutefois, en raison des contraintes de temps et du calendrier de déploiement sur les serveurs 
du SINP-974, ces tests n’ont pas pu être intégrés dans la version du module effectivement 
déployée sur l’environnement serveur. 

Une perspective importante consisterait donc à intégrer ces suites de tests dans la 
version déployée du module, afin de renforcer sa robustesse, de sécuriser les évolutions futures 
et de faciliter sa reprise par d’autres développeurs. La généralisation des tests automatisés 
permettrait également de s’inscrire plus étroitement dans les bonnes pratiques recommandées 
pour le développement et la maintenance des modules externes GeoNature.

Dans l’état actuel, le module reste piloté par une requête HTTP unique côté GeoNature.
Une évolution possible consisterait à découpler totalement le traitement de la requête HTTP,
par exemple via une exécution en tâche de fond (Celery ou job système), afin de rendre le module
pleinement asynchrone côté serveur GeoNature.


\section{Perspectives autour du parser Pl@ntNet et du module api2gn}

Les travaux menés sur l’intégration des données \gls{plantnet} ont permis de valider la faisabilité 
technique d’un pipeline complet, depuis l’interrogation de l’API jusqu’à la production de 
fichiers conformes au standard Darwin Core. Toutefois, ces données étant issues d’une 
application participative, leur exploitation soulève des enjeux spécifiques de validation et de 
qualité scientifique.

\subsection{Ajout d’une interface d’administration dédiée}

Une perspective majeure consisterait à proposer une interface frontend permettant aux 
administrateurs de paramétrer le parser \gls{plantnet} sans recourir à la modification de fichiers de 
configuration. Cette interface pourrait permettre de définir les taxons ciblés, les périmètres 
géographiques, les plages temporelles ainsi que les paramètres de filtrage et de normalisation.

Une telle évolution améliorerait significativement l’ergonomie du dispositif et faciliterait son 
utilisation par des profils non techniques. Elle permettrait également d’harmoniser l’expérience 
utilisateur avec celle du module \gls{quadrige}, en proposant une logique d’interaction cohérente au 
sein de GeoNature.

\subsection{Contraintes liées au module api2gn}

Dans l’état actuel, cette évolution n’est toutefois pas immédiatement réalisable. Le module 
\texttt{api2gn} est un module officiel de GeoNature, dont l’architecture n’a pas été conçue à 
l’origine pour intégrer des interfaces frontend spécifiques à chaque source de données.

Toute extension de ce type nécessiterait soit une évolution structurelle du module 
\texttt{api2gn} lui-même, soit la création d’un module externe complémentaire venant piloter le 
parser \gls{plantnet}. Cette réflexion dépasse le cadre du stage, mais les travaux réalisés constituent 
une base technique solide pour initier ce type d’évolution à l’avenir.

\subsection{Renforcement des mécanismes de validation et d’incrémentalité}

Les données \gls{plantnet} étant saisies par des utilisateurs non experts, la validation des taxons 
constitue un enjeu central. Le parser développé intègre déjà un mécanisme de validation basé 
sur la comparaison des \textit{scientific names} avec les référentiels taxonomiques utilisés par 
GeoNature, ainsi que sur des interrogations complémentaires de services externes lorsque cela 
est nécessaire. Un historique des taxons validés et rejetés est également conservé.

Une perspective d’amélioration consisterait à renforcer ces mécanismes, notamment en affinant 
la gestion des requêtes incrémentales. Actuellement, les extractions sont réalisées par blocs de 
1\,000 occurrences. Il pourrait être pertinent de permettre une reprise plus fine des 
extractions interrompues, afin d’éviter de relancer l’ensemble des requêtes en cas d’arrêt 
partiel du traitement.


\section{Ouvertures transversales}

Les développements réalisés durant ce stage ont permis de mettre en évidence l’intérêt d’adapter 
les stratégies d’intégration des données en fonction de la nature et de l’origine des sources 
mobilisées. Les travaux menés sur \gls{quadrige} et \gls{plantnet} illustrent deux approches 
complémentaires, répondant à des contraintes distinctes.

Pour les données issues de \gls{quadrige}, produites dans un cadre institutionnel et scientifique, 
la priorité a été donnée à la traçabilité et au contrôle des extractions. La séparation entre les 
phases d’extraction, de préparation et d’intégration permet de conserver un contrôle humain sur 
des données pouvant être soumises à des règles de diffusion spécifiques (anonymisation, 
moratoires), tout en préparant leur intégration progressive dans GeoNature.

À l’inverse, les données issues de \gls{plantnet}, bien que participatives, peuvent faire l’objet d’un 
traitement largement automatisé. Le pipeline mis en place intègre des mécanismes de validation 
taxonomique reposant sur les référentiels utilisés par GeoNature, permettant d’identifier, de 
corriger ou de rejeter automatiquement les observations non conformes. Les données validées 
peuvent ainsi être intégrées directement dans la base GeoNature, sans intervention manuelle, 
tout en conservant un historique des décisions prises par le système.

Cette distinction entre intégration contrôlée et intégration automatisée souligne l’importance 
d’une architecture d’import flexible, capable de s’adapter aux caractéristiques des sources de 
données. Elle ouvre la voie à une gestion différenciée des flux, conciliant exigences de qualité 
scientifique, efficacité opérationnelle et montée en charge des systèmes d’information 
naturalistes.



% ------------------------------------------
% CONCLUSION
% ------------------------------------------
\chapter{Conclusion}
Ce stage s’est inscrit dans un contexte à la fois technique,
institutionnel
et environnemental exigeant.
Il avait pour objectif de contribuer
au renforcement des outils numériques
utilisés par la DEAL Réunion
pour la gestion et la valorisation
des données naturalistes.

Les travaux réalisés ont permis
de développer une première brique fonctionnelle
dédiée à l’exploration et à l’extraction
des données marines issues du système d’information Quadrige.
Le module conçu s’intègre à l’architecture de GeoNature,
respecte ses principes de sécurité et de modularité,
et propose une chaîne d’acquisition
structurée, traçable et reproductible.
Il constitue une étape préparatoire essentielle
à l’enrichissement progressif du SINP-974
en données issues du milieu marin.

En parallèle, le travail mené autour de Pl@ntNet
a permis de valider la faisabilité technique
d’une intégration automatisée
de données participatives dans GeoNature.
Le parser développé,
adossé au module api2GN,
met en œuvre des mécanismes robustes
de configuration, de validation taxonomique
et de traçabilité.
Il illustre une approche complémentaire
à celle retenue pour Quadrige,
fondée sur une automatisation maîtrisée
et adaptée à la nature des données traitées.

Au-delà des développements réalisés,
ce stage a mis en évidence
l’importance des choix d’ingénierie
dans un système d’information environnemental.
La conception des outils,
le degré d’automatisation retenu,
la gestion des erreurs
et la traçabilité des traitements
ont un impact direct
sur la qualité des données produites
et sur les usages qui en découlent.
Ces choix doivent donc être pensés
en lien étroit avec les contraintes réglementaires,
les pratiques métier
et les enjeux de diffusion de l’information.

Cette expérience m'a également permis
de mieux appréhender
le fonctionnement d’une administration publique,
où les projets numériques
s’inscrivent dans des temporalités longues,
des environnements contraints
et des logiques de coopération entre acteurs.
Elle a renforcé ma capacité
à adapter mes solutions techniques
à un cadre organisationnel réel,
tout en conservant une exigence
de rigueur et de qualité.

Enfin, ce stage a contribué
à donner du sens à ma formation d’ingénieur généraliste.
Il m’a permis de mobiliser
des compétences en informatique et en traitement de données
au service d’enjeux environnementaux concrets.
Il a confirmé mon intérêt
pour le développement d’outils numériques
utiles, durables
et porteurs d’impact,
en particulier dans les domaines
liés à la gestion des données environnementales
et à la préservation des écosystèmes.

Les perspectives identifiées à l’issue de ce travail
ouvrent la voie à des évolutions techniques
et fonctionnelles ambitieuses,
tant pour l’intégration des données marines
que pour l’exploitation de sources participatives.
Elles constituent autant d’opportunités
pour poursuivre l’amélioration
des systèmes d’information naturalistes
et renforcer leur rôle
dans l’aide à la décision publique.



% ------------------------------------------
% ANNEXES
% ------------------------------------------
\BlueChapterAnnexes{Annexes}
\enlargethispage{1.5\baselineskip}
\begin{figure}[H]
    \centering
    \includegraphics[width=0.9\textwidth]{images/organigramme_ubio.pdf}
    \caption{Organigramme de l'Unité Biodiversité (UBIO) – SEB / \gls{deal} Réunion}
    \label{fig:organigramme-ubio}
\end{figure}


\begin{figure}[h!]
    \centering
    \includegraphics[width=0.85\textwidth]{images/back_front_geonature.png}
    \caption{Architecture générale du backend et du frontend de GeoNature}
    \label{fig:backfront}
\end{figure}



\section{Liens utiles}

Pour accompagner le développement, plusieurs ressources officielles ont été mobilisées :

\textbf{Quadrige}
\begin{itemize}

    \item \href{https://quadrige-core.ifremer.fr/api/extraction/doc?doc=standard&lang=fr&name=result&type=standard}{Documentation API}
    \item \href{https://quadrige.ifremer.fr/support/Mes-donnees/J-extrais-mes-donnees/J-interroge-l-API-pour-extraire-mes-donnees/Je-regarde-des-videos-de-demo-sur-l-API}{Tutoriels vidéo}  
    \item \href{https://quadrige-app.ifremer.fr/}{Accès à l’application}

\end{itemize}

\textbf{GeoNature}
\begin{itemize}
    \item \href{https://github.com/PnX-SI/GeoNature}{Dépôt GitHub}  
    \item \href{https://docs.geonature.fr/}{Documentation générale}  
    \item \href{https://docs.geonature.fr/development.html\#developper-un-module-externe}{Guide de développement des modules externes}  
\end{itemize}

\bigskip

\noindent
L’ensemble de ces éléments a permis de définir précisément le périmètre et les modalités du
développement. Le chapitre suivant présente la réalisation concrète du module, depuis
l’organisation du travail jusqu’aux phases de tests et de validation des fonctionnalités mises en
place.



% ------------------------------------------
% WEBOGRAPHIE
% ------------------------------------------
\BlueChapter{Webographie}
Pour accompagner le développement, plusieurs ressources officielles ont été mobilisées :

\textbf{Quadrige}
\begin{itemize}
    \item \href{https://quadrige-core.ifremer.fr/api/extraction/doc?doc=standard&lang=fr&name=result&type=standard}{Documentation API}
    \item \href{https://quadrige.ifremer.fr/support/Mes-donnees/J-extrais-mes-donnees/J-interroge-l-API-pour-extraire-mes-donnees/Je-regarde-des-videos-de-demo-sur-l-API}{Tutoriels vidéo}  
    \item \href{https://quadrige-app.ifremer.fr/}{Accès à l’application}

\end{itemize}

\textbf{GeoNature}
\begin{itemize}
    \item \href{https://geonature.fr/}{Site officiel}
    \item \href{https://github.com/PnX-SI/GeoNature}{Dépôt GitHub}  
    \item \href{https://docs.geonature.fr/}{Documentation générale}  
    \item \href{https://docs.geonature.fr/development.html\#developper-un-module-externe}{Guide de développement des modules externes}  
\end{itemize}

\textbf{DEAL Réunion}
\begin{itemize}
    \item \href{https://www.reunion.developpement-durable.gouv.fr/}{Site officiel}  
    \item \href{https://deal.reunion.developpement-durable.gouv.fr/La-DEAL-Reunion-r839.html}{Présentation de la DEAL Réunion}

\end{itemize}


\textbf{module quadrige}
\begin{itemize}
    \item \href{https://github.com/basileandre056/geonature_quadrige_extraction.git}{repo_quadrige} module local avec tests fonctionnels mais pas les bonnes versions de node et angular pour intégrer a géonature
    \item \href{https://mesprojets.developpement-durable.gouv.fr/faq}{version_locale} Ma version locale fonctionnelle avec les bonnes versions mais sans les tests mais sans les tests
    \item \href{https://github.com/basileandre056/gn_module_quadrige}{repo_final} Ma version finale en cours d'intégration dans géonature
\end{itemize}


\textbf{module \gls{plantnet}}
    \item \href{https://github.com/basileandre056/app_plantnet.git}{repo_plantnet}Mon client python local et le parseur de résultats.
\end{itemize}



\end{document}


