%================================================================================
%  TEMPLATE DE RAPPORT DE STAGE – ENIB + DEAL
%================================================================================

\documentclass[12pt,a4paper]{report}

% ------------------------------------------
% PACKAGES
% ------------------------------------------
\usepackage[utf8]{inputenc}
\usepackage[T1]{fontenc}
\usepackage[main=french,provide=*]{babel}
\usepackage{graphicx}
\usepackage{lmodern}
\usepackage{geometry}

\geometry{
  left=2cm,
  right=2cm,
  top=2cm,
  bottom=3.5cm
}

\usepackage{titlesec}
\usepackage{setspace}
\usepackage{fancyhdr}
\usepackage{color}
\usepackage{tocloft}
\usepackage{float}

% --- Glossaire ---
\usepackage[acronym]{glossaries}
\makeglossaries
\setacronymstyle{short-long}
% --- Glossaire des sigles ---
\newacronym{deal}{DEAL}{Direction de l’Environnement, de l’Aménagement et du Logement}
\newacronym{seb}{SEB}{Service Eau et Biodiversité}
\newacronym{ubio}{UBIO}{Unité Biodiversité}
\newacronym{enib}{ENIB}{École Nationale d’Ingénieurs de Brest}

\newacronym{crs}{CRS}{Communication Réseau Système}
\newacronym{cai}{CAI}{Conception d'Applications Interactives}

\newacronym{sinp}{SINP}{Système d’Information de l’iNventaire du Patrimoine naturel}
\newacronym{ofb}{OFB}{Office Français de la Biodiversité}
\newacronym{seor}{SEOR}{Société d’Études Ornithologiques de La Réunion}
\newacronym{ifremer}{Ifremer}{Institut Français de Recherche pour l’Exploitation de la Mer}

\newacronym{quadrige}{Quadrige}{Base nationale Ifremer pour le suivi du milieu marin et littoral}
\newacronym{geonature}{Géonature}{Application libre de gestion des données naturalistes}

\newacronym{eee}{EEE}{Espèces Exotiques Envahissantes}
\newacronym{erc}{ERC}{Éviter – Réduire – Compenser}
\newacronym{rse}{RSE}{Responsabilité Sociétale des Entreprises}




% --- Empêche printglossary de créer un saut de page + un titre automatique ---
\renewcommand*{\glossarysection}[2][]{%
    % Ne rien faire : pas de saut de page, pas de section automatique
}

% ------------------------------------------------------------------
% Style personnalisé du glossaire : ENIBStyle
% ------------------------------------------------------------------
\newglossarystyle{enibstyle}{
  \setglossarystyle{long}

  % Désactive toute numérotation de pages en fin d'entrée
  \renewcommand*{\glspostdescription}{}

  % Style : acronyme en bleu ENIB
  \renewcommand*{\glsnamefont}[1]{\textbf{\color{ENIBblue}##1}}

  % Structure du tableau (2 colonnes)
  \renewenvironment{theglossary}{
    \begin{longtable}{@{}p{0.25\textwidth}p{0.72\textwidth}@{}}
  }{
    \end{longtable}
  }

  % Chaque entrée du glossaire (sans lignes)
  \renewcommand{\glossentry}[2]{%
    \glsentryitem{##1}%
    \glstarget{##1}{\glossentryname{##1}} &
    \glossentrydesc{##1}%
    \tabularnewline
  }
}


\onehalfspacing





% Correction du warning fancyhdr
\setlength{\headsep}{12pt}  % valeur par défaut : 25pt
\setlength{\headheight}{42pt}
\setlength{\footskip}{34pt}
\addtolength{\topmargin}{-2pt}

% ------------------------------------------
% COULEURS
% ------------------------------------------
\definecolor{ENIBblue}{RGB}{5,55,105}



% ------------------------------------------
% STYLE TITRES (harmonisé)
% ------------------------------------------

% ===== CHAPITRES NUMÉROTÉS =====
\titleformat{\chapter}
  {\Huge\bfseries\color{ENIBblue}}
  {\thechapter.\ }{0.75em}{}
\titlespacing*{\chapter}{0pt}{0pt}{2em}

% Amélioration : éviter page blanche inutile avant le premier chapitre
\preto\chapter{%
  \ifnum\value{page}>1\clearpage\fi
}



% ===== CHAPITRES NON NUMÉROTÉS =====
\newcommand{\BlueChapter}[1]{%
  \clearpage
  \phantomsection
  {\Huge\bfseries\color{ENIBblue}#1}\par
  \addcontentsline{toc}{chapter}{#1}%
  \vspace{1em}
}

% ===== SECTIONS =====
\titleformat{\section}
  {\large\bfseries\color{black}}
  {\thesection.\ }{0.75em}{}
\titlespacing*{\section}{0pt}{1em}{0.8em}



% ===== SOUS-SECTIONS =====
\titleformat{\subsection}
  {\normalsize\bfseries\color{black}}
  {\thesubsection.\ }{0.75em}{}
\titlespacing*{\subsection}{0pt}{0.5em}{0.5em}

    
\usepackage[colorlinks=true,
            linkcolor=black,
            urlcolor=ENIBblue,
            citecolor=ENIBblue]{hyperref}

\usepackage[normalem]{ulem} % pour \uline sans casser \emph

\let\oldhref\href
\renewcommand{\href}[2]{\oldhref{#1}{\uline{#2}}}


% ------------------------------------------
% TABLE DES MATIÈRES — BLEU
% ------------------------------------------
\renewcommand{\cfttoctitlefont}{\Huge\bfseries\color{ENIBblue}}
\renewcommand{\cftloftitlefont}{\Huge\bfseries\color{ENIBblue}}

\renewcommand{\cftchapfont}{\bfseries\color{ENIBblue}}
\renewcommand{\cftchappagefont}{\bfseries\color{ENIBblue}}



% Style du titre de la table des matières
\renewcommand{\contentsname}{\color{ENIBblue}\Huge\bfseries Table des matières}

% Style du titre de la liste des figures
\renewcommand{\listfigurename}{\color{ENIBblue}\Huge\bfseries Liste des figures}

% Style français des annexes
\addto\captionsfrench{%
  \renewcommand{\appendixname}{Annexe}
}

% ------------------------------------------
% INFORMATIONS DU RAPPORT
% ------------------------------------------
\newcommand{\StudentName}{Basile André}
\newcommand{\StudentEmail}{b1andre@enib.fr}
\newcommand{\StageShortTitle}{Développement Géonature}
\newcommand{\StageTitle}{Développement d'un module externe de l'application Géonature pour importer les données de Quadrige}
\newcommand{\DatesStage}{Septembre – Décembre 2025}
\newcommand{\Entreprise}{DEAL Réunion}
\newcommand{\TuteurEntr}{Rémi Bouilly}
\newcommand{\TuteurAcad}{Jean-François Favennec}

% ------------------------------------------
% PAGE DE GARDE (COMPLÈTE)
% ------------------------------------------
\newcommand{\PageDeGarde}{
\begin{titlepage}
\thispagestyle{empty}

\noindent
\begin{minipage}[t][0.48\textheight][t]{\textwidth}
    \includegraphics[width=\textwidth, height=0.48\textheight]{images/logos/Corail-ile-de-la-reunion.jpg}

    \vspace*{-0.48\textheight}
    \begin{minipage}[t]{\textwidth}
        \vspace{0.6cm}

        \begin{minipage}{0.5\textwidth}
            \hspace{0.2cm}
            \includegraphics[height=1.7cm]{images/logos/logo_DEAL.jpg}
        \end{minipage}
        \begin{minipage}{0.48\textwidth}
            \raggedleft
            {\Large\color{white}\DatesStage\hspace{0.2cm}}
        \end{minipage}

        \vspace{1.6cm}
        \begin{minipage}{0.6\textwidth}
            \raggedright
            \hspace{0.2cm}{\Large\bfseries\color{white}\Entreprise}\\
            \hspace{0.2cm}{\Large\bfseries\color{white}ENIB}
        \end{minipage}

        \vspace{1.8cm}
        \begin{center}
            {\fontsize{28}{32}\selectfont\bfseries\color{white}\StageShortTitle}
        \end{center}

    \end{minipage} 
\end{minipage}   

\vspace{1.8cm}
\begin{minipage}{0.6\textwidth}
    \textbf{Stagiaire :} \StudentName\\[0.25cm]
    \textbf{Tuteur entreprise :} \TuteurEntr\\[0.25cm]
    \textbf{Tuteur académique :} \TuteurAcad
\end{minipage}

\vfill

\begin{center}
    \begin{minipage}{0.25\textwidth}
        \centering
        \includegraphics[height=1.1cm]{images/logos/Enib_inp_2025.png}
    \end{minipage}
    \begin{minipage}{0.45\textwidth}
        \centering
        \rule{\textwidth}{0.4pt}
    \end{minipage}
    \begin{minipage}{0.25\textwidth}
        \raggedleft {\small \StudentEmail}
    \end{minipage}

    \vspace{0.25cm}
    {\Large\bfseries \StudentName}
\end{center}

\end{titlepage}
}


% ------------------------------------------
% EN-TÊTE + PIED DE PAGE
% ------------------------------------------
\renewcommand{\headrulewidth}{0.8pt}
\renewcommand{\headrule}{\color{ENIBblue}\rule{\headwidth}{0.8pt}}

\fancypagestyle{standard}{
    \fancyhf{}

    \fancyhead[L]{\includegraphics[height=0.9cm]{images/logos/Enib_inp_2025.png}}
    \fancyhead[R]{\includegraphics[height=0.9cm]{images/logos/logo_DEAL.jpg}}

    \fancyfoot[C]{\color{ENIBblue}\rule{\textwidth}{0.4pt}}
    \fancyfoot[L]{\small\color{ENIBblue} \StudentEmail}
    \fancyfoot[C]{\small\color{ENIBblue} Stage Assistant Ingénieur\\[-0.2em]\textbf{\StageShortTitle}}
    \fancyfoot[R]{\small\color{ENIBblue} \thepage}
}

\makeatletter
\let\ps@plain\ps@standard
\makeatother

%===============================================================================
% DOCUMENT
%===============================================================================

% Styles et mise en page

\begin{document}

\PageDeGarde

\pagestyle{standard}
\setcounter{page}{1}

% début du contenu 
% ------------------------------------------

\BlueChapter{Remerciements}
MRC AU CHAT ET A ZIZOU
DEDICACE A PERSONNE FALLAIT ETRE LA


\BlueChapter{Résumé (FR)}
\noindent
Au cours de ma formation d’ingénieur généraliste à l’\gls{enib}, j’ai eu l’opportunité d’explorer
des domaines variés, dont l’informatique, le traitement des données et la communication réseau.
Ces enseignements, combinés au contexte environnemental actuel, m’ont donné l’envie d’explorer
comment les outils numériques peuvent répondre à des enjeux concrets, en particulier ceux liés
à l’environnement et au milieu marin, un domaine qui m’inspire depuis longtemps.
C’est cette motivation qui m’a conduit à choisir la \gls{deal},
où le développement d’outils
informatiques joue un rôle clé dans la valorisation et l’accessibilité des données,
offrant une occasion idéale d’allier mes compétences techniques à un sujet qui a du
sens.

\par\medskip

J’ai effectué mon stage au sein de la \gls{deal} Réunion, plus précisément
au \gls{seb}, à l’\gls{ubio}.
L’île de La Réunion présente des enjeux importants de suivi et de préservation de la biodiversité, en raison de la singularité de ses écosystèmes et des pressions 
qu’ils subissent (notamment les \gls{eee} et le développement du territoire). Ce contexte renforce la nécessité d’un suivi rigoureux des données naturalistes. 
\medskip

La \gls{deal} utilise \gls{geonature} pour centraliser et gérer ces données.
Le sujet principal de mon stage consistait à développer un module externe permettant d’y importer automatiquement les données marines issues de \gls{quadrige}. 
Ces données, essentielles pour le suivi environnemental littoral, 
représentent un enjeu majeur pour les missions de la DEAL.

\medskip

N'ayant pas suivi le module \gls{cai}, ce stage m’a permis d’acquérir des compétences complémentaires en développement
logiciel et applicatif, à travers la conception d’un module d’import, la manipulation de données et l’interfaçage de
systèmes hétérogènes, afin d’obtenir un outil simple d’utilisation, robuste et maintenable.

\medskip

Il m’a également offert une immersion dans le fonctionnement d’une administration publique de taille moyenne,
où l’alternance entre travail autonome et collaboration avec les agents en charge des données naturalistes et de
l’administration de \gls{geonature} s’est révélée particulièrement enrichissante. Cette expérience en environnement
professionnel réel a renforcé ma compréhension des enjeux liés à la gestion des données environnementales et a affiné
ma capacité à questionner la pertinence, l’utilité et le sens des projets auxquels je contribue, afin de m’assurer qu’ils
soient en accord avec mes valeurs et porteurs d’un impact positif.

\BlueChapter{Abstract (EN)}
\noindent
As part of my general engineering studies at \gls{enib}, I had the opportunity to explore
a wide range of fields, including computer science, 
 
data processing and network communications. These courses, 
combined with the current environmental context, strengthened my interest in understanding 
how digital tools can address real-world challenges — particularly those related to the environment 
and the marine domain, which has long inspired me. This motivation led me to choose \gls{deal}, where 
the development of digital tools plays a key role in enhancing the value and accessibility of 
environmental data, offering an ideal opportunity to align my technical skills with a topic that 
holds meaning for me.

\par\medskip

I completed my internship at \gls{deal} Réunion, more specifically within the \gls{seb} and the \gls{ubio}. 
Réunion Island faces major biodiversity challenges due to its high proportion of endemic 
species and the presence of particularly sensitive natural habitats. These ecosystems are under strong pressure, 
notably from \gls{eee} and ongoing land-use development, which reinforces the need for rigorous naturalist data.
\gls{deal} uses \gls{geonature} to centralise and manage these datasets.
The main objective of my internship was to develop an external module for this application, enabling the automated 
import of marine and coastal environmental data produced by \gls{ifremer}. These data are essential for biodiversity
monitoring and are accessed through \gls{quadrige},Ifremer’s dedicated information system. 

\par\medskip

Since I had not taken the \gls{cai} module, this internship allowed me to acquire complementary 
skills in software and application development through the design of an import module, data 
manipulation and the interfacing of heterogeneous systems, with the aim of producing a simple, robust 
and maintainable tool.

\par\medskip

It also offered me insight into the functioning of a medium-sized public administration, where the alternation 
between autonomous work and collaboration with staff in charge of naturalist data and the administration of \gls{geonature} 
proved particularly enriching. This experience in a real professional environment strengthened my understanding of the
challenges associated with environmental data management and refined my ability to reflect on the relevance, usefulness 
and purpose of the projects I contribute to, ensuring that they align with my values and have a positive impact.

% ------------------------------------------
% TABLE DES MATIÈRES 
% ------------------------------------------
\clearpage        % force un saut de page
\phantomsection   % assure une bonne ancre pour hyperref
\tableofcontents
\clearpage

% GLOSSAIRE
\BlueChapter{Glossaire}

\begingroup
\setlength{\LTpre}{0pt}
\setlength{\LTpost}{0pt}
\printglossary[type=\acronymtype, style=enibstyle, title={}]
\endgroup

% ------------------------------------------
% CHAPITRES NUMÉROTÉS
% ------------------------------------------

% ------------------------------------------
% PRESENTATION DE L'ENTREPRISE

\chapter{Présentation de l'entreprise}
\section{DEAL Réunion}
La Direction de l’Environnement, de l’Aménagement et du Logement (\gls{deal}) est le service déconcentré de l’État. 
Elle met en œuvre, à l’échelle régionale, les politiques publiques relevant du Ministère de la Transition 
Écologique et de la Cohésion des Territoires, ainsi que du Ministère de la Transition Énergétique. La Réunion 
est un territoire insulaire soumis à de fortes pressions environnementales et à des enjeux d’aménagement complexes. 
Dans ce contexte, la \gls{deal} occupe une place centrale. Elle se situe au croisement des questions d’environnement, 
de biodiversité, d’eau, d’urbanisme et de développement territorial.
\par\medskip

La \gls{deal} Réunion assure l’application des réglementations environnementales. Elle instruit les projets 
d’aménagement et gère les risques naturels. Elle suit la ressource en eau et met en œuvre les politiques de 
protection des milieux naturels. Elle travaille en étroite collaboration avec les collectivités et plusieurs 
établissements publics, tels que l’\gls{ofb}, le \gls{pnrun} ou l’\gls{ifremer}. Elle coopère 
aussi avec de nombreuses associations locales, comme la \gls{seor} ou la \gls{srepen}, qui contribuent activement 
au suivi et à la préservation de la biodiversité insulaire.
\par\medskip

Au sein de cette structure, le Service Eau et Biodiversité (\gls{seb}) porte les missions liées à la préservation 
des milieux aquatiques et terrestres. Il développe la connaissance des espèces, veille à leur protection et régule 
les activités susceptibles d’impacter la biodiversité. Le \gls{seb} se trouve ainsi au cœur des enjeux écologiques 
de l’île.
\par\medskip

Mon stage s’est déroulé au sein de l’Unité Biodiversité (\gls{ubio}). Cette unité est responsable du suivi 
des espèces et des habitats naturels. Elle gère et valorise les données naturalistes et instruit les dossiers
réglementaires liés à la biodiversité. Elle anime également le Système d’Information sur la Nature et les 
Paysages (\gls{sinp}) régional et assure la gestion de la plateforme \glsfirst{borbonica}. L’unité intervient 
aussi sur des thématiques transversales telles que les espèces exotiques envahissantes (\gls{eee}), la séquence 
\gls{erc} ou la diffusion des connaissances naturalistes. L’organisation interne de l’\gls{ubio} est présentée 
en annexe (Fig.~\ref{fig:organigramme-ubio}).
\par\medskip

L’organigramme interne montre une équipe composée de profils scientifiques, techniques et administratifs. 
Ces personnels travaillent de manière complémentaire pour répondre aux enjeux liés à la biodiversité du 
territoire. Mon stage s’inscrit dans cette dynamique, au sein du pôle dédié aux données naturalistes. Il 
contribue à la structuration et à la modernisation des outils numériques utilisés par la \gls{deal}.
\par\medskip

Cette présentation de la \gls{deal} et de son organisation permet de situer le contexte global de mon stage. 
Le chapitre suivant propose un \textbf{diagnostic \gls{rse}} de la structure. Il vise à évaluer ses pratiques 
au regard des enjeux sociaux, environnementaux et organisationnels.
\par\medskip
\section{Diagnostic RSE}
En raison des responsabilités qui lui sont confiées, la \gls{deal} Réunion doit intégrer des considérations sociales, 
environnementales et organisationnelles dans l’ensemble de ses pratiques. Le diagnostic qui suit s’appuie sur la norme 
ISO~26000 et sur les principes de la \gls{rne}, afin d’évaluer la manière dont 
la structure prend en compte ces enjeux.
\subsection{Risques et impacts des activités}
\noindent
\textbf{Impacts environnementaux.}
La \gls{deal} n’engendre pas d’impacts industriels directs. Cependant, ses activités reposent largement sur 
l’utilisation d’outils numériques intensifs : traitement de données naturalistes, alimentation des plateformes 
telles que le \gls{sinp}, \gls{geonature} ou \gls{borbonica}, ainsi que l’import automatisé de données issues de 
l’API \gls{quadrige}, permettant l’accès aux données marines produites par l'\gls{ifremer}. 
Ces pratiques impliquent une consommation énergétique, l’usage d’équipements informatiques et des besoins 
croissants en stockage et en traitement.  
\par\medskip

\textbf{Impacts sociétaux.}
Les décisions publiques s’appuient fortement sur la qualité des données produites et centralisées par la \gls{deal}. Une information environnementale fiable est essentielle pour les collectivités, les bureaux d’études, les associations ou les services de l’État.  
En facilitant l’intégration des données marines issues de \gls{quadrige}, mon travail renforce la transparence, l’égalité d’accès à la connaissance et la capacité des acteurs à prendre des décisions éclairées.
\par\medskip

\textbf{Impacts sociaux.}
Les équipes de la \gls{deal} évoluent dans un environnement de travail pluridisciplinaire mobilisant expertise scientifique, compétences réglementaires et gestion de données.  
Mon immersion au sein de l’\gls{ubio} m’a permis de contribuer à l’amélioration de certains processus internes, notamment par la simplification du flux de données marines. Cette contribution technique a eu pour effet indirect de diminuer la charge de travail liée aux imports manuels, renforçant ainsi l’efficacité opérationnelle du pôle.

\subsection*{Enquête RSE : actions mises en place}

La \gls{deal} engage plusieurs actions structurantes en cohérence avec la norme ISO~26000.

\par\medskip
\textbf{Environnement.}
L’administration encourage la réduction des déplacements, le recours aux outils numériques et une modernisation progressive des systèmes d’information. Mon module s’inscrit dans cette démarche de sobriété numérique en réduisant les traitements redondants et en automatisant les échanges de données.

\par\medskip
\textbf{Social et gouvernance.}
Les conditions de travail, la qualité du dialogue entre services et l’accompagnement des stagiaires constituent des axes importants. Les échanges réguliers au sein du pôle biodiversité ont permis d’adapter l’outil développé aux besoins réels, témoignant d’un fonctionnement concerté et d’une volonté d’amélioration continue.

\par\medskip
\textbf{Sociétal.}
La \gls{deal} contribue directement à la diffusion de données environnementales essentielles au suivi scientifique et aux politiques publiques. Les collaborations avec l'\gls{ofb}, l'\gls{ifremer}, le Parc national ou encore les associations naturalistes renforcent l’ancrage territorial de son action.  
Le module développé participe à cette dynamique en améliorant l’accessibilité et la qualité des données marines, dont dépend une partie de la stratégie environnementale régionale.

\subsection{Stratégie globale en matière de RSE}
\noindent
Bien qu’elle agisse dans un cadre réglementaire strict, la \gls{deal} cherche à renforcer la cohérence et la qualité 
de ses pratiques internes. Elle se trouve dans une dynamique de \textit{pré-conformité active}, allant au-delà des obligations 
minimales en matière de gestion des données, de modernisation numérique et de diffusion de l’information environnementale.

Mon travail s’inscrit dans cette stratégie : l’automatisation de l’import des données de \gls{quadrige} constitue une étape 
vers un système d’information plus robuste, plus efficace et plus aligné avec les principes de responsabilité numérique.

\subsection{Conclusion personnelle}
\noindent
Ce diagnostic montre que la \gls{deal} Réunion intègre progressivement les 
enjeux de la RSE et de la RNE au cœur de ses pratiques, malgré les contraintes 
propres à une administration publique.  
Mon stage m’a permis d’apporter une contribution concrète à cette dynamique, en 
améliorant la gestion des données marines et en facilitant le travail quotidien des agents. 
Cette expérience a renforcé ma conviction que le numérique, lorsqu’il est pensé de manière responsable, 
peut devenir un véritable levier au service de la transition écologique et de l’action publique.



% ------------------------------------------
% Organisation du stage
%-------------------------------------------
\chapter{Organisation du stage}
\section{Contexte du stage}
\noindent
Le \gls{sinp} est un dispositif collectif de mise en partage des données d’observations d’espèces sauvages sur l’île. 
Il a été mis en service à La Réunion en 2018, au travers de la plateforme Borbonica gérée conjointement par 
\gls{pnrun} et la \gls{deal}. 



Borbonica s’appuie sur un portail web, qui permet d’accéder aux différentes interfaces utilisateurs du SINP :
\begin{itemize}
    
    \item Borbonica obs : consultation cartographique des données du SINP. Il s’agit d’une 
    interface web basée sur le plugin Lizmap et reposant sur une base de données PostgreSQL. Cette plateforme offre un accès aux données 
    selon différents niveaux et profils (grand public, experts, validateurs, etc.) ; 
    
    \item Borbonica atlas : consultation des fiches espèces disponibles dans le SINP, sous forme de synthèse des principales informations 
    (observations, documents, photos, etc.). Il s’agit d’un outil issu de la solution open source GéoNature ;
    
    \item Borbonica stats : module statistique qui permet de consulter des tableaux de bord chiffrés sur le contenu de Borbonica 
    (données disponibles, principaux usages, etc.). Ce module s’appuie sur un projet Lizmap, exploitant des données de la base PostgreSQL.

\end{itemize}

Afin de pérenniser le SINP 974, un projet a été lancé en 2023, avec pour principaux objectifs la 
modernisation et la simplification du fonctionnement du système. Plusieurs nouvelles fonctionnalités 
vont être ajoutées, et certaines briques logicielles seront remplacés par de nouvelles solutions plus 
modernes et plus simples à utiliser et à maintenir. C’est le cas notamment de la solution Géonature qui 
est en cours de déploiement pour remplacer le périmètre de Borbonica obs.

\section{Objectifs du stage}
\noindent

À la suite de la modernisation engagée pour le SINP-974 et du déploiement progressif de GeoNature au sein de la \gls{deal}, 
le stage avait pour finalité d’apporter un appui opérationnel au renforcement de la chaîne d’acquisition et de gestion des données naturalistes. 
L’objectif principal était de faciliter l’intégration de données externes dans GeoNature et, plus largement, d’enrichir les informations 
accessibles via la plateforme régionale Borbonica.

Dans ce cadre, plusieurs objectifs spécifiques ont été définis :

\begin{itemize}

    \item développer un module externe GeoNature interfacé avec le système d’information Quadrige de l’\gls{ifremer}, 
        afin d’automatiser la récupération, le filtrage et la préparation des données issues du milieu marin ;

    \item poser les fondations techniques nécessaires à une future intégration des données produites par l’application 
        \gls{plantnet}, en réalisant un client Python et un premier outil de structuration des résultats ;

    \item documenter les procédures d’appel aux API et proposer une organisation des développements garantissant 
    leur maintenabilité et leur évolution dans le temps.

\end{itemize}

En complément du module Quadrige, une seconde orientation du stage concernait la préparation d’un futur connecteur GeoNature dédié aux données issues de \gls{plantnet}. 
Pl@ntNet produit chaque jour un volume important d’observations naturalistes géolocalisées, particulièrement pertinent pour le suivi des \gls{eee} végétales et des dynamiques de végétation.
Afin d’explorer la faisabilité de cette intégration, un travail préliminaire a consisté à :
\begin{itemize}
    \item étudier la documentation de l’API Pl@ntNet v3,
    \item concevoir un client Python robuste, capable d’interroger l’API selon plusieurs critères (taxon, polygone GeoJSON, plage temporelle),
    \item structurer les données extraites au format \textit{Darwin Core} afin d’assurer une compatibilité immédiate avec GeoNature et le SINP-974.
\end{itemize}
Ce travail prépare le développement d’un parser complet destiné au module \texttt{api2gn} de GeoNature.

Ces objectifs s’inscrivent pleinement dans la stratégie portée conjointement par la \gls{deal} Réunion et 
le \gls{pnrun}, visant à fiabiliser et à harmoniser les données du SINP-974 tout en renforçant la 
représentation de thématiques encore peu renseignées, comme le milieu marin. Ils ont guidé l’ensemble des actions 
menées durant le stage et constituent le cadre des travaux détaillés dans les sections suivantes.

\section{Méthodologie de travail}
\noindent

La méthodologie adoptée au cours du stage s’est construite progressivement, en fonction des contraintes techniques 
rencontrées et de l’avancement du projet. Le développement a d’abord été mené en local, faute d’accès immédiat aux 
serveurs du SINP-974. Cette situation a nécessité la mise en place d’un environnement de travail stable, capable de 
reproduire autant que possible les conditions de production. La séparation stricte entre le backend Python/Flask et 
le frontend Angular, conforme à l’architecture modulaire de GeoNature, a facilité cette organisation et permis de 
développer chaque composante de manière indépendante puis intégrée.

L’usage systématique de Git a constitué un élément central de la méthodologie. Il a permis de versionner les 
évolutions, de tester les fonctionnalités étape par étape, puis de transférer les développements vers les serveurs 
de la DEAL dès que l’accès a été ouvert. Ce mode de travail, fondé sur des allers-retours réguliers entre 
l’environnement local et l’environnement distant, a offert une solution efficace pour contourner les limitations 
imposées par la plateforme d’accès à distance utilisée au sein de l’administration.

Parallèlement, une attention particulière a été portée à la configuration de l’environnement de développement. 
Les contraintes liées au proxy ministériel et aux versions spécifiques des outils techniques (Python, Angular, Node)
ont rendu nécessaire la mise en place d’un mécanisme permettant d’adapter automatiquement la configuration réseau 
selon le contexte de connexion. L’utilisation de pyenv a également permis de stabiliser la version de Python utilisée, 
élément indispensable pour assurer la compatibilité avec GeoNature.

Enfin, l’ensemble du travail s’est appuyé sur une démarche itérative, combinant phases de développement, tests exploratoires 
et documentation progressive. Les appels à l’API Quadrige ont d’abord été éprouvés par des tests unitaires avec un client Python local,
avant d’être intégrés au module. De même, les premières interfaces Angular ont fait l’objet de tests fonctionnels 
en local afin de vérifier la cohérence et la fluidité de l’ergonomie.

Cette méthodologie, mêlant adaptation, rigueur technique et cycles courts d’expérimentation, a permis d’assurer la robustesse 
du développement malgré un environnement parfois instable et d’anticiper au mieux les étapes d’intégration dans l’infrastructure du SINP-974.

% ------------------------------------------
% Introduction au fonctionnement de Géonature
% ------------------------------------------
\chapter{Introduction à GeoNature}
\section{Principe général de GeoNature}

GeoNature est une application web dédiée à la gestion, la centralisation et la valorisation des
données naturalistes. Développé initialement en 2010 puis entièrement refondu en 2017, le
projet est aujourd’hui maintenu par le Parc national des Écrins. L’application repose sur une
architecture moderne combinant un backend Python/Flask et un frontend Angular, ce qui lui
permet d’assurer à la fois la saisie, la consultation, la validation et la restitution des données.

\begin{figure}[h!]
    \centering
    \includegraphics[width=0.85\textwidth]{images/back_front_geonature.png}
    \caption{Architecture générale du backend et du frontend de GeoNature}
    \label{fig:backfront}
\end{figure}

Le fonctionnement de GeoNature est modulaire : un noyau applicatif fournit les briques 
communes (API, référentiels, schéma de synthèse), et différents modules viennent étendre les 
fonctionnalités selon les besoins (Occtax, Occhab, Validation, Import, Export, etc.).  
La documentation officielle détaille l’ensemble de cette architecture
\href{https://docs.geonature.fr/}{(documentation officielle)}
ainsi que les sources du projet, disponibles sur 
\href{https://github.com/PnX-SI/GeoNature?tab=readme-ov-file}{GitHub}.


\section{Architecture technique}

GeoNature combine une partie serveur (backend) et une partie cliente (frontend) qui dialoguent
via une API REST.

Le backend est développé en Python à l’aide du framework Flask. Il assure l’ensemble des
traitements métiers, l’accès aux données et la gestion des opérations spatiales via PostgreSQL
et PostGIS. La gestion des utilisateurs et de leurs permissions repose sur UsersHub, qui fournit
l’authentification et le contrôle des accès. L’API exposée par le backend constitue le point
central de communication avec l’interface web.

Le frontend, développé en Angular, constitue l’interface visible par l’utilisateur. Il interroge
exclusivement l’API du backend pour afficher les formulaires, les cartes, les graphiques et les
données attributaires. Des bibliothèques comme Leaflet ou Bootstrap renforcent les
fonctionnalités cartographiques et l’ergonomie générale.  
Cette séparation nette entre backend et frontend garantit la stabilité et la modularité du
système, tout en facilitant les évolutions futures.


\section{Organisation interne et modules}

L’application est structurée autour de modules fonctionnels s’appuyant tous sur les mêmes
référentiels : taxonomie (TaxHub), nomenclatures, utilisateurs (UsersHub) et schéma de
synthèse. Les modules principaux — Occtax pour les observations, Occhab pour les habitats,
ou encore la Validation — s’intègrent directement au cœur applicatif.

Chaque module possède son propre schéma de base de données, son API et ses composants
Angular. Cette organisation modulaire permet de faire évoluer GeoNature, d'ajouter de
nouveaux protocoles d’acquisition ou de développer des extensions externes.  
La documentation décrit précisément les bonnes pratiques et l’architecture à respecter pour
développer un \href{https://docs.geonature.fr/development.html\#developper-un-module-externe}{module GeoNature externe}.


% ------------------------------------------
% Developpement du module d'import Quadrige
% ------------------------------------------
\chapter{Developpement du module d'import Quadrige}
\section{Contexte du projet}

Le \gls{sinp}, déployé à La Réunion au sein de l’application Borbonica, comporte encore très peu de
données issues du milieu marin : selon le bilan 2024, moins de 10~\% des observations concernent
des taxons marins. Cette sous-représentation constitue un frein à la connaissance de la
biodiversité littorale et sous-marine de l’île.  

À l’inverse, le système d’information \gls{quadrige}, maintenu par l’Ifremer, rassemble un volume
considérable de données environnementales et biologiques collectées en mer. La mise en place,
fin 2024, d’une nouvelle API GraphQL rend désormais possible l’interrogation directe et
structurée de ces données, ouvrant la voie à une intégration plus large dans les outils du \gls{sinp}.

Dans ce contexte, le projet a consisté à développer un \textbf{module externe GeoNature} capable
d’interroger l'API de Quadrige, de récupérer les programmes pertinents pour La Réunion, d’en
extraire les observations utiles, puis de produire un fichier d’import conforme aux spécifications
du modèle \gls{sinp}. Le module devait s’intégrer naturellement à l’interface existante de GeoNature,
offrir une exploration intuitive des programmes Quadrige, et préparer les données en vue de leur
intégration finale dans Borbonica.  
Ce développement s’inscrit ainsi dans la dynamique d’amélioration continue du \gls{sinp} Réunion et
vise à renforcer la représentativité des données marines dans la base régionale.

\section{Périmètre fonctionnel}

Le module d’import devait couvrir l’ensemble de la chaîne d’acquisition : de la découverte des
programmes Quadrige jusqu’à la production d’un fichier structuré pour GeoNature.  
La première étape consistait à interroger l’API en mode authentifié afin d’obtenir la liste des
programmes disponibles pour un utilisateur donné. Un filtrage automatique sur un périmètre
géographique — principalement La Réunion, mais extensible à d’autres territoires comme les Îles
Éparses — permettait d’isoler les programmes pertinents. Une interface dédiée intégrée à
GeoNature offrait ensuite la possibilité de rechercher des programmes, d’affiner l’affichage par
mots-clés et de sélectionner ceux à importer.

Une fois les programmes choisis, l’utilisateur pouvait définir les filtres à appliquer aux données
(la période d'intérêt, les champs souhaités, ou encore la reprise des stations déjà extraites).  
Le module interrogeait alors l’API Quadrige pour récupérer les observations correspondantes.
Seuls les champs utiles au modèle \gls{sinp} étaient extraits : identifiants des programmes et stations,
localisation géographique, taxon observé, date, ainsi que les métadonnées essentielles (auteur,
organisme, méthode d’acquisition, etc.). Une transformation était appliquée pour obtenir une
structure compatible avec les mécanismes d’import de GeoNature.

Enfin, le module produisait un fichier CSV intermédiaire, destiné à être importé via
l’infrastructure existante de GeoNature. Chaque opération d’import était consignée dans un
historique affiché dans un second onglet, permettant de suivre les actions réalisées, leur date,
leur statut et les éventuelles erreurs rencontrées. L’ensemble du module était réservé aux
administrateurs, conformément aux pratiques habituelles de contrôle des imports dans GeoNature.

\section{Spécifications techniques}

Techniquement, le module repose sur une architecture conforme à celle de GeoNature. Le
\textbf{backend}, développé en Python avec le framework Flask, se charge de communiquer avec
l’API Quadrige, d’effectuer les transformations nécessaires sur les données et de générer les
fichiers d’export. Le \textbf{frontend}, construit en Angular et Bootstrap, s’intègre à l’interface
existante de GeoNature et fournit les pages de sélection des programmes ainsi que l’historique
des imports.

La communication avec Quadrige s’effectue via des requêtes GraphQL, qui permettent d’obtenir
uniquement les champs nécessaires, réduisant ainsi le volume de données transférées.  
L’authentification repose sur un token fourni par l’Ifremer et configurable directement dans les
paramètres du module. Les autres réglages comprennent l’URL de l’API, le périmètre
géographique par défaut, ainsi que le mapping entre les champs Quadrige et les champs
attendus par le schéma de synthèse de GeoNature.

Un soin particulier a été apporté à la gestion des erreurs : vérification des réponses HTTP,
détection des timeouts, contrôle de la validité des données, et affichage de messages explicites à
l’utilisateur. Toutes les actions — appels API, transformations, exports — sont consignées dans les
logs du module, conformément aux pratiques de GeoNature.

\section{Contraintes, dépendances et livrables}

La réussite du projet dépendait principalement de l’accessibilité de l’API Quadrige, de la
stabilité de ses services, et de la disponibilité d’une documentation actualisée. L’intégration dans
GeoNature nécessitait également de respecter la structure modulaire du noyau applicatif et les
contraintes du modèle \gls{sinp}.

Les livrables attendus comprenaient le code source du module, un fichier de configuration, une
documentation à destination des administrateurs (installation, configuration, maintenance) ainsi
qu’un guide d’utilisation orienté métier.

Plusieurs évolutions ont été envisagées : automatisation des imports périodiques, gestion des
imports incrémentaux, intégration à la gestion des droits de GeoNature et, à plus long terme,
publication du module dans le catalogue officiel des extensions GeoNature.

\section{Liens utiles}

Pour accompagner le développement, plusieurs ressources officielles ont été mobilisées :

\textbf{Quadrige}
\begin{itemize}

    \item \href{https://quadrige-core.ifremer.fr/api/extraction/doc?doc=standard&lang=fr&name=result&type=standard}{Documentation API}
    \item \href{https://quadrige.ifremer.fr/support/Mes-donnees/J-extrais-mes-donnees/J-interroge-l-API-pour-extraire-mes-donnees/Je-regarde-des-videos-de-demo-sur-l-API}{Tutoriels vidéo}  
    \item \href{https://quadrige-app.ifremer.fr/}{Accès à l’application}

\end{itemize}

\textbf{GeoNature}
\begin{itemize}
    \item \href{https://github.com/PnX-SI/GeoNature}{Dépôt GitHub}  
    \item \href{https://docs.geonature.fr/}{Documentation générale}  
    \item \href{https://docs.geonature.fr/development.html\#developper-un-module-externe}{Guide de développement des modules externes}  
\end{itemize}

\bigskip

\noindent
L’ensemble de ces éléments a permis de définir précisément le périmètre et les modalités du
développement. Le chapitre suivant présente la réalisation concrète du module, depuis
l’organisation du travail jusqu’aux phases de tests et de validation des fonctionnalités mises en
place.



% ------------------------------------------
% Travaux réalisés et compétences acquises
% ------------------------------------------
\chapter{Travaux réalisés et compétences acquises}
\section{Travaux réalisés}

Les travaux réalisés au cours du stage se sont articulés autour de deux axes
principaux.

Le premier axe concerne le développement d’un module externe pour GeoNature,
dédié à l’exploration et à l’extraction de données marines issues du système
d’information \gls{quadrige}.

Le second axe porte sur un travail complémentaire autour de l’intégration
de données participatives produites par l’application \gls{plantnet}.
L’objectif était de poser les bases techniques d’un futur parseur,
adapté à une intégration automatisée dans GeoNature.

L’ensemble des développements a été mené dans une logique d’intégration
au système d’information existant.
Les choix effectués prennent en compte les contraintes techniques,
organisationnelles et métier propres au \gls{sinp}.

\subsection{Vue d’ensemble des développements}

Les développements réalisés comprennent plusieurs volets complémentaires.

Ils incluent la conception d’un module GeoNature externe,
interfacé avec l’\gls{api} de \gls{quadrige},
permettant d’explorer les programmes disponibles
et d’en extraire les données associées.

Ils comprennent également la mise en place d’une interface frontend
intégrée à GeoNature.
Cette interface permet de rechercher, filtrer et sélectionner
des programmes et des données à extraire.

Les extractions produisent des fichiers intermédiaires documentés,
destinés à une intégration différée dans GeoNature.
Cette approche permet de conserver un contrôle sur les données produites
avant toute insertion en base.

Enfin, une documentation technique et utilisateur a été rédigée
tout au long du projet.
Elle décrit les principes de fonctionnement du module,
les procédures d’installation,
ainsi que les modalités d’utilisation et de maintenance.

L’ensemble de ces travaux a été conçu pour produire un outil fonctionnel,
maintenable et réutilisable.
Le module est destiné à être repris ou étendu par les équipes de la
\gls{deal} ou par d’autres structures partenaires.

\subsection{Travaux principaux : module \gls{quadrige}}

Les travaux principaux du stage ont porté sur le développement
du module \gls{quadrige} pour GeoNature.

Ce module permet d’interroger l’\gls{api} GraphQL mise à disposition par
l’Ifremer afin d’identifier les programmes pertinents
pour un périmètre géographique donné.
Il offre ensuite la possibilité d’extraire les données associées
selon des critères définis par l’utilisateur.

Le module repose sur une architecture découplée,
séparant clairement le backend et le frontend.
Le backend est responsable des échanges avec l’\gls{api},
de l’authentification,
du suivi des extractions
et de la gestion des fichiers produits.
Le frontend se limite au pilotage des opérations
et à la visualisation des résultats.

Les extractions réalisées produisent des fichiers intermédiaires,
notamment des fichiers \gls{csv} et des archives compressées.
Ces fichiers sont accompagnés de métadonnées décrivant précisément
les paramètres d’extraction utilisés.
Cette organisation garantit la traçabilité des opérations
et facilite l’analyse des résultats.

Le module a été conçu comme une brique préparatoire.
Il ne réalise pas directement l’import des données en base GeoNature,
mais vise à sécuriser les étapes amont
et à faciliter une intégration future
dans le respect des règles du \gls{sinp}.

\subsection{Travaux complémentaires : \gls{plantnet}}

En parallèle du développement du module \gls{quadrige},
un travail complémentaire a été mené autour de l’intégration
des données issues de \gls{plantnet}.

L’objectif de ce travail était d’évaluer la faisabilité technique
d’une intégration automatisée de données participatives
dans GeoNature.
Il s’agissait également de poser les bases d’un futur parseur
utilisable par le module \gls{api2gn}.

Un client Python générique a été développé
pour interroger l’\gls{api} \gls{plantnet},
récupérer les observations
et enregistrer les réponses brutes au format \gls{json}.

Ces données ont ensuite été transformées
en fichiers \gls{csv} conformes au standard Darwin Core.
Une attention particulière a été portée à la normalisation des champs,
à la gestion des dates,
aux coordonnées géographiques
et à la génération d’identifiants stables.

Enfin, un pipeline automatisé, intégrant des mécanismes de validation et de contrôle
reposant sur une configuration externalisée.
Ce pipeline constitue aujourd’hui la base technique
du parseur \gls{plantnet} intégré au module \gls{api2gn}.

Bien que complémentaire au module \gls{quadrige},
ce travail s’inscrit dans la même logique globale.
Il vise à démontrer la capacité de GeoNature
à intégrer des sources de données hétérogènes,
tout en respectant les exigences de qualité,
de traçabilité et de cohérence du \gls{sinp}.

\section{Difficultés rencontrées et solutions apportées}
\noindent
Le déroulement du stage a été marqué par un ensemble de contraintes techniques qui ont largement influencé 
l’organisation du travail. L’absence d’une installation locale stable de GeoNature a constitué l’un des principaux obstacles : 
les incompatibilités entre les versions de Debian, les erreurs liées au serveur applicatif et les difficultés rencontrées avec 
Docker ont rendu nécessaire une approche alternative, centrée sur une instance de travail personnalisée. Cette situation a été 
accentuée par les limitations du réseau de la DEAL, où la présence d’un proxy strict perturbait l’utilisation de nombreux outils 
de développement. La mise en place d’un mécanisme dynamique de gestion du proxy a progressivement permis de stabiliser l’environnement.

La migration des serveurs du SINP-974 vers un nouvel hébergeur a également retardé l’accès à un environnement de test réaliste. 
Pendant plusieurs semaines, le développement a dû être poursuivi sans possibilité de déployer ou de tester directement sur 
l’environnement cible. Une fois l’accès ouvert via une machine virtuelle bastion, les premières manipulations ont été freinées 
par les restrictions imposées par l’outil d’accès à distance. L’usage de Git comme canal de transfert a alors été déterminant 
pour poursuivre les tests et ajuster le module.

Ces difficultés ont nécessité une grande capacité d’adaptation et ont conduit à affiner l’organisation du projet : clarification 
des environnements, adoption d’outils de gestion de version adaptés, multiplication des tests exploratoires et documentation 
progressive pour sécuriser les phases suivantes d’intégration.


\paragraph{Difficultés spécifiques au travail sur Pl@ntNet}

Le développement du client Pl@ntNet a également présenté plusieurs défis techniques. 
L’API renvoie une structure JSON très hétérogène selon les taxons, les médias associés ou les métadonnées disponibles, ce qui a nécessité la création d’un parseur flexible, capable d’absorber des variations importantes de schéma sans échouer.

Par ailleurs, le standard Darwin Core impose un modèle strict qui ne correspond pas directement aux champs fournis par Pl@ntNet. 
Il a donc fallu définir un mapping robuste et configurable pour garantir la compatibilité avec GeoNature et Borbonica.

Enfin, l’intégration du parser dans le module \texttt{api2gn} a nécessité de modifier certaines migrations SQL existantes, qui empêchaient l’installation correcte du module. 
Ces ajustements feront l’objet d’un retour formalisé aux développeurs officiels de GeoNature afin d’améliorer la compatibilité du module avec des sources de données externes diversifiées.

\section{Compétences acquises}
\noindent

Ce stage m’a permis de développer une maîtrise approfondie de l’écosystème technique de GeoNature, tant sur le plan backend 
(Python, Flask, SQLAlchemy, Marshmallow) que frontend (Angular, TypeScript). J’ai acquis une expérience concrète dans l’utilisation 
d’API REST et GraphQL, dans la mise en place de tests unitaires et fonctionnels, ainsi que dans la gestion rigoureuse 
d’environnements soumis à des contraintes fortes.

Ce travail m’a également permis de me familiariser avec les pratiques du développement open source,
notamment via l’utilisation de forks GitHub, la gestion des issues et la rédaction de documentation
technique destinée à des utilisateurs externes.


Au-delà des aspects techniques, cette expérience m’a amené à travailler en autonomie tout en collaborant étroitement 
avec des acteurs variés : agents en charge des données, administrateurs réseau, prestataires et agents du Parc national. 
J’ai pu renforcer mes compétences en organisation de projet, en documentation technique et en adaptation à des environnements 
complexes, autant de dimensions essentielles dans un contexte 
d’ingénierie logicielle appliquée aux enjeux environnementaux.

% ------------------------------------------  

% ------------------------------------------
% PERSPECTIVES ET CONCLUSION
% ------------------------------------------
\chapter{Perspectives}



% ------------------------------------------
% CONCLUSION
% ------------------------------------------
\chapter{Conclusion}


% ------------------------------------------
% BIBLIOGRAPHIE
% ------------------------------------------
\BlueChapter{Bibliographie}
\input{chapters/bibliographie}
% ------------------------------------------
% ANNEXES
% ------------------------------------------
\BlueChapter{Annexes}
\begin{figure}[H]
    \centering
    \includegraphics[width=0.9\textwidth]{images/organigramme_ubio.pdf}
    \caption{Organigramme de l'Unité Biodiversité (UBIO) – SEB / \gls{deal} Réunion}
    \label{fig:organigramme-ubio}
\end{figure}


\begin{figure}[h!]
    \centering
    \includegraphics[width=0.85\textwidth]{images/back_front_geonature.png}
    \caption{Architecture générale du backend et du frontend de GeoNature}
    \label{fig:backfront}
\end{figure}



\section{Liens utiles}

Pour accompagner le développement, plusieurs ressources officielles ont été mobilisées :

\textbf{Quadrige}
\begin{itemize}

    \item \href{https://quadrige-core.ifremer.fr/api/extraction/doc?doc=standard&lang=fr&name=result&type=standard}{Documentation API}
    \item \href{https://quadrige.ifremer.fr/support/Mes-donnees/J-extrais-mes-donnees/J-interroge-l-API-pour-extraire-mes-donnees/Je-regarde-des-videos-de-demo-sur-l-API}{Tutoriels vidéo}  
    \item \href{https://quadrige-app.ifremer.fr/}{Accès à l’application}

\end{itemize}

\textbf{GeoNature}
\begin{itemize}
    \item \href{https://github.com/PnX-SI/GeoNature}{Dépôt GitHub}  
    \item \href{https://docs.geonature.fr/}{Documentation générale}  
    \item \href{https://docs.geonature.fr/development.html\#developper-un-module-externe}{Guide de développement des modules externes}  
\end{itemize}

\bigskip

\noindent
L’ensemble de ces éléments a permis de définir précisément le périmètre et les modalités du
développement. Le chapitre suivant présente la réalisation concrète du module, depuis
l’organisation du travail jusqu’aux phases de tests et de validation des fonctionnalités mises en
place.



% ------------------------------------------
% WEBOGRAPHIE
% ------------------------------------------
\BlueChapter{Webographie}
Pour accompagner le développement, plusieurs ressources officielles ont été mobilisées :

\textbf{Quadrige}
\begin{itemize}
    \item \href{https://quadrige-core.ifremer.fr/api/extraction/doc?doc=standard&lang=fr&name=result&type=standard}{Documentation API}
    \item \href{https://quadrige.ifremer.fr/support/Mes-donnees/J-extrais-mes-donnees/J-interroge-l-API-pour-extraire-mes-donnees/Je-regarde-des-videos-de-demo-sur-l-API}{Tutoriels vidéo}  
    \item \href{https://quadrige-app.ifremer.fr/}{Accès à l’application}

\end{itemize}

\textbf{GeoNature}
\begin{itemize}
    \item \href{https://geonature.fr/}{Site officiel}
    \item \href{https://github.com/PnX-SI/GeoNature}{Dépôt GitHub}  
    \item \href{https://docs.geonature.fr/}{Documentation générale}  
    \item \href{https://docs.geonature.fr/development.html\#developper-un-module-externe}{Guide de développement des modules externes}  
\end{itemize}

\textbf{DEAL Réunion}
\begin{itemize}
    \item \href{https://www.reunion.developpement-durable.gouv.fr/}{Site officiel}  
    \item \href{https://deal.reunion.developpement-durable.gouv.fr/La-DEAL-Reunion-r839.html}{Présentation de la DEAL Réunion}

\end{itemize}


\textbf{module quadrige}
\begin{itemize}
    \item \href{https://github.com/basileandre056/geonature_quadrige_extraction.git}{repo_quadrige} module local avec tests fonctionnels mais pas les bonnes versions de node et angular pour intégrer a géonature
    \item \href{https://mesprojets.developpement-durable.gouv.fr/faq}{version_locale} Ma version locale fonctionnelle avec les bonnes versions mais sans les tests mais sans les tests
    \item \href{https://github.com/basileandre056/gn_module_quadrige}{repo_final} Ma version finale en cours d'intégration dans géonature
\end{itemize}


\textbf{module \gls{plantnet}}
    \item \href{https://github.com/basileandre056/app_plantnet.git}{repo_plantnet}Mon client python local et le parseur de résultats.
\end{itemize}



\end{document}