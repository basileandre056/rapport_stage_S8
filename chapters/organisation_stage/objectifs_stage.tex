\noindent

À la suite de la modernisation engagée pour le SINP-974 et du déploiement progressif de GeoNature au sein de la \gls{deal}, 
le stage avait pour finalité d’apporter un appui opérationnel au renforcement de la chaîne d’acquisition et de gestion des données naturalistes. 
L’objectif principal était de faciliter l’intégration de données externes dans GeoNature et, plus largement, d’enrichir les informations 
accessibles via la plateforme régionale Borbonica.

Dans ce cadre, plusieurs objectifs spécifiques ont été définis :

\begin{itemize}

    \item développer un module externe GeoNature interfacé avec le système d’information Quadrige de l’\gls{ifremer}, 
        afin d’automatiser la récupération, le filtrage et la préparation des données issues du milieu marin ;

    \item poser les fondations techniques nécessaires à une future intégration des données produites par l’application 
        \gls{plantnet}, en réalisant un client Python et un premier outil de structuration des résultats ;

    \item documenter les procédures d’appel aux API et proposer une organisation des développements garantissant 
    leur maintenabilité et leur évolution dans le temps.

\end{itemize}

En complément du module Quadrige, une seconde orientation du stage concernait la préparation d’un futur connecteur GeoNature dédié aux données issues de \gls{plantnet}. 
Pl@ntNet produit chaque jour un volume important d’observations naturalistes géolocalisées, particulièrement pertinent pour le suivi des \gls{eee} végétales et des dynamiques de végétation.
Afin d’explorer la faisabilité de cette intégration, un travail préliminaire a consisté à :
\begin{itemize}
    \item étudier la documentation de l’API Pl@ntNet v3,
    \item concevoir un client Python robuste, capable d’interroger l’API selon plusieurs critères (taxon, polygone GeoJSON, plage temporelle),
    \item structurer les données extraites au format \textit{Darwin Core} afin d’assurer une compatibilité immédiate avec GeoNature et le SINP-974.
\end{itemize}
Ce travail prépare le développement d’un parser complet destiné au module \texttt{api2gn} de GeoNature.

Ces objectifs s’inscrivent pleinement dans la stratégie portée conjointement par la \gls{deal} Réunion et 
le \gls{pnrun}, visant à fiabiliser et à harmoniser les données du SINP-974 tout en renforçant la 
représentation de thématiques encore peu renseignées, comme le milieu marin. Ils ont guidé l’ensemble des actions 
menées durant le stage et constituent le cadre des travaux détaillés dans les sections suivantes.
