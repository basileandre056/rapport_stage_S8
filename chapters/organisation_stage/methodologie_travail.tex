\noindent

La méthodologie adoptée au cours du stage s’est construite progressivement, en fonction des contraintes techniques 
rencontrées et de l’avancement du projet. Le développement a d’abord été mené en local, faute d’accès immédiat aux 
serveurs du SINP-974. Cette situation a nécessité la mise en place d’un environnement de travail stable, capable de 
reproduire autant que possible les conditions de production. La séparation stricte entre le backend Python/Flask et 
le frontend Angular, conforme à l’architecture modulaire de GeoNature, a facilité cette organisation et permis de 
développer chaque composante de manière indépendante puis intégrée.

L’usage systématique de Git a constitué un élément central de la méthodologie. Il a permis de versionner les 
évolutions, de tester les fonctionnalités étape par étape, puis de transférer les développements vers les serveurs 
de la DEAL dès que l’accès a été ouvert. Ce mode de travail, fondé sur des allers-retours réguliers entre 
l’environnement local et l’environnement distant, a offert une solution efficace pour contourner les limitations 
imposées par la plateforme d’accès à distance utilisée au sein de l’administration.

Parallèlement, une attention particulière a été portée à la configuration de l’environnement de développement. 
Les contraintes liées au proxy ministériel et aux versions spécifiques des outils techniques (Python, Angular, Node)
ont rendu nécessaire la mise en place d’un mécanisme permettant d’adapter automatiquement la configuration réseau 
selon le contexte de connexion. L’utilisation de pyenv a également permis de stabiliser la version de Python utilisée, 
élément indispensable pour assurer la compatibilité avec GeoNature.

Enfin, l’ensemble du travail s’est appuyé sur une démarche itérative, combinant phases de développement, tests exploratoires 
et documentation progressive. Les appels à l’API Quadrige ont d’abord été éprouvés par des tests unitaires avec un client Python local,
avant d’être intégrés au module. De même, les premières interfaces Angular ont fait l’objet de tests fonctionnels 
en local afin de vérifier la cohérence et la fluidité de l’ergonomie.

Cette méthodologie, mêlant adaptation, rigueur technique et cycles courts d’expérimentation, a permis d’assurer la robustesse 
du développement malgré un environnement parfois instable et d’anticiper au mieux les étapes d’intégration dans l’infrastructure du SINP-974.