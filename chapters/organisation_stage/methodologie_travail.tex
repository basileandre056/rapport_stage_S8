\noindent

La méthodologie adoptée au cours du stage s’est construite de manière progressive et pragmatique. 
Elle a été définie en tenant compte des contraintes techniques du projet, du cadre institutionnel de la DEAL 
et de l’organisation particulière de l’équipe. Le projet étant mené uniquement à deux personnes — mon maître de stage, 
Rémi Bouilly, et moi-même — il n’était pas pertinent de mettre en place une méthodologie formelle lourde. Le cadre de travail 
retenu s’inspire néanmoins des principes des méthodes agiles, tout en restant volontairement souple.

Le développement du module Quadrige et des outils associés a ainsi été conduit selon une démarche itérative et incrémentale. 
Le travail était organisé en cycles courts, assimilables à des sprints d’une durée moyenne d’environ une semaine. Chaque sprint 
correspondait à un ensemble cohérent de fonctionnalités, définies en amont, puis développées, testées et ajustées avant d’engager 
le cycle suivant. Cette organisation permettait d’obtenir régulièrement des résultats fonctionnels, même partiels, tout en 
identifiant rapidement les difficultés techniques ou fonctionnelles.

Dans ce contexte de travail en effectif réduit, il n’était ni nécessaire ni utile de mettre en place des rituels agiles classiques 
tels que les daily meetings ou les revues de sprint formelles. En revanche, des points réguliers étaient organisés une à deux fois 
par semaine avec mon maître de stage. Ces échanges permettaient de faire un état d’avancement, de valider les choix techniques effectués 
et d’ajuster les priorités à court terme en fonction des besoins métier. Ils ont joué un rôle central dans le bon déroulement du projet, 
en assurant une compréhension partagée des objectifs tout en me laissant une large autonomie dans la mise en œuvre.

Le cadrage initial du stage s’est appuyé sur deux cahiers des charges distincts, fournis par mon maître de stage. Le premier concernait 
le développement du module externe GeoNature interfacé avec Quadrige, tandis que le second portait sur la conception d’un parser dédié 
aux données issues de Pl@ntNet et à leur structuration au format Darwin Core. Ces documents définissaient le périmètre fonctionnel 
attendu, les principales contraintes techniques ainsi que les livrables envisagés. Ils ont servi de fil conducteur tout au long du 
stage, tout en restant suffisamment flexibles pour intégrer des ajustements liés aux contraintes rencontrées ou à l’évolution du projet.




Afin de suivre l’avancement global du projet et de visualiser les grandes phases du stage, un diagramme de Gantt a été utilisé. 
Il a permis de planifier les principales étapes, depuis la prise en main de l’environnement GeoNature jusqu’aux phases de tests, 
de documentation et de préparation des livrables. Cet outil offrait une vision synthétique du projet et facilitait l’anticipation 
des périodes plus sensibles, notamment lors des phases de migration des serveurs ou d’attente d’accès aux environnements distants.

À un niveau plus opérationnel, le travail quotidien était organisé sous forme de tâches unitaires, réparties selon un statut simple : 
à faire, en cours ou terminées. Cette organisation, inspirée des tableaux de type Kanban, permettait de visualiser rapidement l’état 
d’avancement, de prioriser les actions et de conserver une trace claire du travail réalisé. Elle s’est révélée particulièrement adaptée 
à un projet individuel encadré, en apportant une structure légère sans alourdir inutilement le processus.

La gestion du code source reposait intégralement sur l’utilisation de Git. L’ensemble des développements a été versionné dès les 
premières phases du projet, aussi bien pour le module Quadrige que pour les outils liés à Pl@ntNet. Les évolutions étaient enregistrées 
de manière régulière, avec des messages de commit explicites. Cette pratique a permis de sécuriser les développements, de tester différentes 
approches sans risque de perte de code et de faciliter le transfert vers les serveurs de la DEAL une fois l’accès ouvert. Dans un contexte 
où l’accès aux environnements distants était limité et parfois instable, Git a également servi de principal moyen de synchronisation entre 
l’environnement local et l’environnement distant.

Le développement a d’abord été mené intégralement en local, en l’absence d’un accès immédiat aux serveurs du SINP-974. Cette situation 
a nécessité la mise en place d’un environnement de travail stable, capable de reproduire au mieux les conditions de production. La séparation 
stricte entre le backend Python/Flask et le frontend Angular, conforme à l’architecture modulaire de GeoNature, a facilité cette 
organisation et permis de développer chaque composante de manière indépendante.

Les contraintes liées au proxy ministériel et aux versions spécifiques des outils techniques ont également nécessité une attention 
particulière à la configuration de l’environnement. L’utilisation de pyenv a permis de stabiliser la version de Python utilisée, condition indispensable pour garantir la compatibilité avec GeoNature et ses dépendances.

Enfin, l’ensemble du travail s’est appuyé sur une démarche de validation progressive. Les appels à l’API Quadrige ont d’abord été testés 
à l’aide de scripts Python indépendants, avant leur intégration dans le module GeoNature. De la même manière, les premières interfaces 
Angular ont été testées en local afin de vérifier leur cohérence fonctionnelle et leur ergonomie avant toute intégration plus large.

Cette méthodologie, mêlant principes agiles, rigueur technique et adaptation constante au contexte, a permis d’assurer la 
robustesse des développements malgré un environnement parfois contraint. Elle a également facilité l’anticipation des étapes 
futures d’intégration et de reprise du projet par la DEAL ou ses partenaires.