\noindent
Le \gls{sinp} est un dispositif collectif de mise en partage des données d’observations d’espèces sauvages sur l’île. 
Il a été mis en service à La Réunion en 2018, au travers de la plateforme Borbonica gérée conjointement par 
\gls{pnrun} et la \gls{deal}. 



Borbonica s’appuie sur un portail web, qui permet d’accéder aux différentes interfaces utilisateurs du SINP :
\begin{itemize}
    
    \item Borbonica obs : consultation cartographique des données du SINP. Il s’agit d’une 
    interface web basée sur le plugin Lizmap et reposant sur une base de données PostgreSQL. Cette plateforme offre un accès aux données 
    selon différents niveaux et profils (grand public, experts, validateurs, etc.) ; 
    
    \item Borbonica atlas : consultation des fiches espèces disponibles dans le SINP, sous forme de synthèse des principales informations 
    (observations, documents, photos, etc.). Il s’agit d’un outil issu de la solution open source GéoNature ;
    
    \item Borbonica stats : module statistique qui permet de consulter des tableaux de bord chiffrés sur le contenu de Borbonica 
    (données disponibles, principaux usages, etc.). Ce module s’appuie sur un projet Lizmap, exploitant des données de la base PostgreSQL.

\end{itemize}

Afin de pérenniser le SINP 974, un projet a été lancé en 2023, avec pour principaux objectifs la 
modernisation et la simplification du fonctionnement du système. Plusieurs nouvelles fonctionnalités 
vont être ajoutées, et certaines briques logicielles seront remplacés par de nouvelles solutions plus 
modernes et plus simples à utiliser et à maintenir. C’est le cas notamment de la solution Géonature qui 
est en cours de déploiement pour remplacer le périmètre de Borbonica obs.
