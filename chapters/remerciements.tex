Je tiens tout d’abord à remercier mon tuteur en entreprise, M. Rémi Bouilly, pour son accompagnement tout au long de ce stage.
Il m’a accordé sa confiance dès le début du projet et m’a permis de travailler avec une grande autonomie.
Sa disponibilité et la qualité de ses retours ont été essentielles pour structurer mon travail et orienter mes choix techniques.
Nos échanges réguliers ont constitué un appui précieux, tant pour résoudre des problématiques concrètes que pour mieux comprendre les enjeux liés aux données naturalistes et à leur valorisation.

Je remercie également mon tuteur académique, M. Jean-François Favennec, pour son suivi et ses conseils tout au long du stage.
Ses remarques m’ont aidé à prendre du recul sur les travaux réalisés et à adopter une démarche plus rigoureuse et structurée.
Son accompagnement a contribué à inscrire ce projet dans une véritable logique d’ingénierie.

Je souhaite remercier l’ensemble des agents de la DEAL Réunion pour leur accueil et leur bienveillance.
J’ai évolué dans un environnement de travail agréable, où les échanges étaient simples et constructifs.
Les discussions avec les différents agents m’ont permis de mieux appréhender le fonctionnement d’une administration publique et les contraintes associées à la gestion des données environnementales.

Je remercie tout particulièrement les membres de l’Unité Biodiversité (UBIO) pour leur disponibilité et leur intérêt pour le travail mené.
Leurs retours concrets et leur expertise métier ont donné du sens aux développements réalisés.
Ils ont contribué à orienter le projet vers des solutions réellement utiles et adaptées aux besoins du service.

Enfin, je remercie l’ENIB pour la formation d’ingénieur généraliste dispensée, qui m’a permis de mobiliser des compétences variées tout au long de ce stage.
Cette expérience a été particulièrement enrichissante, tant sur le plan technique que personnel.
Elle a renforcé mon intérêt pour le développement d’outils numériques ayant un impact concret, en lien avec les enjeux environnementaux et territoriaux.