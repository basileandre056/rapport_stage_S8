
Les développements réalisés durant ce stage ont permis de mettre en évidence l’intérêt d’adapter 
les stratégies d’intégration des données en fonction de la nature et de l’origine des sources 
mobilisées. Les travaux menés sur \gls{quadrige} et \gls{plantnet} illustrent deux approches 
complémentaires, répondant à des contraintes distinctes.

Pour les données issues de \gls{quadrige}, produites dans un cadre institutionnel et scientifique, 
la priorité a été donnée à la traçabilité et au contrôle des extractions. La séparation entre les 
phases d’extraction, de préparation et d’intégration permet de conserver un contrôle humain sur 
des données pouvant être soumises à des règles de diffusion spécifiques (anonymisation, 
moratoires), tout en préparant leur intégration progressive dans GeoNature.

À l’inverse, les données issues de \gls{plantnet}, bien que participatives, peuvent faire l’objet d’un 
traitement largement automatisé. Le pipeline mis en place intègre des mécanismes de validation 
taxonomique reposant sur les référentiels utilisés par GeoNature, permettant d’identifier, de 
corriger ou de rejeter automatiquement les observations non conformes. Les données validées 
peuvent ainsi être intégrées directement dans la base GeoNature, sans intervention manuelle, 
tout en conservant un historique des décisions prises par le système.

Cette distinction entre intégration contrôlée et intégration automatisée souligne l’importance 
d’une architecture d’import flexible, capable de s’adapter aux caractéristiques des sources de 
données. Elle ouvre la voie à une gestion différenciée des flux, conciliant exigences de qualité 
scientifique, efficacité opérationnelle et montée en charge des systèmes d’information 
naturalistes.
