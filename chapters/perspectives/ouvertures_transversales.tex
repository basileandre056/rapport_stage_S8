
Au-delà des deux modules développés, les travaux réalisés ouvrent des perspectives plus larges 
sur l’architecture des imports dans GeoNature. L’approche retenue, fondée sur des modules 
externes, une forte traçabilité et une séparation claire entre extraction, préparation et import, 
pourrait être généralisée à d’autres sources de données environnementales.

Cette logique modulaire facilite la mutualisation des développements, la reprise des projets par 
différents acteurs et l’adaptation des outils aux spécificités territoriales. Elle s’inscrit 
pleinement dans la stratégie de modernisation du \gls{sinp} et dans les objectifs de fiabilisation 
et d’harmonisation des données naturalistes portés par la DEAL Réunion et ses partenaires.