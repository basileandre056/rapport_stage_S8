
Le module d’import \gls{quadrige} développé durant le stage constitue une première brique 
fonctionnelle permettant d’explorer, d’extraire et de préparer des données marines issues du 
système d’information \gls{quadrige}. Plusieurs axes d’amélioration peuvent être envisagés afin 
d’en accroître la maturité et de faciliter son intégration opérationnelle à long terme.


\subsection{Amélioration de l’ergonomie et de l’intégration visuelle}

Une première évolution concernerait l’intégration visuelle du module au sein de l’interface 
GeoNature. L’ajout d’un pictogramme dédié dans la barre de navigation ou dans la liste des modules 
permettrait d’identifier plus clairement le module \gls{quadrige} parmi les autres extensions 
disponibles.

Ce pictogramme contribuerait à améliorer l’expérience utilisateur en rendant le module plus 
visible et plus intuitif pour les administrateurs amenés à l’utiliser ponctuellement. Il 
faciliterait également son appropriation par de nouveaux utilisateurs, notamment dans un 
contexte de reprise du projet ou de déploiement sur d’autres territoires.


\subsection{Génération directe de fichiers au format GeoNature}

À l’heure actuelle, le module produit des fichiers intermédiaires (CSV bruts, CSV filtrés, 
fichiers JSON de paramètres) destinés à être analysés et, le cas échéant, retravaillés avant 
import dans GeoNature. Cette approche garantit une forte traçabilité, mais implique encore une 
étape manuelle avant l’intégration finale.

Une évolution naturelle du module consisterait à ajouter une étape optionnelle de transformation 
automatique des données extraites vers un format directement compatible avec les mécanismes 
d’import de GeoNature. Cette étape pourrait s’appuyer sur le schéma de synthèse et sur les règles 
existantes des modules d’import afin de produire un fichier immédiatement prêt à être injecté 
dans le système.

Cette fonctionnalité pourrait être activée de manière conditionnelle, laissant le choix à 
l’administrateur entre une extraction préparatoire avec contrôle manuel ou une extraction 
complète incluant la génération d’un fichier directement importable.


\subsection{Centralisation de la configuration côté backend}

Au cours du développement, certaines difficultés ont été identifiées dans la gestion de la 
configuration du module, notamment concernant les localisations proposées lors du filtrage des 
programmes et les champs disponibles pour l’extraction. Une partie de ces paramètres est 
actuellement définie directement dans le frontend Angular, ce qui limite la flexibilité du 
module et complique son évolution.

Une perspective importante consisterait à centraliser l’ensemble de ces paramètres dans un 
fichier de configuration côté backend. Le module expose déjà un endpoint permettant d’accéder 
au contenu de ce fichier, mais son exploitation n’a pas pu être finalisée durant le stage par 
manque de temps.

Une refonte partielle du frontend permettrait ainsi de charger dynamiquement les localisations 
disponibles et les champs d’extraction depuis le backend, d’éviter les valeurs codées en dur dans 
l’interface et de faciliter l’adaptation du module à d’autres territoires sans modification du 
code frontend. Cette approche renforcerait la maintenabilité globale du module et réduirait les 
risques d’incohérence entre backend et frontend.

\subsection{Vers une automatisation partielle des imports}


À plus long terme, le module pourrait évoluer vers une automatisation partielle ou contrôlée des 
imports, par exemple via des imports périodiques planifiés, des imports incrémentaux basés sur 
les dates de mise à jour des données, ou des déclenchements manuels avec validation préalable.

De telles évolutions nécessiteraient toutefois une réflexion approfondie sur la gestion des 
doublons, la validation scientifique des données et la traçabilité des mises à jour, afin de 
rester cohérentes avec les exigences du \gls{sinp} et les pratiques de contrôle des données 
naturalistes.