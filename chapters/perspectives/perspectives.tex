Les travaux réalisés au cours de ce stage ont permis de poser des bases techniques solides pour 
l’intégration de données externes au sein de \gls{geonature}, en particulier pour les données 
issues du milieu marin via \gls{quadrige}, ainsi que pour les données participatives produites 
par \gls{plantnet}.  
Plusieurs perspectives d’évolution ont été identifiées, tant sur le plan fonctionnel que 
technique. Elles visent à renforcer l’ergonomie des outils développés, à améliorer leur 
intégration dans l’écosystème GeoNature et à faciliter leur appropriation par les administrateurs 
et gestionnaires de données.

\section{Perspectives d’évolution du module d’import Quadrige}
Le module d’import {\gls{quadrige}} développé durant le stage constitue une première brique 
fonctionnelle permettant d’explorer, d’extraire et de préparer des données marines issues du 
système d’information \gls{quadrige}. Plusieurs axes d’amélioration peuvent être envisagés afin d’en 
accroître la maturité et de faciliter son intégration opérationnelle à long terme.

\subsection{Amélioration de l’ergonomie et de l’intégration visuelle}
    
Une première évolution concernerait l’intégration visuelle du module au sein de l’interface 
GeoNature. L’ajout d’un pictogramme dédié dans la barre de navigation ou dans la liste des 
modules permettrait d’identifier plus clairement le module \gls{quadrige} parmi les autres extensions 
disponibles.

Ce pictogramme contribuerait à améliorer l’expérience utilisateur en rendant le module plus 
visible et plus intuitif pour les administrateurs amenés à l’utiliser ponctuellement. Il 
faciliterait également son appropriation par de nouveaux utilisateurs, notamment dans un 
contexte de reprise du projet ou de déploiement sur d’autres territoires.

\subsection{Ajout d’une étape de mise en forme et d’intégration des données}

À l’heure actuelle, le module produit des archives de données (fichiers CSV bruts, CSV filtrés, 
fichiers JSON de paramètres) à l’issue de la phase d’extraction depuis l’API \gls{quadrige}. Ces 
archives constituent une étape intermédiaire volontaire, permettant un contrôle manuel des 
résultats avant toute intégration dans GeoNature.

Une évolution naturelle du module consisterait à ajouter une étape supplémentaire, positionnée 
après la génération des fichiers ZIP, dédiée à la mise en forme finale et à l’intégration des 
données dans la base GeoNature. Cette étape pourrait inclure la transformation des données vers 
le schéma de synthèse, ainsi que leur préparation pour un import direct dans la base de données.

Cette évolution permettrait de proposer une chaîne complète allant de l’extraction à 
l’intégration, tout en conservant la possibilité de valider manuellement les données en amont. 
Elle resterait compatible avec les exigences de traçabilité et de contrôle associées au 
\gls{sinp}.

\subsection{Clarification et extension de la gestion de la configuration}

Le module repose déjà sur un fichier de configuration centralisé, chargé côté backend dans 
l’environnement GeoNature. Ce fichier permet notamment de définir les paramètres techniques 
essentiels tels que l’URL de l’API \gls{quadrige} et le jeton d’authentification, qui sont ensuite mis à 
disposition du backend.

Le backend expose également le contenu de ce fichier de configuration via une route dédiée, 
permettant théoriquement au frontend d’accéder dynamiquement à ces informations. Toutefois, 
par manque de temps, certaines données métier — telles que les localisations suggérées lors du 
filtrage des programmes ou la liste des champs disponibles pour l’extraction — restent définies 
directement dans le frontend.

Une évolution souhaitable consisterait à étendre le rôle du fichier de configuration afin d’y 
inclure ces éléments métier, puis à adapter le frontend pour les charger dynamiquement depuis 
le backend. Cette approche permettrait de réduire les valeurs codées en dur dans l’interface, 
d’améliorer la cohérence entre backend et frontend et de faciliter l’adaptation du module à 
d’autres territoires ou contextes d’usage.

\subsection{Vers une automatisation partielle et maîtrisée des imports}

À plus long terme, le module pourrait évoluer vers une automatisation partielle ou contrôlée des 
imports, par exemple via des imports périodiques planifiés, des imports incrémentaux basés sur 
les dates de mise à jour des données, ou des déclenchements manuels avec validation préalable.

Toutefois, certaines spécificités des données \gls{quadrige} imposent de conserver un contrôle humain 
sur les résultats des extractions. Certaines informations peuvent être anonymisées (par exemple 
les données relatives aux observateurs) ou soumises à des moratoires de diffusion. Il est donc 
actuellement nécessaire de vérifier les résultats des extractions avant tout traitement visant à 
leur intégration dans GeoNature.

Ces contraintes devront être pleinement prises en compte dans toute évolution vers une 
automatisation plus poussée, afin de rester cohérent avec les exigences du \gls{sinp} et les 
pratiques de contrôle des données naturalistes.


\subsection{Renforcement de la stratégie de tests et de validation}



Les modules externes GeoNature sont conçus pour s’inscrire dans une logique de maintenance à 
long terme. À ce titre, la documentation officielle recommande la mise en place de tests 
automatisés, notamment des tests unitaires et fonctionnels côté backend (via PyTest) ainsi que 
des tests de bout en bout (\textit{end-to-end}) côté frontend (via Cypress).

Dans le cadre de ce stage, des tests ont été développés et validés sur la version locale du 
module Quadrige, afin de vérifier le bon fonctionnement des principaux composants : appels à 
l’API Quadrige, gestion des erreurs, génération des fichiers d’export et cohérence des routes 
exposées par le backend. De la même manière, des scénarios de tests frontend ont été amorcés 
pour valider les parcours utilisateurs essentiels.

Toutefois, en raison des contraintes de temps et du calendrier de déploiement sur les serveurs 
du SINP-974, ces tests n’ont pas pu être intégrés dans la version du module effectivement 
déployée sur l’environnement serveur. 

Une perspective importante consisterait donc à intégrer ces suites de tests dans la 
version déployée du module, afin de renforcer sa robustesse, de sécuriser les évolutions futures 
et de faciliter sa reprise par d’autres développeurs. La généralisation des tests automatisés 
permettrait également de s’inscrire plus étroitement dans les bonnes pratiques recommandées 
pour le développement et la maintenance des modules externes GeoNature.

Dans l’état actuel, le module reste piloté par une requête HTTP unique côté GeoNature.
Une évolution possible consisterait à découpler totalement le traitement de la requête HTTP,
par exemple via une exécution en tâche de fond (Celery ou job système), afin de rendre le module
pleinement asynchrone côté serveur GeoNature.


\section{Perspectives autour du parser Pl@ntNet et du module api2gn}

Les travaux menés sur l’intégration des données \gls{plantnet} ont permis de valider la faisabilité 
technique d’un pipeline complet, depuis l’interrogation de l’API jusqu’à la production de 
fichiers conformes au standard Darwin Core. Toutefois, ces données étant issues d’une 
application participative, leur exploitation soulève des enjeux spécifiques de validation et de 
qualité scientifique.

\subsection{Ajout d’une interface d’administration dédiée}

Une perspective majeure consisterait à proposer une interface frontend permettant aux 
administrateurs de paramétrer le parser \gls{plantnet} sans recourir à la modification de fichiers de 
configuration. Cette interface pourrait permettre de définir les taxons ciblés, les périmètres 
géographiques, les plages temporelles ainsi que les paramètres de filtrage et de normalisation.

Une telle évolution améliorerait significativement l’ergonomie du dispositif et faciliterait son 
utilisation par des profils non techniques. Elle permettrait également d’harmoniser l’expérience 
utilisateur avec celle du module \gls{quadrige}, en proposant une logique d’interaction cohérente au 
sein de GeoNature.

\subsection{Contraintes liées au module api2gn}

Dans l’état actuel, cette évolution n’est toutefois pas immédiatement réalisable. Le module 
\texttt{api2gn} est un module officiel de GeoNature, dont l’architecture n’a pas été conçue à 
l’origine pour intégrer des interfaces frontend spécifiques à chaque source de données.

Toute extension de ce type nécessiterait soit une évolution structurelle du module 
\texttt{api2gn} lui-même, soit la création d’un module externe complémentaire venant piloter le 
parser \gls{plantnet}. Cette réflexion dépasse le cadre du stage, mais les travaux réalisés constituent 
une base technique solide pour initier ce type d’évolution à l’avenir.

\subsection{Renforcement des mécanismes de validation et d’incrémentalité}

Les données \gls{plantnet} étant saisies par des utilisateurs non experts, la validation des taxons 
constitue un enjeu central. Le parser développé intègre déjà un mécanisme de validation basé 
sur la comparaison des \textit{scientific names} avec les référentiels taxonomiques utilisés par 
GeoNature, ainsi que sur des interrogations complémentaires de services externes lorsque cela 
est nécessaire. Un historique des taxons validés et rejetés est également conservé.

Une perspective d’amélioration consisterait à renforcer ces mécanismes, notamment en affinant 
la gestion des requêtes incrémentales. Actuellement, les extractions sont réalisées par blocs de 
1\,000 occurrences. Il pourrait être pertinent de permettre une reprise plus fine des 
extractions interrompues, afin d’éviter de relancer l’ensemble des requêtes en cas d’arrêt 
partiel du traitement.


\section{Ouvertures transversales}

Les développements réalisés durant ce stage ont permis de mettre en évidence l’intérêt d’adapter 
les stratégies d’intégration des données en fonction de la nature et de l’origine des sources 
mobilisées. Les travaux menés sur \gls{quadrige} et \gls{plantnet} illustrent deux approches 
complémentaires, répondant à des contraintes distinctes.

Pour les données issues de \gls{quadrige}, produites dans un cadre institutionnel et scientifique, 
la priorité a été donnée à la traçabilité et au contrôle des extractions. La séparation entre les 
phases d’extraction, de préparation et d’intégration permet de conserver un contrôle humain sur 
des données pouvant être soumises à des règles de diffusion spécifiques (anonymisation, 
moratoires), tout en préparant leur intégration progressive dans GeoNature.

À l’inverse, les données issues de \gls{plantnet}, bien que participatives, peuvent faire l’objet d’un 
traitement largement automatisé. Le pipeline mis en place intègre des mécanismes de validation 
taxonomique reposant sur les référentiels utilisés par GeoNature, permettant d’identifier, de 
corriger ou de rejeter automatiquement les observations non conformes. Les données validées 
peuvent ainsi être intégrées directement dans la base GeoNature, sans intervention manuelle, 
tout en conservant un historique des décisions prises par le système.

Cette distinction entre intégration contrôlée et intégration automatisée souligne l’importance 
d’une architecture d’import flexible, capable de s’adapter aux caractéristiques des sources de 
données. Elle ouvre la voie à une gestion différenciée des flux, conciliant exigences de qualité 
scientifique, efficacité opérationnelle et montée en charge des systèmes d’information 
naturalistes.
