
Les travaux menés sur l’intégration des données \gls{plantnet} ont permis de valider la faisabilité 
technique d’un pipeline complet, depuis l’interrogation de l’API jusqu’à la production de 
fichiers conformes au standard Darwin Core. Toutefois, ces données étant issues d’une 
application participative, leur exploitation soulève des enjeux spécifiques de validation et de 
qualité scientifique.

\subsection{Ajout d’une interface d’administration dédiée}

Une perspective majeure consisterait à proposer une interface frontend permettant aux 
administrateurs de paramétrer le parser \gls{plantnet} sans recourir à la modification de fichiers de 
configuration. Cette interface pourrait permettre de définir les taxons ciblés, les périmètres 
géographiques, les plages temporelles ainsi que les paramètres de filtrage et de normalisation.

Une telle évolution améliorerait significativement l’ergonomie du dispositif et faciliterait son 
utilisation par des profils non techniques. Elle permettrait également d’harmoniser l’expérience 
utilisateur avec celle du module \gls{quadrige}, en proposant une logique d’interaction cohérente au 
sein de GeoNature.

\subsection{Contraintes liées au module api2gn}

Dans l’état actuel, cette évolution n’est toutefois pas immédiatement réalisable. Le module 
\texttt{api2gn} est un module officiel de GeoNature, dont l’architecture n’a pas été conçue à 
l’origine pour intégrer des interfaces frontend spécifiques à chaque source de données.

Toute extension de ce type nécessiterait soit une évolution structurelle du module 
\texttt{api2gn} lui-même, soit la création d’un module externe complémentaire venant piloter le 
parser \gls{plantnet}. Cette réflexion dépasse le cadre du stage, mais les travaux réalisés constituent 
une base technique solide pour initier ce type d’évolution à l’avenir.

\subsection{Renforcement des mécanismes de validation et d’incrémentalité}

Les données \gls{plantnet} étant saisies par des utilisateurs non experts, la validation des taxons 
constitue un enjeu central. Le parser développé intègre déjà un mécanisme de validation basé 
sur la comparaison des \textit{scientific names} avec les référentiels taxonomiques utilisés par 
GeoNature, ainsi que sur des interrogations complémentaires de services externes lorsque cela 
est nécessaire. Un historique des taxons validés et rejetés est également conservé.

Une perspective d’amélioration consisterait à renforcer ces mécanismes, notamment en affinant 
la gestion des requêtes incrémentales. Actuellement, les extractions sont réalisées par blocs de 
1\,000 occurrences. Il pourrait être pertinent de permettre une reprise plus fine des 
extractions interrompues, afin d’éviter de relancer l’ensemble des requêtes en cas d’arrêt 
partiel du traitement.