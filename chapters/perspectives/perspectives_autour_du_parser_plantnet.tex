



Les travaux menés sur l’intégration des données \gls{plantnet} ont permis de valider la 
faisabilité technique d’un pipeline complet, depuis l’interrogation de l’\gls{api} jusqu’à la 
production de fichiers conformes au standard Darwin Core. Toutefois, le parser développé 
repose actuellement sur des fichiers de configuration manipulés côté backend.


\subsection{Ajout d’une interface d’administration dédiée}

Une perspective majeure consisterait à proposer une interface frontend permettant aux 
administrateurs de paramétrer le parser \gls{plantnet} sans recourir à la modification de 
fichiers de configuration. Cette interface pourrait offrir la possibilité de définir les taxons 
ciblés, les périmètres géographiques (boîtes englobantes ou polygones), les plages temporelles, 
ainsi que les paramètres de normalisation et de filtrage.

Une telle évolution améliorerait significativement l’ergonomie du dispositif et faciliterait 
son utilisation par des profils non techniques. Elle permettrait également d’harmoniser 
l’expérience utilisateur avec celle du module \gls{quadrige}, en proposant une logique 
d’interaction cohérente au sein de GeoNature.


\subsection{Contraintes liées au module api2gn}

Dans l’état actuel, cette évolution n’est toutefois pas immédiatement réalisable. Le module 
\texttt{api2gn} est un module officiel de GeoNature, dont l’architecture n’a pas été conçue à 
l’origine pour intégrer des interfaces frontend spécifiques à chaque source de données.

Toute extension de ce type nécessiterait soit une évolution structurelle du module \texttt{api2gn} 
lui-même, soit la création d’un module externe complémentaire venant piloter le parser 
\gls{plantnet}. Cette réflexion dépasse le cadre du stage, mais les travaux réalisés constituent 
une base technique solide pour initier ce type d’évolution à l’avenir.


\subsection{Vers une intégration renforcée des données participatives}


À plus long terme, l’intégration des données issues de \gls{plantnet} dans GeoNature pourrait 
contribuer à enrichir significativement les données naturalistes disponibles, en particulier 
pour le suivi des espèces exotiques envahissantes végétales.

L’automatisation du pipeline \gls{api} → Darwin Core → GeoNature, couplée à des mécanismes de 
filtrage et de validation adaptés, offrirait un levier intéressant pour exploiter des volumes 
importants de données participatives tout en respectant les exigences de qualité et de 
traçabilité du \gls{sinp}.
