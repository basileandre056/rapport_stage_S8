% --- Glossaire des sigles ---

\newacronym{api}{API}{Interface de Programmation de l'Application}
\newacronym{apirest}{API REST}{Interface de programmation reposant sur le protocole HTTP, utilisant des opérations standard (GET, POST, PUT, DELETE) pour accéder et manipuler des ressources identifiées par des URL}
\newacronym{api2gn}{api2GN}{Module Géonature permettant la création et l'utilisation de parseurs}

\newacronym{borbonica}{Borbonica}{Plateforme régionale de diffusion des données naturalistes du SINP}
\newacronym{cai}{CAI}{Conception d'Applications Interactives}
\newacronym{cdnom}{cd\_nom}{Identifiant taxonomique unique utilisé dans le référentiel TAXREF}
\newacronym{crs}{CRS}{Communication Réseau Système}
\newacronym{csrpn}{CSRPN}{Conseil Scientifique Régional du Patrimoine Naturel}
\newacronym{csv}{CSV}{Comma-Separated Values, format de fichier texte structuré permettant de représenter des données tabulaires}

\newacronym{deal}{DEAL}{Direction de l’Environnement, de l’Aménagement et du Logement}
\newacronym{dwc}{DwC}{Darwin Core, standard international de structuration des données de biodiversité}
\newacronym{eee}{EEE}{Espèces Exotiques Envahissantes}
\newacronym{enib}{ENIB}{École Nationale d’Ingénieurs de Brest}
\newacronym{erc}{ERC}{Éviter – Réduire – Compenser}
\newacronym{etl}{ETL}{Extract, Transform, Load, processus d’extraction, de transformation et de chargement de données}

\newacronym{geojson}{GeoJSON}{Format standard basé sur JSON permettant de représenter des objets géographiques}
\newacronym{geonature}{Géonature}{Application de gestion et de centralisation des données naturalistes}
\newacronym{graphql}{GraphQL}{Langage de requête permettant d’interroger une API en décrivant précisément la structure des données attendues}

\newacronym{ifremer}{Ifremer}{Institut Français de Recherche pour l’Exploitation de la Mer}
\newacronym{json}{JSON}{JavaScript Object Notation, format léger d’échange de données structuré et lisible}

\newacronym{mnhn}{MNHN}{Muséum national d’Histoire naturelle}

\newacronym{ofb}{OFB}{Office Français de la Biodiversité}

\newacronym{plantnet}{Pl@ntNet}{Application mobile de reconnaissance des plantes par photo}
\newacronym{pnrun}{PNRun}{Parc National de La Réunion}
\newacronym{postgis}{PostGIS}{Extension spatiale de PostgreSQL permettant le stockage et le traitement de données géographiques}
\newacronym{postgresql}{PostgreSQL}{Système de gestion de base de données relationnelle open source}

\newacronym{quadrige}{Quadrige}{Système d’information de l’Ifremer dédié aux données environnementales marines}

\newacronym{rne}{RNE}{Responsabilité Numérique des Entreprises}
\newacronym{rse}{RSE}{Responsabilité Sociétale des Entreprises}

\newacronym{seb}{SEB}{Service Eau et Biodiversité}
\newacronym{seor}{SEOR}{Société d’Études Ornithologiques de La Réunion}
\newacronym{sig}{SIG}{Système d’Information Géographique}
\newacronym{sinp}{SINP-974}{Système d’Information de l’iNventaire du Patrimoine naturel de la Réunion}
\newacronym{srepen}{SREPEN}{Société réunionnaise pour l'étude et la protection de la nature}

\newacronym{taxref}{TAXREF}{Référentiel taxonomique national de la faune, de la flore et de la fonge, maintenu par le MNHN}
\newacronym{taxrefld}{TAXREF-LD}{Service web du MNHN exposant le référentiel taxonomique TAXREF sous forme de données liées (Linked Data), permettant l’interrogation et la résolution taxonomique via une API}

\newacronym{ubio}{UBIO}{Unité Biodiversité}
\newacronym{wfs}{WFS}{Web Feature Service, service web normalisé permettant l’accès à des données géographiques vectorielles}
