Plusieurs arbitrages techniques ont été réalisés
tout au long du stage.
Ces choix ont été guidés par les contraintes métier,
les usages existants
et les exigences de fiabilité
associées aux données environnementales.

Un premier choix structurant a consisté
à adopter une approche intermédiaire.
Le processus repose sur une succession d’étapes distinctes :
extraction des données,
contrôle,
puis intégration différée.

Une automatisation complète et immédiate
de l’import en base GeoNature
n’a pas été retenue.
Ce choix permet de conserver
un contrôle humain sur les données extraites,
notamment dans un contexte
où certaines informations
peuvent être soumises
à des règles de diffusion spécifiques.

Dans cette logique,
l’intégration directe des données \gls{quadrige}
dans la base GeoNature
n’a volontairement pas été implémentée.
Le module produit des fichiers intermédiaires documentés,
afin de sécuriser les étapes amont
et de faciliter l’analyse des résultats.

Concernant le filtrage géographique,
un traitement a posteriori
à l’aide de la bibliothèque \textit{pandas}
a été privilégié.
Ce choix s’explique par le fonctionnement même
de l’\gls{api} \gls{quadrige},
qui raisonne à l’échelle des programmes
et non des stations individuelles.

Enfin, la configuration des modules \gls{quadrige}
et \gls{plantnet}
a été entièrement externalisée
dans des fichiers dédiés.
Ce choix permet d’adapter les paramètres d’extraction
sans modifier le code,
de limiter les risques d’erreur
et de faciliter le déploiement
sur d’autres environnements.

Ces arbitrages traduisent une volonté claire
de privilégier la robustesse,
la traçabilité
et la réutilisabilité,
plutôt qu’une automatisation maximale
au détriment du contrôle des données.
