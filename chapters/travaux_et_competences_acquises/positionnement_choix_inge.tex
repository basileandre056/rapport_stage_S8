Le travail réalisé au cours de ce stage ne se limite pas
au développement ponctuel d’un script
ou à l’utilisation isolée d’une \gls{api} externe.

Le module \gls{quadrige} développé s’inscrit pleinement
dans l’écosystème GeoNature.
Il respecte l’architecture logicielle existante,
les conventions de développement
et les mécanismes de sécurité du système.

L’objectif n’était pas uniquement de rendre possible
l’interrogation de l’\gls{api} \gls{quadrige}.
Il s’agissait de concevoir une chaîne d’acquisition
structurée, reproductible et traçable.

Cette chaîne permet d’explorer les données disponibles,
de les extraire selon des critères maîtrisés,
puis de les préparer pour une intégration ultérieure.
Elle laisse volontairement une place
au contrôle humain et à l’analyse métier.

Le module a été conçu comme une extension à part entière
du système d’information.
Il respecte la séparation des responsabilités
entre le frontend et le backend,
ainsi que les principes de configuration centralisée
propres à GeoNature.

Une attention particulière a été portée
à la maintenabilité du code.
Les choix d’architecture,
la structuration des modules
et la documentation associée
ont été pensés pour faciliter
la reprise du projet par d’autres développeurs.

Le travail mené relève ainsi
d’une démarche d’ingénierie logicielle.
Il vise à produire un outil durable,
adaptable à d’autres territoires
et cohérent avec les pratiques
de gestion des données naturalistes
portées par la \gls{deal} et ses partenaires.
