\noindent

Ce stage m’a permis de développer une maîtrise approfondie de l’écosystème technique de GeoNature, tant sur le plan backend 
(Python, Flask, SQLAlchemy, Marshmallow) que frontend (Angular, TypeScript). J’ai acquis une expérience concrète dans l’utilisation 
d’API REST et GraphQL, dans la mise en place de tests unitaires et fonctionnels, ainsi que dans la gestion rigoureuse 
d’environnements soumis à des contraintes fortes.

Ce travail m’a également permis de me familiariser avec les pratiques du développement open source,
notamment via l’utilisation de forks GitHub, la gestion des issues et la rédaction de documentation
technique destinée à des utilisateurs externes.


Au-delà des aspects techniques, cette expérience m’a amené à travailler en autonomie tout en collaborant étroitement 
avec des acteurs variés : agents en charge des données, administrateurs réseau, prestataires et agents du Parc national. 
J’ai pu renforcer mes compétences en organisation de projet, en documentation technique et en adaptation à des environnements 
complexes, autant de dimensions essentielles dans un contexte 
d’ingénierie logicielle appliquée aux enjeux environnementaux.