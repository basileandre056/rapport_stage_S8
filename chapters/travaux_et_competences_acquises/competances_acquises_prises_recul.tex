Ce stage m’a permis d’acquérir et de consolider
un ensemble de compétences techniques,
méthodologiques
et transversales,
directement mobilisables
dans un contexte professionnel d’ingénierie.

Sur le plan technique,
j’ai développé une compréhension approfondie
de l’architecture de GeoNature.
J’ai travaillé à la fois sur le backend
(Python, Flask, SQLAlchemy)
et sur le frontend
(Angular, TypeScript),
en respectant une séparation stricte
des responsabilités.

J’ai acquis une expérience concrète
dans l’interfaçage de systèmes hétérogènes,
à travers l’utilisation d’API REST et GraphQL.
Ce travail m’a permis de mieux comprendre
les enjeux liés à l’authentification,
à la gestion des erreurs
et à la robustesse des échanges réseau.

La manipulation de données volumineuses
et hétérogènes
m’a conduit à mettre en œuvre
des traitements de normalisation,
de filtrage
et de validation.
J’ai notamment travaillé
sur des formats standards
tels que CSV, JSON
et Darwin Core,
en tenant compte
des contraintes imposées
par les référentiels nationaux.

Ce stage m’a également permis
de renforcer mes compétences
en ingénierie logicielle.
La structuration du code,
la gestion de la configuration,
l’écriture de tests
et la documentation
ont été des éléments centraux du travail.
J’ai appris à concevoir un outil
pensé pour être maintenu,
repris
et adapté dans le temps.

Sur le plan méthodologique,
j’ai travaillé en grande autonomie,
tout en échangeant régulièrement
avec mon tuteur et les agents concernés.
J’ai appris à organiser mon travail
dans un contexte de projet réel,
soumis à des contraintes de délais,
d’accès aux environnements
et d’incertitudes techniques.

Le stage m’a également permis
de mieux appréhender
le fonctionnement d’une administration publique.
J’ai découvert un cadre de travail
où les outils numériques
s’inscrivent dans des processus longs,
structurés
et fortement contraints.
Cette expérience m’a aidé
à adapter mes choix techniques
aux réalités organisationnelles
et aux usages métiers.

Au-delà des compétences techniques,
ce stage m’a conduit
à prendre du recul
sur la place du numérique
dans les politiques publiques environnementales.
Les outils développés
ne constituent pas une finalité en soi.
Ils participent à la structuration,
à la fiabilité
et à la diffusion de données
qui alimentent des décisions
ayant un impact direct
sur les territoires et les écosystèmes.

Travailler sur des données naturalistes
implique une responsabilité particulière.
Les choix techniques effectués
influencent la qualité des analyses produites,
la visibilité de certaines thématiques
et la capacité des acteurs
à agir de manière éclairée.

Ce stage m’a ainsi permis
de mieux distinguer
le fait de produire du code
et celui de concevoir un outil réellement utile.
Un outil utile est compréhensible,
maintenable,
adapté à son contexte d’usage
et aligné avec les enjeux
qu’il vise à servir.

Cette expérience correspond pleinement
aux objectifs de la formation
d’ingénieur généraliste dispensée à l’ENIB.
Elle a renforcé ma capacité
à articuler compétences techniques,
compréhension des enjeux sociétaux
et sens donné aux projets réalisés.


Les compétences acquises et les outils développés au cours de ce stage
ouvrent naturellement la voie à plusieurs perspectives d’évolution,
présentées dans le chapitre suivant.
