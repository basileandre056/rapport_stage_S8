
Le déroulement du stage a été marqué
par plusieurs contraintes techniques
ayant influencé l’organisation du travail.

L’absence d’une installation locale stable de GeoNature
a constitué l’une des principales difficultés.
Les incompatibilités entre certaines versions logicielles,
les contraintes liées à Docker
et les erreurs du serveur applicatif
ont rendu nécessaire
la mise en place d’un environnement de travail spécifique.

Cette situation a été renforcée
par les contraintes réseau de la \gls{deal}.
La présence d’un proxy strict
a perturbé l’utilisation
de plusieurs outils de développement.
La mise en place progressive
d’un mécanisme de gestion dynamique du proxy
a permis de stabiliser l’environnement.

La migration des serveurs du \gls{sinp}
vers un nouvel hébergeur
a également limité l’accès
à un environnement de test réaliste.
Pendant plusieurs semaines,
le développement a dû être poursuivi
sans possibilité de déploiement direct.

Une fois l’accès ouvert
via une machine virtuelle bastion,
les premières manipulations
ont été freinées par les restrictions
liées à l’accès distant.
L’utilisation de Git
comme canal principal de transfert
s’est révélée déterminante
pour poursuivre les tests
et ajuster le module.

Le travail sur les données \gls{plantnet}
a également soulevé des difficultés spécifiques.
L’\gls{api} renvoie des structures \gls{json}
très hétérogènes,
variables selon les taxons
et les métadonnées disponibles.
Cela a nécessité la conception
d’un parseur robuste et flexible.

Par ailleurs, le standard Darwin Core
impose un modèle strict,
qui ne correspond pas directement
aux champs fournis par \gls{plantnet}.
Un travail de mapping configurable
a été nécessaire
pour garantir la compatibilité
avec GeoNature et Borbonica.

Enfin, l’intégration du parseur
au module \gls{api2gn}
a mis en évidence certaines limites
dans les migrations SQL existantes.
Des ajustements ont été réalisés
afin de permettre une installation correcte du module.

Ces difficultés ont nécessité
une forte capacité d’adaptation.
Elles ont conduit
à renforcer la documentation,
à multiplier les tests exploratoires
et à clarifier progressivement
les environnements de travail.

Elles ont également contribué
à structurer une démarche rigoureuse,
essentielle dans un contexte
d’ingénierie logicielle appliquée
aux données environnementales.
