Tout au long du stage, ces arbitrages ont été confrontés aux contraintes techniques concrètes du projet.
Son déroulement a ainsi été marqué par plusieurs difficultés ayant influencé son organisation.


L’absence d’une installation locale stable de GeoNature
a constitué l’une des principales difficultés.
Les incompatibilités entre certaines versions logicielles,
les contraintes liées à Docker
et les erreurs du serveur applicatif
ont rendu nécessaire
la mise en place d’un environnement de travail spécifique.

Cette situation a été renforcée
par les contraintes réseau de la \gls{deal}.
La présence d’un proxy strict
a perturbé l’utilisation
de plusieurs outils de développement.
La mise en place progressive
d’un mécanisme de gestion dynamique du proxy
a permis de stabiliser l’environnement.

La migration des serveurs du \gls{sinp}
vers un nouvel hébergeur
a également limité l’accès
à un environnement de test réaliste.
Pendant plusieurs semaines,
le développement a dû être poursuivi
sans possibilité de déploiement direct.

Une fois l’accès ouvert
via une machine virtuelle bastion,
les premières manipulations
ont été freinées par les restrictions
liées à l’accès distant.
L’utilisation de Git
comme canal principal de transfert
s’est révélée déterminante
pour poursuivre les tests
et ajuster le module.


Une difficulté est apparue en fin de stage, lors du déploiement du module Quadrige sur le serveur de la DEAL.

Le volume important de programmes Quadrige sur le périmètre de La Réunion était déjà connu en amont. Les tests réalisés 
en local permettaient d’extraire plusieurs dizaines de programmes sans difficulté particulière, y compris lors de traitements 
longs. Le module avait donc été conçu dès le départ pour gérer un nombre élevé de programmes, supérieur à 80.

Les problèmes sont apparus lors des premiers déploiements en environnement réel. Le serveur GeoNature, basé sur une distribution 
Debian, utilise une chaîne applicative reposant sur Apache et Gunicorn. Dans cette configuration, des timeouts bloquants sont 
appliqués aux requêtes HTTP dépassant une durée d’environ 30 secondes.

Lors du lancement des extractions Quadrige en conditions réelles, certaines requêtes longues dépassaient ce seuil. Cela entraînait 
l’interruption complète du traitement, même lorsque seules quelques extractions restaient en cours. Ces échecs ne reflétaient pas un 
problème fonctionnel de l’API Quadrige, mais une incompatibilité entre la durée des traitements et les contraintes du serveur applicatif.

Pour répondre à cette contrainte, j’ai revu la logique d’orchestration des appels à l’API Quadrige. Les extractions sont désormais 
lancées en batch, en soumettant l’ensemble des programmes sélectionnés lors d’une première phase rapide. Un mécanisme de polling 
global permet ensuite de suivre l’état de chaque extraction de manière non bloquante, à l’aide de requêtes courtes compatibles avec 
les timeouts imposés par Apache et Gunicorn.

Cette approche permet de découpler le lancement des extractions de leur suivi, en évitant une logique strictement séquentielle. 
Les appels lourds à l’API Quadrige sont remplacés par une succession de requêtes courtes de suivi d’état, compatibles avec les 
contraintes de timeout imposées par Apache et Gunicorn.

Bien que le traitement reste piloté par une requête unique côté GeoNature, cette organisation réduit fortement les risques de blocage 
liés aux traitements longs. Elle permet également de maîtriser le temps global d’exécution et d’éviter qu’un programme lent n’entraîne 
l’échec de l’ensemble de l’extraction

Cette solution ne repose pas sur une exécution asynchrone complète côté serveur GeoNature, mais sur une orchestration optimisée des 
appels à l’API Quadrige. Elle permet néanmoins de contourner efficacement les limitations liées aux timeouts applicatifs en production.

Enfin, la récupération des résultats et le téléchargement des fichiers sont réalisés programme par programme. Les erreurs sont gérées 
individuellement, ce qui permet de conserver les extractions valides même en cas d’échec partiel.

Cette évolution a permis de rendre le module compatible avec les contraintes de l’environnement de production. Elle a supprimé les 
timeouts rencontrés lors des premiers déploiements et renforcé la robustesse du module face aux montées en charge, tout en conservant 
une traçabilité fine des opérations réalisées.



Le travail sur les données \gls{plantnet}
a également soulevé des difficultés spécifiques.
L’\gls{api} renvoie des structures \gls{json}
très hétérogènes,
variables selon les taxons
et les métadonnées disponibles.
Cela a nécessité la conception
d’un parseur robuste et flexible.

Par ailleurs, le standard Darwin Core
impose un modèle strict,
qui ne correspond pas directement
aux champs fournis par \gls{plantnet}.
Un travail de mapping configurable
a été nécessaire
pour garantir la compatibilité
avec GeoNature et Borbonica.

Enfin, l’intégration du parseur
au module \gls{api2gn}
a mis en évidence certaines limites
dans les migrations SQL existantes.
Des ajustements ont été réalisés
afin de permettre une installation correcte du module.

Ces difficultés ont nécessité
une forte capacité d’adaptation.
Elles ont conduit
à renforcer la documentation,
à multiplier les tests exploratoires
et à clarifier progressivement
les environnements de travail.

Elles ont également contribué
à structurer une démarche rigoureuse,
essentielle dans un contexte
d’ingénierie logicielle appliquée
aux données environnementales.
