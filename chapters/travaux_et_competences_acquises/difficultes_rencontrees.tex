\noindent
Le déroulement du stage a été marqué par un ensemble de contraintes techniques qui ont largement influencé 
l’organisation du travail. L’absence d’une installation locale stable de GeoNature a constitué l’un des principaux obstacles : 
les incompatibilités entre les versions de Debian, les erreurs liées au serveur applicatif et les difficultés rencontrées avec 
Docker ont rendu nécessaire une approche alternative, centrée sur une instance de travail personnalisée. Cette situation a été 
accentuée par les limitations du réseau de la DEAL, où la présence d’un proxy strict perturbait l’utilisation de nombreux outils 
de développement. La mise en place d’un mécanisme dynamique de gestion du proxy a progressivement permis de stabiliser l’environnement.

La migration des serveurs du SINP-974 vers un nouvel hébergeur a également retardé l’accès à un environnement de test réaliste. 
Pendant plusieurs semaines, le développement a dû être poursuivi sans possibilité de déployer ou de tester directement sur 
l’environnement cible. Une fois l’accès ouvert via une machine virtuelle bastion, les premières manipulations ont été freinées 
par les restrictions imposées par l’outil d’accès à distance. L’usage de Git comme canal de transfert a alors été déterminant 
pour poursuivre les tests et ajuster le module.

Ces difficultés ont nécessité une grande capacité d’adaptation et ont conduit à affiner l’organisation du projet : clarification 
des environnements, adoption d’outils de gestion de version adaptés, multiplication des tests exploratoires et documentation 
progressive pour sécuriser les phases suivantes d’intégration.


\paragraph{Difficultés spécifiques au travail sur Pl@ntNet}

Le développement du client Pl@ntNet a également présenté plusieurs défis techniques. 
L’API renvoie une structure JSON très hétérogène selon les taxons, les médias associés ou les métadonnées disponibles, ce qui a nécessité la création d’un parseur flexible, capable d’absorber des variations importantes de schéma sans échouer.

Par ailleurs, le standard Darwin Core impose un modèle strict qui ne correspond pas directement aux champs fournis par Pl@ntNet. 
Il a donc fallu définir un mapping robuste et configurable pour garantir la compatibilité avec GeoNature et Borbonica.

Enfin, l’intégration du parser dans le module \texttt{api2gn} a nécessité de modifier certaines migrations SQL existantes, qui empêchaient l’installation correcte du module. 
Ces ajustements feront l’objet d’un retour formalisé aux développeurs officiels de GeoNature afin d’améliorer la compatibilité du module avec des sources de données externes diversifiées.
