\noindent

Les travaux réalisés au cours du stage se sont concentrés principalement sur le développement d’un module externe pour 
GeoNature capable d’interagir avec le système d’information Quadrige. La première étape a consisté à concevoir 
l’architecture du backend, fondée sur un blueprint Flask dédié, afin d’assurer l’authentification auprès de l’API,
d’interroger les programmes disponibles et de structurer les données nécessaires à un futur import dans GeoNature.
Une attention particulière a été portée à la construction des requêtes GraphQL, qui permettent de cibler précisément 
les informations utiles tout en limitant le volume des échanges.

Parallèlement, le développement du frontend a permis de proposer une première interface d’exploration des programmes 
Quadrige directement dans GeoNature. Bien qu’encore partielle, cette interface constitue une base solide pour les 
futures fonctionnalités de sélection, de filtrage et d’extraction. Ces travaux ont donné lieu à plusieurs phases de 
tests, notamment via PyTest pour le backend et Cypress pour l’interface utilisateur, afin de valider les mécanismes 
centraux avant leur intégration complète.

En complément, un client Python interfaçant Pl@ntNet a été développé. Il permet d’interroger le service, de traiter 
automatiquement les résultats et de structurer les informations selon un format exploitable. Ce travail préliminaire
prépare une future extension de GeoNature vers de nouvelles sources d’acquisition de données naturalistes.

Enfin, une documentation détaillée a été rédigée tout au long du projet. Elle décrit les procédures d’appel aux API, 
les choix techniques effectués et les étapes nécessaires à l’installation ou à la maintenance du module. Ce travail de 
capitalisation est essentiel pour permettre la reprise du projet par la DEAL ou ses partenaires.
