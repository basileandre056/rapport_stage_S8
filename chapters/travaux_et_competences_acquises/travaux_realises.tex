\noindent

Les travaux réalisés au cours du stage se sont concentrés principalement sur le développement d’un module externe pour 
GeoNature, capable d’interagir avec le système d’information \gls{quadrige}. La première étape a consisté à concevoir 
un client Python permettant de tester l’envoi de requêtes GraphQL auprès de l’\gls{api} \gls{quadrige}, de récupérer les données 
et de gérer l’authentification via un jeton.

Une fois les fonctions d’interrogation terminées, je suis passé au développement d’un prototype local du frontend Angular.
Il se compose de quatre composants principaux. Le premier, \texttt{program-list}, affiche la liste des programmes extraits 
en fonction d'une localisation (par exemple «~126-~» comme préfixe pour La Réunion). C’est depuis ce composant principal 
que s’effectuent les échanges backend/frontend.

Le composant \texttt{program-filter} permet de définir un filtre incluant une localisation et un nom de filtre. 
Les critères sont sauvegardés dans le fichier memory/last\_filter.json.  
Ce filtre sert à extraire les programmes selon des règles bien définies. La requête GraphQL utilisée est 
\texttt{executeProgramExtraction}, qui comporte de nombreux paramètres optionnels, mais dont seuls ceux relatifs à la 
localisation nous intéressent. L’objectif est d’extraire tout programme comportant au moins une instance dans la zone spécifiée.

Un bouton «~Extraire les programmes~» déclenche la fonction d’extraction, qui envoie la requête GraphQL à \gls{quadrige} 
et récupère les programmes correspondant au filtre. Un message d’état informe l’utilisateur de l’avancement 
(«~extraction en cours~», «~programmes extraits~») et, en cas d’erreur, un message approprié s’affiche.

Lorsque l’extraction est terminée, un fichier CSV est produit et stocké dans \texttt{memory/csv\_brut\_suffixe}, 
associé au filtre utilisé. Il contient toutes les instances des programmes ayant au moins une occurrence dans la zone filtrée. 


Le fichier CSV généré à l’issue de cette première extraction contient l’ensemble des instances associées aux programmes sélectionnés, 
y compris celles situées en dehors du périmètre géographique d’intérêt. Cette situation résulte du fonctionnement même de l’\gls{api} 
\gls{quadrige}, qui raisonne à l’échelle des programmes et non des stations individuelles.

Afin de garantir la cohérence géographique des données exploitées, un filtrage a posteriori a été mis en œuvre à l’aide de la 
bibliothèque pandas. Ce traitement consiste à ne conserver que les instances dont le code de monitoring location correspond au 
territoire ciblé, identifié par un préfixe spécifique (par exemple « 126- » pour La Réunion).

Ce filtrage permet de produire un second fichier CSV, strictement limité aux stations pertinentes, utilisé à la fois pour 
l’affichage dans l’interface utilisateur et comme base pour les extractions de données ultérieures.

Par la suite, je n’ai besoin que du code du programme pour extraire les données recherchées.

Une fois les programmes extraits, ils sont affichés dans \texttt{program-list} avec leur nom, leurs dates de début et de fin 
et le(s) responsable(s). L’utilisateur peut sélectionner les programmes via une barre de recherche, des cases cochables et 
une option «~tout cocher~».

Un composant \texttt{data-filter} permet ensuite de filtrer les données des programmes sélectionnés selon plusieurs critères, 
notamment les dates ou la localisation (automatiquement reprise depuis le dernier filtre créé).

---

En parallèle, j’ai conçu l’architecture du backend, fondée sur un \textit{blueprint} Flask dédié, afin d’assurer 
l’authentification auprès de l’\gls{api} \gls{quadrige}, l’interrogation des programmes disponibles et la structuration des données 
nécessaires à un futur import dans GeoNature. Une attention particulière a été portée à la construction des requêtes GraphQL, 
afin de cibler les informations utiles tout en limitant le volume des échanges.

Le développement du frontend a permis de mettre en place une première interface d’exploration des programmes \gls{quadrige} 
directement dans GeoNature. Bien qu’encore partielle, cette interface constitue une base solide pour les futures 
fonctionnalités de sélection, de filtrage et d’extraction. Ces travaux ont donné lieu à plusieurs phases de tests, 
notamment via PyTest pour le backend et Cypress pour l’interface utilisateur, afin de valider les mécanismes centraux 
avant leur intégration complète.

\paragraph{Travaux parallèles sur l’intégration de Pl@ntNet}

En parallèle du développement principal consacré à \gls{quadrige}, j’ai également initié un travail structurant autour de l’intégration des données \gls{plantnet} dans GeoNature. 
L’objectif était double~: d'une part valider la possibilité technique d’interroger l’\gls{api} Pl@ntNet depuis un service Python, et d'autre part poser les bases d’un parseur Darwin Core utilisable par un module GeoNature, selon une logique similaire à celle mise en place pour \gls{quadrige}.

Pour cela, j’ai développé un client Python générique capable de~:
\begin{itemize}
    \item envoyer des requêtes POST vers l’endpoint \texttt{dwc/occurrence/search} de l’\gls{api} Pl@ntNet~;
    \item gérer dynamiquement les paramètres taxonomiques, spatiaux (bbox ou polygone GeoJSON) et temporels~;
    \item enregistrer les réponses brutes au format JSON~;
    \item transformer ces réponses en un CSV conforme au standard Darwin Core (DwC), en s’appuyant sur des fichiers de configuration modulaires.
\end{itemize}

Une attention particulière a été portée à la normalisation des données~:
\begin{itemize}
    \item homogénéisation des dates au format ISO-8601~;
    \item nettoyage et standardisation des coordonnées ainsi que du datum géodésique~;
    \item génération d’identifiants stables (\texttt{occurrenceID})~;
    \item mappage des vocabulaires Pl@ntNet vers ceux utilisés dans GBIF (\texttt{basisOfRecord})~;
    \item sérialisation propre des champs complexes (listes, objets JSON).
\end{itemize}

Ce travail a conduit à la mise en place d’un pipeline complet :

\[
\text{\gls{api}} \;\rightarrow\; \text{JSON brut} \;\rightarrow\; \text{CSV Darwin Core normalisé}
\]

entièrement automatisé et adaptable à d’autres territoires ou taxons via un simple fichier de configuration.

Ce client constitue aujourd’hui la base technique du parser Pl@ntNet destiné au module \texttt{\gls{api}2gn}, dont l’intégration a nécessité plusieurs ajustements dans les migrations SQL et dans la structure interne du module.


Enfin, une documentation détaillée a été rédigée tout au long du projet. Elle décrit les procédures d’appel aux \gls{api}, 
les choix techniques effectués ainsi que les étapes d’installation et de maintenance du module. Ce travail de 
c\gls{api}talisation est essentiel pour permettre la reprise du projet par la DEAL ou ses partenaires.
