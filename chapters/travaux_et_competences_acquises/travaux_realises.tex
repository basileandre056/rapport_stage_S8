
Les travaux réalisés au cours du stage se sont articulés autour de deux axes
principaux.

Le premier axe concerne le développement d’un module externe pour GeoNature,
dédié à l’exploration et à l’extraction de données marines issues du système
d’information \gls{quadrige}.

Le second axe porte sur un travail complémentaire autour de l’intégration
de données participatives produites par l’application \gls{plantnet}.
L’objectif était de poser les bases techniques d’un futur parseur,
adapté à une intégration automatisée dans GeoNature.

L’ensemble des développements a été mené dans une logique d’intégration
au système d’information existant.
Les choix effectués prennent en compte les contraintes techniques,
organisationnelles et métier propres au \gls{sinp}.

\subsection{Vue d’ensemble des développements}

Les développements réalisés comprennent plusieurs volets complémentaires.

Ils incluent la conception d’un module GeoNature externe,
interfacé avec l’\gls{api} de \gls{quadrige},
permettant d’explorer les programmes disponibles
et d’en extraire les données associées.

Ils comprennent également la mise en place d’une interface frontend
intégrée à GeoNature.
Cette interface permet de rechercher, filtrer et sélectionner
des programmes et des données à extraire.

Les extractions produisent des fichiers intermédiaires documentés,
destinés à une intégration différée dans GeoNature.
Cette approche permet de conserver un contrôle sur les données produites
avant toute insertion en base.

Enfin, une documentation technique et utilisateur a été rédigée
tout au long du projet.
Elle décrit les principes de fonctionnement du module,
les procédures d’installation,
ainsi que les modalités d’utilisation et de maintenance.

L’ensemble de ces travaux a été conçu pour produire un outil fonctionnel,
maintenable et réutilisable.
Le module est destiné à être repris ou étendu par les équipes de la
\gls{deal} ou par d’autres structures partenaires.

\subsection{Travaux principaux : module \gls{quadrige}}

Les travaux principaux du stage ont porté sur le développement
du module \gls{quadrige} pour GeoNature.

Ce module permet d’interroger l’\gls{api} GraphQL mise à disposition par
l’Ifremer afin d’identifier les programmes pertinents
pour un périmètre géographique donné.
Il offre ensuite la possibilité d’extraire les données associées
selon des critères définis par l’utilisateur.

Le module repose sur une architecture découplée,
séparant clairement le backend et le frontend.
Le backend est responsable des échanges avec l’\gls{api},
de l’authentification,
du suivi des extractions
et de la gestion des fichiers produits.
Le frontend se limite au pilotage des opérations
et à la visualisation des résultats.

Les extractions réalisées produisent des fichiers intermédiaires,
notamment des fichiers \gls{csv} et des archives compressées.
Ces fichiers sont accompagnés de métadonnées décrivant précisément
les paramètres d’extraction utilisés.
Cette organisation garantit la traçabilité des opérations
et facilite l’analyse des résultats.

Le module a été conçu comme une brique préparatoire.
Il ne réalise pas directement l’import des données en base GeoNature,
mais vise à sécuriser les étapes amont
et à faciliter une intégration future
dans le respect des règles du \gls{sinp}.

\subsection{Travaux complémentaires : \gls{plantnet}}

En parallèle du développement du module \gls{quadrige},
un travail complémentaire a été mené autour de l’intégration
des données issues de \gls{plantnet}.

L’objectif de ce travail était d’évaluer la faisabilité technique
d’une intégration automatisée de données participatives
dans GeoNature.
Il s’agissait également de poser les bases d’un futur parseur
utilisable par le module \gls{api2gn}.

Un client Python générique a été développé
pour interroger l’\gls{api} \gls{plantnet},
récupérer les observations
et enregistrer les réponses brutes au format \gls{json}.

Ces données ont ensuite été transformées
en fichiers \gls{csv} conformes au standard Darwin Core.
Une attention particulière a été portée à la normalisation des champs,
à la gestion des dates,
aux coordonnées géographiques
et à la génération d’identifiants stables.

Enfin, un pipeline automatisé, intégrant des mécanismes de validation et de contrôle
reposant sur une configuration externalisée.
Ce pipeline constitue aujourd’hui la base technique
du parseur \gls{plantnet} intégré au module \gls{api2gn}.

Bien que complémentaire au module \gls{quadrige},
ce travail s’inscrit dans la même logique globale.
Il vise à démontrer la capacité de GeoNature
à intégrer des sources de données hétérogènes,
tout en respectant les exigences de qualité,
de traçabilité et de cohérence du \gls{sinp}.
