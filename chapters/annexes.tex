\begin{figure}[H]
    \centering
    \includegraphics[
        width=0.8\textwidth,
        height=0.8\textheight,
        keepaspectratio
    ]{images/organigramme_ubio.pdf}
    \caption{Organigramme de l'Unité Biodiversité (UBIO) – SEB / \gls{deal} Réunion}
    \label{fig:organigramme-ubio}
\end{figure}






\begin{figure}[H]
    \centering
    \includegraphics[width=0.85\textwidth]{images/quadrige/data_filter.png}
    \caption{Interface de filtrage des données extraites pour les programmes Quadrige}
    \label{fig:data-filter}
\end{figure}



\textbf{Documentation du module Pl@ntNet}

La documentation associée au module Pl@ntNet a été produite sous forme de fichiers Markdown
afin de faciliter sa maintenance et sa réutilisation.

Elle comprend :
\begin{itemize}
    \item une documentation technique décrivant l’architecture du client Python et du parser api2gn 
    \item une documentation utilisateur détaillant les paramètres configurables et les cas d’usage 
    \item un fichier \texttt{README.md} présentant le projet, son installation et ses dépendances
\end{itemize}

L’ensemble de cette documentation est disponible dans le dépôt GitHub, dans le dossier documentation :
\href{https://github.com/basileandre056/api2GN/tree/main/documentation}{documentation}

