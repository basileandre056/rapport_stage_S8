\begin{figure}[H]
    \centering
    \includegraphics[width=0.9\textwidth]{images/organigramme_ubio.pdf}
    \caption{Organigramme de l'Unité Biodiversité (UBIO) – SEB / \gls{deal} Réunion}
    \label{fig:organigramme-ubio}
\end{figure}


\begin{figure}[h!]
    \centering
    \includegraphics[width=0.85\textwidth]{images/back_front_geonature.png}
    \caption{Architecture générale du backend et du frontend de GeoNature}
    \label{fig:backfront}
\end{figure}



\section{Liens utiles}

Pour accompagner le développement, plusieurs ressources officielles ont été mobilisées :

\textbf{Quadrige}
\begin{itemize}

    \item \href{https://quadrige-core.ifremer.fr/api/extraction/doc?doc=standard&lang=fr&name=result&type=standard}{Documentation API}
    \item \href{https://quadrige.ifremer.fr/support/Mes-donnees/J-extrais-mes-donnees/J-interroge-l-API-pour-extraire-mes-donnees/Je-regarde-des-videos-de-demo-sur-l-API}{Tutoriels vidéo}  
    \item \href{https://quadrige-app.ifremer.fr/}{Accès à l’application}

\end{itemize}

\textbf{GeoNature}
\begin{itemize}
    \item \href{https://github.com/PnX-SI/GeoNature}{Dépôt GitHub}  
    \item \href{https://docs.geonature.fr/}{Documentation générale}  
    \item \href{https://docs.geonature.fr/development.html\#developper-un-module-externe}{Guide de développement des modules externes}  
\end{itemize}

\bigskip

\noindent
L’ensemble de ces éléments a permis de définir précisément le périmètre et les modalités du
développement. Le chapitre suivant présente la réalisation concrète du module, depuis
l’organisation du travail jusqu’aux phases de tests et de validation des fonctionnalités mises en
place.
