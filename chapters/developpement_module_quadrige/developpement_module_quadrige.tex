\section{Contexte du projet}

Le \gls{sinp}, déployé à La Réunion au sein de l’application Borbonica, comporte encore très peu de
données issues du milieu marin : selon le bilan 2024, moins de 10~\% des observations concernent
des taxons marins. Cette sous-représentation constitue un frein à la connaissance de la
biodiversité littorale et sous-marine de l’île.  

À l’inverse, le système d’information \gls{quadrige}, maintenu par l’Ifremer, rassemble un volume
considérable de données environnementales et biologiques collectées en mer. La mise en place,
fin 2024, d’une nouvelle API GraphQL rend désormais possible l’interrogation directe et
structurée de ces données, ouvrant la voie à une intégration plus large dans les outils du \gls{sinp}.

Dans ce contexte, le projet a consisté à développer un \textbf{module externe GeoNature} capable
d’interroger l'API de Quadrige, de récupérer les programmes pertinents pour La Réunion, d’en
extraire les observations utiles, puis de produire un fichier d’import conforme aux spécifications
du modèle \gls{sinp}. Le module devait s’intégrer naturellement à l’interface existante de GeoNature,
offrir une exploration intuitive des programmes Quadrige, et préparer les données en vue de leur
intégration finale dans Borbonica.  
Ce développement s’inscrit ainsi dans la dynamique d’amélioration continue du \gls{sinp} Réunion et
vise à renforcer la représentativité des données marines dans la base régionale.

\section{Périmètre fonctionnel}

Le module d’import devait couvrir l’ensemble de la chaîne d’acquisition : de la découverte des
programmes Quadrige jusqu’à la production d’un fichier structuré pour GeoNature.  
La première étape consistait à interroger l’API en mode authentifié afin d’obtenir la liste des
programmes disponibles pour un utilisateur donné. Un filtrage automatique sur un périmètre
géographique — principalement La Réunion, mais extensible à d’autres territoires comme les Îles
Éparses — permettait d’isoler les programmes pertinents. Une interface dédiée intégrée à
GeoNature offrait ensuite la possibilité de rechercher des programmes, d’affiner l’affichage par
mots-clés et de sélectionner ceux à importer.

Une fois les programmes choisis, l’utilisateur pouvait définir les filtres à appliquer aux données
(la période d'intérêt, les champs souhaités, ou encore la reprise des stations déjà extraites).  
Le module interrogeait alors l’API Quadrige pour récupérer les observations correspondantes.
Seuls les champs utiles au modèle \gls{sinp} étaient extraits : identifiants des programmes et stations,
localisation géographique, taxon observé, date, ainsi que les métadonnées essentielles (auteur,
organisme, méthode d’acquisition, etc.). Une transformation était appliquée pour obtenir une
structure compatible avec les mécanismes d’import de GeoNature.

Enfin, le module produisait un fichier CSV intermédiaire, destiné à être importé via
l’infrastructure existante de GeoNature. Chaque opération d’import était consignée dans un
historique affiché dans un second onglet, permettant de suivre les actions réalisées, leur date,
leur statut et les éventuelles erreurs rencontrées. L’ensemble du module était réservé aux
administrateurs, conformément aux pratiques habituelles de contrôle des imports dans GeoNature.

\section{Spécifications techniques}

Techniquement, le module repose sur une architecture conforme à celle de GeoNature. Le
\textbf{backend}, développé en Python avec le framework Flask, se charge de communiquer avec
l’API Quadrige, d’effectuer les transformations nécessaires sur les données et de générer les
fichiers d’export. Le \textbf{frontend}, construit en Angular et Bootstrap, s’intègre à l’interface
existante de GeoNature et fournit les pages de sélection des programmes ainsi que l’historique
des imports.

La communication avec Quadrige s’effectue via des requêtes GraphQL, qui permettent d’obtenir
uniquement les champs nécessaires, réduisant ainsi le volume de données transférées.  
L’authentification repose sur un token fourni par l’Ifremer et configurable directement dans les
paramètres du module. Les autres réglages comprennent l’URL de l’API, le périmètre
géographique par défaut, ainsi que le mapping entre les champs Quadrige et les champs
attendus par le schéma de synthèse de GeoNature.

Un soin particulier a été apporté à la gestion des erreurs : vérification des réponses HTTP,
détection des timeouts, contrôle de la validité des données, et affichage de messages explicites à
l’utilisateur. Toutes les actions — appels API, transformations, exports — sont consignées dans les
logs du module, conformément aux pratiques de GeoNature.

\section{Contraintes, dépendances et livrables}

La réussite du projet dépendait principalement de l’accessibilité de l’API Quadrige, de la
stabilité de ses services, et de la disponibilité d’une documentation actualisée. L’intégration dans
GeoNature nécessitait également de respecter la structure modulaire du noyau applicatif et les
contraintes du modèle \gls{sinp}.

Les livrables attendus comprenaient le code source du module, un fichier de configuration, une
documentation à destination des administrateurs (installation, configuration, maintenance) ainsi
qu’un guide d’utilisation orienté métier.

Plusieurs évolutions ont été envisagées : automatisation des imports périodiques, gestion des
imports incrémentaux, intégration à la gestion des droits de GeoNature et, à plus long terme,
publication du module dans le catalogue officiel des extensions GeoNature.

\section{Liens utiles}

Pour accompagner le développement, plusieurs ressources officielles ont été mobilisées :

\textbf{Quadrige}
\begin{itemize}

    \item \href{https://quadrige-core.ifremer.fr/api/extraction/doc?doc=standard&lang=fr&name=result&type=standard}{Documentation API}
    \item \href{https://quadrige.ifremer.fr/support/Mes-donnees/J-extrais-mes-donnees/J-interroge-l-API-pour-extraire-mes-donnees/Je-regarde-des-videos-de-demo-sur-l-API}{Tutoriels vidéo}  
    \item \href{https://quadrige-app.ifremer.fr/}{Accès à l’application}

\end{itemize}

\textbf{GeoNature}
\begin{itemize}
    \item \href{https://github.com/PnX-SI/GeoNature}{Dépôt GitHub}  
    \item \href{https://docs.geonature.fr/}{Documentation générale}  
    \item \href{https://docs.geonature.fr/development.html\#developper-un-module-externe}{Guide de développement des modules externes}  
\end{itemize}

\bigskip

\noindent
L’ensemble de ces éléments a permis de définir précisément le périmètre et les modalités du
développement. Le chapitre suivant présente la réalisation concrète du module, depuis
l’organisation du travail jusqu’aux phases de tests et de validation des fonctionnalités mises en
place.
