Ce stage s’est inscrit dans un contexte à la fois technique,
institutionnel
et environnemental exigeant.
Il avait pour objectif de contribuer
au renforcement des outils numériques
utilisés par la DEAL Réunion
pour la gestion et la valorisation
des données naturalistes.

Les travaux réalisés ont permis
de développer une première brique fonctionnelle
dédiée à l’exploration et à l’extraction
des données marines issues du système d’information Quadrige.
Le module conçu s’intègre à l’architecture de GeoNature,
respecte ses principes de sécurité et de modularité,
et propose une chaîne d’acquisition
structurée, traçable et reproductible.
Il constitue une étape préparatoire essentielle
à l’enrichissement progressif du SINP-974
en données issues du milieu marin.

En parallèle, le travail mené autour de Pl@ntNet
a permis de valider la faisabilité technique
d’une intégration automatisée
de données participatives dans GeoNature.
Le parser développé,
adossé au module api2GN,
met en œuvre des mécanismes robustes
de configuration, de validation taxonomique
et de traçabilité.
Il illustre une approche complémentaire
à celle retenue pour Quadrige,
fondée sur une automatisation maîtrisée
et adaptée à la nature des données traitées.

Au-delà des développements réalisés,
ce stage a mis en évidence
l’importance des choix d’ingénierie
dans un système d’information environnemental.
La conception des outils,
le degré d’automatisation retenu,
la gestion des erreurs
et la traçabilité des traitements
ont un impact direct
sur la qualité des données produites
et sur les usages qui en découlent.
Ces choix doivent donc être pensés
en lien étroit avec les contraintes réglementaires,
les pratiques métier
et les enjeux de diffusion de l’information.

Cette expérience m'a également permis
de mieux appréhender
le fonctionnement d’une administration publique,
où les projets numériques
s’inscrivent dans des temporalités longues,
des environnements contraints
et des logiques de coopération entre acteurs.
Elle a renforcé ma capacité
à adapter mes solutions techniques
à un cadre organisationnel réel,
tout en conservant une exigence
de rigueur et de qualité.

Enfin, ce stage a contribué
à donner du sens à ma formation d’ingénieur généraliste.
Il m’a permis de mobiliser
des compétences en informatique et en traitement de données
au service d’enjeux environnementaux concrets.
Il a confirmé mon intérêt
pour le développement d’outils numériques
utiles, durables
et porteurs d’impact,
en particulier dans les domaines
liés à la gestion des données environnementales
et à la préservation des écosystèmes.

Les perspectives identifiées à l’issue de ce travail
ouvrent la voie à des évolutions techniques
et fonctionnelles ambitieuses,
tant pour l’intégration des données marines
que pour l’exploitation de sources participatives.
Elles constituent autant d’opportunités
pour poursuivre l’amélioration
des systèmes d’information naturalistes
et renforcer leur rôle
dans l’aide à la décision publique.
