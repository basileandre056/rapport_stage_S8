\noindent
As part of my general engineering studies at \gls{enib}, I had the opportunity to explore
a wide range of fields, including computer science, 
 
data processing and network communications. These courses, 
combined with the current environmental context, strengthened my interest in understanding 
how digital tools can address real-world challenges — particularly those related to the environment 
and the marine domain, which has long inspired me. This motivation led me to choose \gls{deal}, where 
the development of digital tools plays a key role in enhancing the value and accessibility of 
environmental data, offering an ideal opportunity to align my technical skills with a topic that 
holds meaning for me.

\par\medskip

I completed my internship at \gls{deal} Réunion, more specifically within the \gls{seb} and the \gls{ubio}. 
Réunion Island faces major biodiversity challenges due to its high proportion of endemic 
species and the presence of particularly sensitive natural habitats. These ecosystems are under strong pressure, 
notably from \gls{eee} and ongoing land-use development, which reinforces the need for rigorous naturalist data.
\gls{deal} uses \gls{geonature} to centralise and manage these datasets.
The main objective of my internship was to develop an external module for this application, enabling the automated 
import of marine and coastal environmental data produced by \gls{ifremer}. These data are essential for biodiversity
monitoring and are accessed through \gls{quadrige},Ifremer’s dedicated information system. 

\par\medskip

Since I had not taken the \gls{cai} module, this internship allowed me to acquire complementary 
skills in software and application development through the design of an import module, data 
manipulation and the interfacing of heterogeneous systems, with the aim of producing a simple, robust 
and maintainable tool.

\par\medskip

It also offered me insight into the functioning of a medium-sized public administration, where the alternation 
between autonomous work and collaboration with staff in charge of naturalist data and the administration of \gls{geonature} 
proved particularly enriching. This experience in a real professional environment strengthened my understanding of the
challenges associated with environmental data management and refined my ability to reflect on the relevance, usefulness 
and purpose of the projects I contribute to, ensuring that they align with my values and have a positive impact.