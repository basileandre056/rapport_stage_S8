\section{Principe général de GeoNature}

GeoNature est une application web dédiée à la gestion, la centralisation et la valorisation des
données naturalistes. Développé initialement en 2010 puis entièrement refondu en 2017, le
projet est aujourd’hui maintenu par le Parc national des Écrins. L’application repose sur une
architecture moderne combinant un backend Python/Flask et un frontend Angular, ce qui lui
permet d’assurer à la fois la saisie, la consultation, la validation et la restitution des données.

\begin{figure}[h!]
    \centering
    \includegraphics[width=0.85\textwidth]{images/back_front_geonature.png}
    \caption{Architecture générale du backend et du frontend de GeoNature}
    \label{fig:backfront}
\end{figure}

Le fonctionnement de GeoNature est modulaire : un noyau applicatif fournit les briques 
communes (API, référentiels, schéma de synthèse), et différents modules viennent étendre les 
fonctionnalités selon les besoins (Occtax, Occhab, Validation, Import, Export, etc.).  
La documentation officielle détaille l’ensemble de cette architecture
\href{https://docs.geonature.fr/}{(documentation officielle)}
ainsi que les sources du projet, disponibles sur 
\href{https://github.com/PnX-SI/GeoNature?tab=readme-ov-file}{GitHub}.


\section{Architecture technique}

GeoNature combine une partie serveur (backend) et une partie cliente (frontend) qui dialoguent
via une API REST.

Le backend est développé en Python à l’aide du framework Flask. Il assure l’ensemble des
traitements métiers, l’accès aux données et la gestion des opérations spatiales via PostgreSQL
et PostGIS. La gestion des utilisateurs et de leurs permissions repose sur UsersHub, qui fournit
l’authentification et le contrôle des accès. L’API exposée par le backend constitue le point
central de communication avec l’interface web.

Le frontend, développé en Angular, constitue l’interface visible par l’utilisateur. Il interroge
exclusivement l’API du backend pour afficher les formulaires, les cartes, les graphiques et les
données attributaires. Des bibliothèques comme Leaflet ou Bootstrap renforcent les
fonctionnalités cartographiques et l’ergonomie générale.  
Cette séparation nette entre backend et frontend garantit la stabilité et la modularité du
système, tout en facilitant les évolutions futures.


\section{Organisation interne et modules}

L’application est structurée autour de modules fonctionnels s’appuyant tous sur les mêmes
référentiels : taxonomie (TaxHub), nomenclatures, utilisateurs (UsersHub) et schéma de
synthèse. Les modules principaux — Occtax pour les observations, Occhab pour les habitats,
ou encore la Validation — s’intègrent directement au cœur applicatif.

Chaque module possède son propre schéma de base de données, son API et ses composants
Angular. Cette organisation modulaire permet de faire évoluer GeoNature, d'ajouter de
nouveaux protocoles d’acquisition ou de développer des extensions externes.  
La documentation décrit précisément les bonnes pratiques et l’architecture à respecter pour
développer un \href{https://docs.geonature.fr/development.html\#developper-un-module-externe}{module GeoNature externe}.
