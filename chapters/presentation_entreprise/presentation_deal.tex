\noindent
La Direction de l’Environnement, de l’Aménagement et du Logement (DEAL) est le service déconcentré de 
l’État chargé de mettre en œuvre, à l’échelle régionale, les politiques publiques relevant du Ministère de la 
Transition Écologique et de la Cohésion des Territoires, ainsi que du Ministère de la Transition Énergétique. 
À La Réunion, territoire insulaire soumis à de fortes pressions environnementales et à des enjeux d’aménagement complexes, 
la DEAL occupe une place centrale au croisement des questions d’environnement, de biodiversité, d’eau, d’urbanisme et 
de développement territorial.
\par\medskip

La DEAL Réunion assure notamment l’application des réglementations environnementales, l’instruction des projets 
d’aménagement, la gestion des risques naturels, le suivi de la ressource en eau ou encore la mise en œuvre des 
politiques de protection des milieux naturels. Elle travaille en étroite collaboration avec les collectivités, 
les établissements publics (OFB, Parc national de La Réunion, Ifremer…), les opérateurs de l’État, ainsi qu’un 
ensemble d’associations naturalistes locales, telles que la SEOR (Société d’Études Ornithologiques de La Réunion), 
qui contribuent activement au suivi et à la préservation de la biodiversité insulaire.
\par\medskip

Au sein de cette structure, le Service Eau et Biodiversité (SEB) porte les missions dédiées à la préservation 
des milieux aquatiques et terrestres, à la connaissance et à la protection des espèces, ainsi qu’à la régulation 
des activités susceptibles d’impacter la biodiversité. Le SEB se situe au cœur des enjeux écologiques de l’île, 
notamment du fait de la présence d’une biodiversité exceptionnellement riche, en grande partie endémique, mais 
aussi particulièrement vulnérable.
\par\medskip


Mon stage s’est déroulé au sein de l’Unité Biodiversité (UBIO), unité en charge du suivi des espèces et des habitats 
naturels, de la gestion et de la valorisation des données naturalistes, ainsi que de l’instruction des dossiers 
réglementaires liés à la biodiversité. L’unité assure également l’animation du Système d’Information sur 
la Nature et les Paysages (SINP) régional, la gestion de la plateforme de données Borbonica, et intervient sur des 
thématiques transversales telles que les espèces exotiques envahissantes, la séquence ERC (Éviter – Réduire – Compenser) 
ou encore la diffusion des connaissances naturalistes.
L’organisation interne de l’UBIO est présentée en annexe (Fig.~\ref{fig:organigramme-ubio}).

\par\medskip


L’organigramme interne montre une équipe pluridisciplinaire rassemblant des profils scientifiques, techniques 
et administratifs, travaillant de manière complémentaire pour répondre aux enjeux biodiversité du territoire. 
Mon stage s’inscrit directement dans cette dynamique, au sein du pôle dédié aux données naturalistes, en soutien 
à la structuration et à la modernisation des outils numériques utilisés par la DEAL.
\par\medskip

La compréhension du rôle institutionnel de la DEAL et de son organisation interne
constitue un préalable essentiel pour appréhender les enjeux liés à la conduite de projets,
qu’ils soient techniques, environnementaux ou organisationnels. Dans cette continuité,
le chapitre suivant propose un \textbf{diagnostic RSE} (Responsabilité Sociétale des Entreprises)
appliqué à la DEAL Réunion. Celui-ci permet d’analyser la manière dont la structure
intègre les dimensions sociale, environnementale et économique dans ses missions,
et d’évaluer la cohérence de ses pratiques avec les principes du développement durable.
\par\medskip

Cette présentation de la DEAL et de son organisation permet de situer le contexte de mon stage.  
Le chapitre suivant propose un \textbf{diagnostic RSE} de la structure, afin d’évaluer ses 
pratiques au regard des enjeux sociaux, environnementaux et organisationnels.
\par\medskip


