\noindent
La Direction de l’Environnement, de l’Aménagement et du Logement (\gls{deal}) 
est le service déconcentré de l’État chargé de mettre en œuvre, à l’échelle régionale, 
les politiques publiques relevant du Ministère de la Transition Écologique et de la Cohésion 
des Territoires, ainsi que du Ministère de la Transition Énergétique. À La Réunion, territoire 
insulaire soumis à de fortes pressions environnementales et à des enjeux d’aménagement complexes, 
la \gls{deal} occupe une place centrale au croisement des questions d’environnement, de biodiversité, 
d’eau, d’urbanisme et de développement territorial.
\par\medskip

La \gls{deal} Réunion assure notamment l’application des réglementations environnementales, 
l’instruction des projets d’aménagement, la gestion des risques naturels, le suivi de la ressource 
en eau ou encore la mise en œuvre des politiques de protection des milieux naturels. Elle travaille 
en étroite collaboration avec les collectivités, les établissements publics (\gls{ofb}, Parc national 
de La Réunion, \gls{ifremer}…), les opérateurs de l’État, ainsi qu’un ensemble d’associations naturalistes 
locales, telles que la \gls{seor}, qui contribuent activement au suivi et à la préservation de la biodiversité insulaire.
\par\medskip

Au sein de cette structure, le Service Eau et Biodiversité (\gls{seb}) porte les missions dédiées 
à la préservation des milieux aquatiques et terrestres, à la connaissance et à la protection des espèces, 
ainsi qu’à la régulation des activités susceptibles d’impacter la biodiversité. Le \gls{seb} se situe au cœur 
des enjeux écologiques de l’île, notamment du fait de la présence d’une biodiversité exceptionnellement riche, 
en grande partie endémique, mais aussi particulièrement vulnérable.
\par\medskip

Mon stage s’est déroulé au sein de l’Unité Biodiversité (\gls{ubio}), unité en charge du suivi des espèces 
et des habitats naturels, de la gestion et de la valorisation des données naturalistes, ainsi que de 
l’instruction des dossiers réglementaires liés à la biodiversité. L’unité assure également l’animation du 
Système d’Information sur la Nature et les Paysages (\gls{sinp}) régional, la gestion de la plateforme 
\glsfirst{borbonica}, et intervient sur des thématiques transversales telles que les espèces exotiques envahissantes 
(\gls{eee}), la séquence \gls{erc} ou encore la diffusion des connaissances naturalistes.  
L’organisation interne de l’\gls{ubio} est présentée en annexe (Fig.~\ref{fig:organigramme-ubio}).
\par\medskip

L’organigramme interne montre une équipe pluridisciplinaire rassemblant des profils scientifiques, techniques 
et administratifs, travaillant de manière complémentaire pour répondre aux enjeux liés à la biodiversité du territoire. 
Mon stage s’inscrit directement dans cette dynamique, au sein du pôle dédié aux données naturalistes, en soutien à la 
structuration et à la modernisation des outils numériques utilisés par la \gls{deal}.
\par\medskip

Cette présentation de la \gls{deal} et de son organisation permet de situer le contexte global de mon stage.  
Le chapitre suivant propose un \textbf{diagnostic \gls{rse}} de la structure, afin d’évaluer ses pratiques 
au regard des enjeux sociaux, environnementaux et organisationnels.
\par\medskip
