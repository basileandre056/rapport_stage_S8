La Direction de l’Environnement, de l’Aménagement et du Logement (\gls{deal}) est le service déconcentré de l’État. 
Elle met en œuvre, à l’échelle régionale, les politiques publiques relevant du Ministère de la Transition 
Écologique et de la Cohésion des Territoires, ainsi que du Ministère de la Transition Énergétique. La Réunion 
est un territoire insulaire soumis à de fortes pressions environnementales et à des enjeux d’aménagement complexes. 
Dans ce contexte, la \gls{deal} occupe une place centrale. Elle se situe au croisement des questions d’environnement, 
de biodiversité, d’eau, d’urbanisme et de développement territorial.
\par\medskip

La \gls{deal} Réunion assure l’application des réglementations environnementales. Elle instruit les projets 
d’aménagement et gère les risques naturels. Elle suit la ressource en eau et met en œuvre les politiques de 
protection des milieux naturels. Elle travaille en étroite collaboration avec les collectivités et plusieurs 
établissements publics, tels que l’\gls{ofb}, le \gls{pnrun} ou l’\gls{ifremer}. Elle coopère 
aussi avec de nombreuses associations locales, comme la \gls{seor} ou la \gls{srepen}, qui contribuent activement 
au suivi et à la préservation de la biodiversité insulaire.
\par\medskip

Au sein de cette structure, le Service Eau et Biodiversité (\gls{seb}) porte les missions liées à la préservation 
des milieux aquatiques et terrestres. Il développe la connaissance des espèces, veille à leur protection et régule 
les activités susceptibles d’impacter la biodiversité. Le \gls{seb} se trouve ainsi au cœur des enjeux écologiques 
de l’île.
\par\medskip

Mon stage s’est déroulé au sein de l’Unité Biodiversité (\gls{ubio}). Cette unité est responsable du suivi 
des espèces et des habitats naturels. Elle gère et valorise les données naturalistes et instruit les dossiers
réglementaires liés à la biodiversité. Elle anime également le Système d’Information sur la Nature et les 
Paysages (\gls{sinp}) régional et assure la gestion de la plateforme \glsfirst{borbonica}. L’unité intervient 
aussi sur des thématiques transversales telles que les espèces exotiques envahissantes (\gls{eee}), la séquence 
\gls{erc} ou la diffusion des connaissances naturalistes. L’organisation interne de l’\gls{ubio} est présentée 
en annexe (Fig.~\ref{fig:organigramme-ubio}).
\par\medskip

L’organigramme interne montre une équipe composée de profils scientifiques, techniques et administratifs. 
Ces personnels travaillent de manière complémentaire pour répondre aux enjeux liés à la biodiversité du 
territoire. Mon stage s’inscrit dans cette dynamique, au sein du pôle dédié aux données naturalistes. Il 
contribue à la structuration et à la modernisation des outils numériques utilisés par la \gls{deal}.
\par\medskip

Cette présentation de la \gls{deal} et de son organisation permet de situer le contexte global de mon stage. 
Le chapitre suivant propose un \textbf{diagnostic \gls{rse}} de la structure. Il vise à évaluer ses pratiques 
au regard des enjeux sociaux, environnementaux et organisationnels.
\par\medskip