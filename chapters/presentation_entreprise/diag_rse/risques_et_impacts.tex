\noindent
\textbf{Impacts environnementaux}\\
La \gls{deal} n’engendre pas d’impacts industriels directs. Cependant, plusieurs de ses activités 
présentent des risques environnementaux indirects :

\begin{itemize}
    \item un usage intensif d’outils numériques (bases de données naturalistes, SIG, plateformes \gls{sinp}, \gls{geonature}, \gls{borbonica}), impliquant une consommation énergétique importante et une dépendance aux infrastructures serveurs ;
    \item le stockage et le traitement de volumes croissants de données, contribuant à l’empreinte carbone numérique et à la production de déchets électroniques liés au renouvellement du matériel ;
    \item des déplacements professionnels réguliers pour l’instruction de projets, le suivi de chantiers ou les missions de terrain, occasionnant des émissions de gaz à effet de serre.
\end{itemize}

Ces activités consomment de l’énergie, mobilisent des ressources non renouvelables et contribuent à l’empreinte environnementale globale du service.
\par\medskip


\textbf{Impacts sociétaux}\\
Les missions de la \gls{deal} comportent également plusieurs risques sociétaux :

\begin{itemize}
    \item une influence directe sur les décisions d’aménagement (urbanisme, infrastructures, gestion des risques), pouvant affecter l’organisation du territoire et la qualité de vie des habitants ;
    \item une responsabilité dans l’instruction des permis CITES et le suivi des espèces menacées, dont une mauvaise évaluation pourrait fragiliser le patrimoine naturel ;
    \item la coordination des actions de lutte contre les espèces exotiques envahissantes (\gls{eee}), où un retard ou un défaut d’intervention peut entraîner une dégradation durable des écosystèmes ;
    \item une forte dépendance aux partenaires (collectivités, \gls{ofb}, \gls{ifremer}, Parc national, associations), rendant certaines actions vulnérables à des manques de coordination ou de moyens.
\end{itemize}

Ces risques peuvent affecter la cohérence des politiques publiques, la préservation des milieux naturels et l’équité territoriale.
\par\medskip


\textbf{Impacts sociaux}\\
Le fonctionnement interne de la \gls{deal} comporte également des risques sociaux. Ses équipes 
travaillent dans un environnement pluridisciplinaire soumis à des contraintes organisationnelles et réglementaires :

\begin{itemize}
    \item une charge de travail variable selon les périodes, pouvant entraîner des tensions dans la répartition des tâches et la disponibilité des agents ;
    \item des missions de terrain présentant des risques pour la santé et la sécurité (déplacements fréquents, conditions difficiles, chantiers, zones accidentées) ;
    \item un besoin constant de mise à jour des compétences (réglementations, outils numériques, SIG), créant un risque d’obsolescence professionnelle pour les agents non formés ;
    \item un renouvellement régulier des stagiaires ou contractuels, pouvant fragiliser la continuité des projets ou l’organisation des équipes.
\end{itemize}

Ces éléments peuvent affecter la qualité de vie au travail, la sécurité en mission et la stabilité organisationnelle.
\par\medskip
