\noindent
\textbf{Impacts environnementaux.}
La \gls{deal} n’engendre pas d’impacts industriels directs. Cependant, ses activités reposent largement sur 
l’utilisation d’outils numériques intensifs : traitement de données naturalistes, alimentation des plateformes 
telles que le \gls{sinp}, \gls{geonature} ou \gls{borbonica}, ainsi que l’import automatisé de données issues de 
l’API \gls{quadrige}, permettant l’accès aux données marines produites par l'\gls{ifremer}. 
Ces pratiques impliquent une consommation énergétique, l’usage d’équipements informatiques et des besoins 
croissants en stockage et en traitement.  
\par\medskip

\textbf{Impacts sociétaux.}
Les décisions publiques s’appuient fortement sur la qualité des données produites et centralisées par la \gls{deal}. Une information environnementale fiable est essentielle pour les collectivités, les bureaux d’études, les associations ou les services de l’État.  
En facilitant l’intégration des données marines issues de \gls{quadrige}, mon travail renforce la transparence, l’égalité d’accès à la connaissance et la capacité des acteurs à prendre des décisions éclairées.
\par\medskip

\textbf{Impacts sociaux.}
Les équipes de la \gls{deal} évoluent dans un environnement de travail pluridisciplinaire mobilisant expertise scientifique, compétences réglementaires et gestion de données.  
Mon immersion au sein de l’\gls{ubio} m’a permis de contribuer à l’amélioration de certains processus internes, notamment par la simplification du flux de données marines. Cette contribution technique a eu pour effet indirect de diminuer la charge de travail liée aux imports manuels, renforçant ainsi l’efficacité opérationnelle du pôle.

\subsection*{Enquête RSE : actions mises en place}

La \gls{deal} engage plusieurs actions structurantes en cohérence avec la norme ISO~26000.

\par\medskip
\textbf{Environnement.}
L’administration encourage la réduction des déplacements, le recours aux outils numériques et une modernisation progressive des systèmes d’information. Mon module s’inscrit dans cette démarche de sobriété numérique en réduisant les traitements redondants et en automatisant les échanges de données.

\par\medskip
\textbf{Social et gouvernance.}
Les conditions de travail, la qualité du dialogue entre services et l’accompagnement des stagiaires constituent des axes importants. Les échanges réguliers au sein du pôle biodiversité ont permis d’adapter l’outil développé aux besoins réels, témoignant d’un fonctionnement concerté et d’une volonté d’amélioration continue.

\par\medskip
\textbf{Sociétal.}
La \gls{deal} contribue directement à la diffusion de données environnementales essentielles au suivi scientifique et aux politiques publiques. Les collaborations avec l'\gls{ofb}, l'\gls{ifremer}, le Parc national ou encore les associations naturalistes renforcent l’ancrage territorial de son action.  
Le module développé participe à cette dynamique en améliorant l’accessibilité et la qualité des données marines, dont dépend une partie de la stratégie environnementale régionale.
