\noindent
Bien qu’elle agisse dans un cadre réglementaire strict, la \gls{deal} cherche à renforcer la cohérence et la qualité de 
ses pratiques internes. Elle s’inscrit dans une dynamique de \textit{pré-conformité active}, allant au-delà des obligations 
minimales en matière d’organisation, de gestion de l’information et de modernisation numérique. Cette démarche vise à 
améliorer la fiabilité des données environnementales, à harmoniser les outils utilisés par les services et à garantir une 
application rigoureuse des réglementations.

Sur le plan sociétal, la \gls{deal} renforce ses méthodes de travail pour limiter les risques identifiés. Elle veille
à la transparence dans l’instruction des projets d’aménagement, à sécuriser les procédures liées aux permis CITES 
et à structurer davantage la coordination des actions de lutte contre les (\gls{eee}). 

Cette dynamique repose également sur une coopération accrue avec les collectivités territoriales, les établissements publics
et les associations naturalistes, afin de réduire les vulnérabilités 
liées à la gouvernance et à la circulation de l’information. Ces partenariats contribuent à une meilleure gestion des risques 
sociaux, environnementaux et sociétaux auxquels le territoire est confronté.
\par\medskip