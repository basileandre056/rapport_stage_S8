\noindent
As part of my engineering studies at ENIB, I completed an assistant engineer
internship. Building on coursework in computer science, electronics, data processing,
and the CRS module (Communication, Networks and Systems), I was particularly
interested in contributing my technical skills to projects related to biodiversity and
the marine environment.
\par\medskip

I carried out my internship on Réunion Island within DEAL Réunion, and more
specifically in the Water and Biodiversity Service (SEB), Biodiversity Unit (UBIO).
The island faces major biodiversity challenges due to its high number of endemic
species and the vulnerability of its natural habitats. These ecosystems are under
pressure from invasive alien species and land-use development, which makes
reliable and well-structured biodiversity data essential for environmental management.
DEAL uses the Géonature application to centralise and manage such data, and my
internship consisted in developing an external module to automatically import data
from Quadrige, Ifremer’s database dedicated to marine and coastal environmental
monitoring.
\par\medskip

This project allowed me to deepen my skills in software development, data
modelling and system interoperability, in continuity with the concepts introduced in
the CRS module. As I had not taken the CAI module, the internship also provided an
opportunity to develop complementary skills in application development, data
handling and the design of specialised tools through concrete operational needs.
\par\medskip

I also gained insight into the functioning of a public administration and the
organisation of a medium-sized institution. The internship offered a valuable balance
between autonomy and teamwork, particularly through exchanges with staff involved
in naturalist data management and the administration of Géonature.
\par\medskip

Overall, this internship provided substantial professional experience in software
development within a real-world context, while strengthening my understanding of
environmental data management. It represents an important step in consolidating my
technical skills and applying them to biodiversity monitoring and conservation.
