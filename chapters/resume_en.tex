\noindent
As part of my general engineering studies at \gls{enib}, I had the opportunity to explore
a wide range of fields, including computer science, data processing and network communications.
These courses, combined with the current environmental context, strengthened my interest in
understanding how digital tools can address concrete challenges — particularly those related
to the environment and the marine domain, which has long been a source of inspiration for me.
This motivation led me to choose \gls{deal},
where the development of digital solutions plays a key role in enhancing the value and accessibility
of environmental data, offering an ideal opportunity to align my technical skills with a topic
that holds personal meaning.
\par\medskip

I completed my internship within \gls{deal} Réunion, more specifically
in the \gls{seb}, within the \gls{ubio}.
Réunion Island faces major biodiversity challenges due to its high proportion of endemic species
and the presence of particularly sensitive natural habitats. These ecosystems are under significant
pressure, notably from \gls{eee} and ongoing land-use development, which reinforces the need
for rigorous and well-structured naturalist data.
\gls{deal} relies on \gls{geonature} to centralise and manage these datasets.
The primary objective of my internship was to develop an external module for this application,
allowing the automated import of marine and coastal environmental data produced by \gls{ifremer}.
Such data are essential for \gls{deal} agents in carrying out biodiversity monitoring missions.
They are accessed through \gls{quadrige}, which exposes a GraphQL-based \gls{api}.
\par\medskip

This project enabled me to apply and deepen my skills in software development,
data structuring and interoperability between heterogeneous systems, in continuity with the concepts
covered in the \gls{crs} module. Since I had not taken the \gls{cai} module, the internship also provided
an opportunity to develop complementary skills in application development, data management
and the design of specialised operational tools, grounded in concrete field requirements.
\par\medskip

I also gained insight into the functioning of a public administration and the specificities
of a medium-sized organisation. The internship offered a valuable balance between autonomy
and collaborative work, particularly through discussions with staff involved in naturalist data
management and the administration of \gls{geonature}.
\par\medskip

Overall, this internship provided significant professional experience in software development
within a real operational setting, while strengthening my understanding of the challenges associated
with environmental data management. It marks an important step in consolidating my technical skills
and applying them to biodiversity monitoring and conservation efforts.