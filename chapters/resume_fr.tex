\noindent
Au cours de ma formation d’ingénieur généraliste à l’\gls{enib}, j’ai eu l’opportunité d’explorer
des domaines variés, dont l’informatique, le traitement des données et la communication réseau.
Ces enseignements, combinés au contexte environnemental actuel, m’ont donné l’envie d’explorer
comment les outils numériques peuvent répondre à des enjeux concrets, en particulier ceux liés
à l’environnement et au milieu marin, un domaine qui m’inspire depuis longtemps.
C’est cette motivation qui m’a conduit à choisir la \gls{deal},
où le développement d’outils
informatiques joue un rôle clé dans la valorisation et l’accessibilité des données,
offrant une occasion idéale d’allier mes compétences techniques à un sujet qui a du
sens.

\par\medskip

J’ai effectué mon stage au sein de la \gls{deal} Réunion, plus précisément
au \gls{seb}, à l’\gls{ubio}.
L’île de La Réunion est un territoire qui présente des
enjeux majeurs de suivi et de préservation de la biodiversité, en raison de la forte proportion
d’espèces endémiques et de la présence d’habitats naturels particulièrement sensibles. Ces milieux
sont soumis à de fortes pressions, notamment liées aux \gls{eee} et au
développement du territoire, ce qui renforce la nécessité d’un suivi rigoureux des données
naturalistes. La \gls{deal} utilise \gls{geonature} pour centraliser et gérer ces données.
Le sujet principal de mon stage consistait à développer un module externe de cette application, permettant d’importer automatiquement
les données de l’\gls{ifremer}. Ces données sont dédiées au suivi de l’environnement marin et littoral, et sont donc très utiles
pour les agents de la \gls{deal} dans le cadre de leurs missions de suivi de la biodiversité.
Pour récupérer ces données, il faut passer par \gls{quadrige}, qui expose une \gls{api} au format GraphQL.

\medskip

N'ayant pas suivi le module \gls{cai}, ce stage m’a permis d’acquérir des compétences complémentaires en développement
logiciel et applicatif, à travers la conception d’un module d’import, la manipulation de données et l’interfaçage de
systèmes hétérogènes, afin d’obtenir un outil simple d’utilisation, robuste et maintenable.

\medskip

Il m’a également offert une immersion dans le fonctionnement d’une administration publique de taille moyenne,
où l’alternance entre travail autonome et collaboration avec les agents en charge des données naturalistes et de
l’administration de \gls{geonature} s’est révélée particulièrement enrichissante. Cette expérience en environnement
professionnel réel a renforcé ma compréhension des enjeux liés à la gestion des données environnementales et a affiné
ma capacité à questionner la pertinence, l’utilité et le sens des projets auxquels je contribue, afin de m’assurer qu’ils
soient en accord avec mes valeurs et porteurs d’un impact positif.