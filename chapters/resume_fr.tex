Dans le cadre de ma formation d’ingénieur à l’ENIB, je dois réaliser un stage en
tant qu’assistant ingénieur. Ce stage intervient à la suite de plusieurs
enseignements en informatique, électronique et traitement de données, ainsi que
de projets réalisés durant les semestres précédents. Naturellement, j’ai orienté ma
recherche de stage vers des structures en lien avec l’environnement marin, un
domaine qui m’intéresse tout particulièrement.

La DEAL Réunion utilise l’application Géonature pour la gestion et le suivi de ses
données naturalistes. L’opportunité de réaliser mon stage au sein de cette
structure représentait un cadre idéal pour mettre à profit mes compétences
techniques tout en contribuant à un outil déployé à grande échelle dans les
services de l’État. Le sujet proposé consistait à développer un module externe
permettant d’importer automatiquement des données issues de Quadrige dans
Géonature, un thème en parfaite adéquation avec mon intérêt pour le traitement
de données, les outils libres et l’informatique environnementale.

Ce projet m’a permis de mobiliser et d’approfondir mes connaissances en
développement logiciel, en structuration de données et en interfaçage entre
systèmes hétérogènes. J’ai ainsi pu travailler sur des problématiques concrètes
touchant aux formats d’échange, aux processus d’import automatisés et à
l’intégration d’un module dans une architecture existante.

J’ai également pu découvrir le fonctionnement d’une administration publique et
les spécificités organisationnelles d’une structure de taille moyenne. Le stage a
offert un équilibre intéressant entre autonomie et travail collaboratif, notamment
lors des échanges avec les agents impliqués dans la gestion des données ou
l’administration de Géonature.

Ce stage m’a donc apporté une expérience significative en développement dans un
contexte professionnel réel, tout en renforçant ma compréhension des enjeux liés
à la gestion de données environnementales. Il constitue une étape importante dans
la consolidation de mes compétences d’ingénieur polyvalent, à l’interface entre
informatique, data et environnement.
