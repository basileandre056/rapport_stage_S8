\noindent
Au cours de ma formation d’ingénieur généraliste à l’\gls{enib}, j’ai eu l’opportunité d’explorer
des domaines variés, dont l’informatique, le traitement des données et la communication réseau.
Ces enseignements, combinés au contexte environnemental actuel, m’ont donné l’envie d’explorer
comment les outils numériques peuvent répondre à des enjeux concrets, en particulier ceux liés
à l’environnement et au milieu marin, un domaine qui m’inspire depuis longtemps.
C’est cette motivation qui m’a conduit à choisir la \gls{deal},
où le développement d’outils
informatiques joue un rôle clé dans la valorisation et l’accessibilité des données,
offrant une occasion idéale d’allier mes compétences techniques à un sujet qui a du
sens pour moi.
\par\medskip

J’ai effectué mon stage au sein de la \gls{deal} Réunion, plus précisément
au \gls{seb}, à l’\gls{ubio}.
L’île de La Réunion est un territoire qui présente des
enjeux majeurs de suivi et de préservation de la biodiversité, en raison de la forte proportion
d’espèces endémiques et de la présence d’habitats naturels particulièrement sensibles. Ces milieux
sont soumis à de fortes pressions, notamment liées aux \gls{eee} et au
développement du territoire, ce qui renforce la nécessité d’un suivi rigoureux des données
naturalistes. La \gls{deal} utilise \gls{geonature} pour centraliser et gérer ces données.
Le sujet principal de mon stage consistait à développer un module externe de cette application, permettant d’importer automatiquement
les données de l’\gls{ifremer}. Ces données sont dédiées au suivi de l’environnement marin et littoral, et sont donc très utiles
pour les agents de la \gls{deal} dans le cadre de leurs missions de suivi de la biodiversité.
Pour récupérer ces données, il faut passer par \gls{quadrige}, qui expose une \gls{api} au format GraphQL.

\par\medskip

Ce projet m’a permis de mobiliser et d’approfondir mes connaissances en développement
logiciel, en structuration de données et en interfaçage entre systèmes hétérogènes, en continuité
avec les notions abordées en \gls{crs}. Ne suivant pas le module \gls{cai}, ce stage m’a également offert
l’occasion d’acquérir des compétences complémentaires en développement applicatif, gestion
de données et conception d’outils métiers, au travers de situations concrètes et de besoins
opérationnels.
\par\medskip

J’ai par ailleurs découvert le fonctionnement d’une administration publique et les spécificités
d’une structure de taille moyenne. Le stage a offert un équilibre enrichissant entre autonomie
et travail collaboratif, notamment lors des échanges avec les agents impliqués dans la gestion
des données naturalistes et l’administration de \gls{geonature}.
\par\medskip

Ce stage m’a ainsi apporté une expérience significative en développement dans un contexte
professionnel réel, tout en renforçant ma compréhension des enjeux liés à la gestion des données
environnementales. Il constitue une étape importante dans la consolidation de mes compétences
techniques, mises au service du suivi et de la préservation de la biodiversité.