\noindent
Dans le cadre de ma formation d’ingénieur à l’\gls{enib}, je dois réaliser un stage en tant
qu’assistant ingénieur. Il s’inscrit dans la continuité des enseignements suivis en informatique,
électronique, traitement de données, ainsi que du module \gls{crs}.
J’ai naturellement orienté ma recherche vers des structures en lien avec la biodiversité et
l’environnement marin, un domaine qui m’intéresse tout particulièrement.
\par\medskip

J’ai effectué mon stage à La Réunion, au sein de la \gls{deal} Réunion, plus précisément
au Service Eau et Biodiversité (\gls{seb}), Unité Biodiversité (\gls{ubio}). Ce territoire présente des
enjeux majeurs de suivi et de préservation de la biodiversité, en raison de la forte proportion
d’espèces endémiques et de la présence d’habitats naturels particulièrement sensibles. Ces milieux
sont soumis à de fortes pressions, notamment liées aux espèces exotiques envahissantes (\gls{eee}) et au
développement du territoire, ce qui renforce la nécessité d’un suivi rigoureux des données
naturalistes. La \gls{deal} utilise l’application \gls{geonature} pour centraliser et gérer ces données, et
le sujet de stage consistait à développer un module externe permettant d’importer automatiquement
des données issues de \gls{quadrige}, la base de données de l’\gls{ifremer} dédiée au suivi de
l’environnement marin et littoral.
\par\medskip

Ce projet m’a permis de mobiliser et d’approfondir mes connaissances en développement
logiciel, en structuration de données et en interfaçage entre systèmes hétérogènes, en continuité
avec les notions abordées en \gls{crs}. Ne suivant pas le module \gls{cai}, ce stage m’a également offert
l’occasion d’acquérir des compétences complémentaires en développement applicatif, gestion
de données et conception d’outils métiers, au travers de situations concrètes et de besoins
opérationnels.
\par\medskip

J’ai par ailleurs découvert le fonctionnement d’une administration publique et les spécificités
d’une structure de taille moyenne. Le stage a offert un équilibre enrichissant entre autonomie
et travail collaboratif, notamment lors des échanges avec les agents impliqués dans la gestion
des données naturalistes et l’administration de \gls{geonature}.
\par\medskip

Ce stage m’a ainsi apporté une expérience significative en développement dans un contexte
professionnel réel, tout en renforçant ma compréhension des enjeux liés à la gestion des données
environnementales. Il constitue une étape importante dans la consolidation de mes compétences
techniques, mises au service du suivi et de la préservation de la biodiversité.
